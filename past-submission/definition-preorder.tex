%% \begin{figure}[t]
%%   \hrulefill
%%   \begin{center}
%%   \begin{tabular}{cc}%@{\hskip 20pt}c}
%%     \begin{tikzpicture}
%%       \node (p) {$p$};
%%       \node[right=+40pt of p] (p') {$p'$};

%%       \path[->]
%%       (p) edge [bend left] node[above] {$a$}  (p')
%%       (p') edge [bend left] node[above] {$\co{a}$}  (p);
%%     \end{tikzpicture}
%%     &
%%     \begin{tabular}{l@{\hskip 4pt}c@{\hskip 0pt}l}
%%       \begin{tikzcd}
%%       p \arrow[r, "\co{\aa}"]
%%       &
%%       p' \arrow[d, "\aa"]
%%       \\
%%       &
%%       q
%%     \end{tikzcd}
%%     &$\Rightarrow$&\,\,
%%     $p \st{ \tau } q$ or $p = q$
%%     %% \begin{tikzcd}
%%     %%   p
%%     %%   \arrow[r, "\co{\aa}"]
%%     %%   \arrow[d, "\alpha"]
%%     %%   &
%%     %%   p' \arrow[d, "\alpha"]
%%     %%   \\
%%     %%   p'' \arrow[r, "\co{\aa}"]
%%     %%   &
%%     %%   q
%%     %% \end{tikzcd}
%%     \end{tabular}
%%     \\[5pt]
%%     \boom
%%   %%   \begin{tabular}{c}
%%   %%   \begin{tikzcd}
%%   %%     p \arrow[r, "\aa"]
%%   %%     &
%%   %%     p' \arrow[d, "\co{\aa}"]
%%   %%     \\
%%   %%     &
%%   %%     \equiv p
%%   %%   \end{tikzcd}
%%   %% \end{tabular}
%% &

%%     \fwdfeedback
%%   \end{tabular}
%%   \end{center}
%%   \caption{Axioms of forwarders.}
%%   \label{fig:axioms-forwarders}
%%     \hrulefill
%% \end{figure}




\section{Behavioural characterisations}
\label{sec:bhv-preorder}

We first recall the definition of the 
standard alternative preorder~$\asleq$,
and show how to use it to characterise~$\testleqS$ in our asynchronous setting.
Then we recall the other standard alternative preorder, namely $\msleq$,
and prove that it also captures~$\testleqS$, by applying our first characterisation.
%We obtain as corollary the equality of the \mustpreorder and failure refinement.



%% In this section we recall the definitions of the %two
%% standard alternative preorders~$\asleq$ and~$\msleq$.
%% Then we show how to use~$\asleq$ to characterise~$\testleqS$ in our setting%. We then use this result to
%% , and then rely on this result to
%% prove that also~$\msleq$ captures the \mustpreorder.

%We achieve the first result by showing that, merely by lifting LTSs of
%output-buffered agents with feedback to LTSs of forwarders,~$\asleq$ captures~$\testleqS$.

% for any LTS of output-buffered agents with feedback.
%In fact we show how to adapt the alternative preorder~$\asleq$ was introduced by \cite{DBLP:books/daglib/0066919}
%To characterise~$\testleqS$ %in~\CCS, and then
%we show that,


%% In the synchronous setting a standard fashion to capture the must preorder
%% is to compare how related processes converge along traces, and how their
%% potential deadlocks match. We clarify this last point.

\subsection{The acceptance-set approach} 
%{\bfseries The acceptance-set approach.} % 
The {\em ready set} of a program~$\server$ is defined as
\coqLTS{ready_set} $R( \server ) = I( \server ) \cup O( \server )$,
and it contains all the {\em visible} actions that~$\server$ can
immediately
% (i.e. strongly)
perform.  If a program~$\server$ is
stable, \ie it cannot perform any
% further step of computation ($\tau$-transition),
$\tau$-transition, we say that it is a {\em potential
  deadlock}.  In general, the ready set of a potential
deadlock~$\server$ shows how to make~$\server$ move to a different
state, possibly one that can perform further computation: if
$R( \server ) = \emptyset$ then there is no way to make~$\server$ move
on, while if~$R( \server )$ contains some action, then~$\server$ is
% essentially
a state waiting for the environment to interact with it.
Indeed, potential deadlocks are called {\em waiting states}
in~\cite{DBLP:conf/ecoop/HondaT91}.
In particular, in an asynchronous setting the outputs of a
% potentially deadlocked program~$\server$
potential deadlock~$\server$ show how~it %$\server$
can unlock the inputs
of a client, which in turn may lead the client to a novel state that
can make~$\server$ move, possibly to a state that can perform further
computation.
%This further motivates the intuition of a {\em potential} deadlock.
A standard manner to capture all the ways out of the potential
deadlocks that a program~$\server$ encounters after executing a
trace~$\trace$ is its {\em acceptance set}:
$\accP{ \server }{ \trace }{ \st{} } = \setof{ R( \stateA ) }{ \state \wt{ \trace } \stateA \stable }$.

In our presentation we indicate explicitly the third parameter of
$\mathcal{A}$, \ie the transition relation of the LTS at hand, because
when necessary we will manipulate this parameter.
For any two LTSs $\genlts_\StatesA, \genlts_\StatesB$ and servers
  $ \serverA \in \StatesA, \serverB \in \StatesB$, we 
  write $\accP{ \serverA }{ \trace }{\st{}_\StatesA} \ll \accP{
    \serverB }{ \trace }{ \st{}_\StatesB}$ %whenever
  if for every
  $R \in \acc{ \serverB }{ \trace }$ there exists
  $\widehat{R} \in \acc{ \serverA }{ \trace}$ such that $\widehat{R}
  \subseteq R$.
  We can now recall the definition of the behavioural preorder à la
  Hennessy, $\asleq$, which is based on acceptance
  sets~\cite{DBLP:books/daglib/0066919}.

  \newcommand{\cnvalongLTS}[2]{ \Downarrow_{#1} #2 }

   \begin{definition}\coqMT{bhv_pre}
  \label{def:accset-leq}%
  \label{def:standard-char}
%%   For every $\genlts_A, \genlts_B$, and servers $\serverA \in A, \serverB \in B$ we let
%%   $$
%%   \begin{array}{lcll}
%%     \serverA \bhvleqone \serverB & \text{whenever} & \forall \trace \in \Actfin \suchthat
%%     \serverA \cnvalongLTS{\StatesA}{ \trace } \implies \serverB \cnvalongLTS{\StatesB}{ \trace }
%%     \\[5pt]
%%     \serverA \bhvleqtwo \serverB  & \text{whenever} & \forall \trace \in \Actfin \suchthat
%%       \serverA \cnvalongLTS{\StatesA}{ \trace }  \implies \\
%%       && %\multicolumn{2}{l}{
%%         \accP{ \serverA }{ \trace }{ \st{}_{\StatesA} }
%%       \ll \accP{ \serverB }{ \trace }{ \st{}_{\StatesB} %}
%%     }
%%     \\[5pt]
%% \serverA \asleq \serverB & \text{whenever} & \serverA \bhvleqone
%% \serverB  \text{ and }  \serverA \bhvleqtwo \serverB & \blacksquare
%% \end{array}
  %%   $$
  {We write}
  \begin{itemize}
  \item
    $
    \serverA \bhvleqone \serverB$ whenever $\forall \trace \in \Actfin \suchthat
    \serverA \cnvalongLTS{\StatesA}{ \trace } \implies \serverB \cnvalongLTS{\StatesB}{ \trace }
    $,
    
  \item
    $
    \serverA \bhvleqtwo \serverB$ whenever $\forall \trace \in \Actfin \suchthat
    \serverA \cnvalongLTS{\StatesA}{ \trace }  \implies 
    \accP{ \serverA }{ \trace }{ \st{}_{\StatesA} }
    \ll \accP{ \serverB }{ \trace }{ \st{}_{\StatesB} }
    $,
  \item
    $\serverA \asleq \serverB$ whenever $\serverA \bhvleqone
    \serverB  \text{ and }  \serverA \bhvleqtwo \serverB$.\hfill$\blacksquare$
  \end{itemize}  
   \end{definition}

In the synchronous setting, the behavioural preorder~$\asleq$ is closely related to the denotational semantics
based on Acceptance Trees  proposed by Hennessy in
\cite{DBLP:journals/jacm/Hennessy85,DBLP:books/daglib/0066919}.
There the predicates need not be annotated with the LTS that they are used on,
because those works treat a unique LTS.
Castellani and Hennessy~\cite{DBLP:conf/fsttcs/CastellaniH98} show in their Example~4 that
the condition on acceptance sets, \ie $\bhvleqtwo$, is too demanding
in an asynchronous setting.
  %%%%%%%%%%%%%%%%%%%%%%%%%%%%%% OLD DEFINITION %%%%%%%%%%%%%%%%%%%%%%%%%%%%%%%%%%%%%%%%%%%%%%%
%%   \begin{definition}\coqMT{bhv_pre}
%%   \label{def:accset-leq}%
%%   \label{def:standard-char}
%%   For every $\genlts_A, \genlts_B$, and servers $\serverA \in A$
%%   and $\serverB \in B$ we let
%%   $$
%%   \begin{array}{lcll}
%%     \serverA \bhvleqone \serverB & \text{whenever} & \forall \trace \in \Actfin \suchthat
%%     \serverA \cnvalong{ \trace } \implies \serverB \cnvalong{ \trace }
%%     \\[5pt]
%%     \serverA \bhvleqtwo \serverB  & \text{whenever} \\
%%     & \multicolumn{2}{l}{\forall \trace \in \Actfin \suchthat
%%       \serverA \cnvalong{ \trace }  \implies \acc{ \serverA }{ \trace }
%%       \ll \acc{ \serverB }{ \trace }
%%     }
%%     \\[5pt]
%% \serverA \asleq \serverB & \text{whenever} & \serverA \bhvleqone
%% \serverB  \text{ and }  \serverA \bhvleqtwo \serverB & \blacksquare
%% \end{array}
%%   $$
%% \end{definition}
  %%%%%%%%%%%%%%%%%%%%%%%%%%%%%% OLD DEFINITION%%%%%%%%%%%%%%%%%%%%%%%%%%%%%%%%%%%%%%%%%%%%%%%

   %%% CORRECT AND IMPLEMENTED
%% \ilacom{In page 4 we wrote:\\
%%   ``To reason simultaneously on different LTSs, we will use the
%%   symbols~$\genlts_A$ and~$\genlts_B$ to denote respectively some
%%   LTS~$\lts{\StatesA}{L}{\st{}_A}$ and some
%%   LTS~$\lts{\StatesB}{L}{\st{}_B}$.'' \\
%%   So, if we want to define the preorder $\bhvleqtwo$ between two
%%   servers $\serverA \in A$ and $\serverB \in B$, we should use
%%   $\accP{ \server }{ \trace }{ \st{}_A }$ and
%%   $\accP{ q }{ \trace }{ \st{}_B }$ in the definition above. To be
%%   completely rigorous we should also add indices $A$ and $B$ to the
%%   respective convergence predicates, namely write
%%   $\server \cnvalong_A{\trace }$ and $q \cnvalong_B{\trace }$.}
%In particular in~\cite{DBLP:books/daglib/0066919}, Hennessy does not need to annotate the

%% \ilacom{In fact,
%%   in~\cite{DBLP:conf/fsttcs/CastellaniH98} we said that acceptance
%%   sets were too generous ;-)...  Indeed, acceptance sets are too
%%   generous in the sense that they include also inputs, while inputs
%%   should not be observable in an asynchronous setting.  It is the
%%   preorder $\bhvleqtwo$ that becomes too demanding when using such
%%   acceptance sets.} to capture the \mustpreorder in the asynchronous
%% setting.
%
%, \ie ${\testleqS} \not\subseteq {\asleq}$.
%% The argument is that while the equality $\Nil \testeqS \aa.\co{\aa}$ is
%% true, we have $\aa.\co{\aa} \bhvleqtwo \Nil$, and thus to
%% $\aa.\co{\aa} \asleq \Nil$.
%%\ilacom{??? Something wrong here. Besides, this is not Example~4.}
Letting $\serverA = \aa.\Nil$ and $\serverB = \Nil$, they show that
$\serverA \testleqS \serverB $ but $ \serverA \not \asleq \serverB$,
because $\acc{ \serverA }{ \epsilon } =\set{\set{a}}$ and
$\acc{ \serverB }{ \epsilon } =\set{\emptyset}$, and corresponding to
the ready set $\emptyset \in \acc{ \serverB }{ \epsilon }$ there is no
ready set $\widehat{R} \in \acc{ \serverA }{ \trace}$ such that
$\widehat{R} \subseteq \emptyset$. %
%
%%% NEW EXPLANATION THAT FOLLOWS ILARIA SUGGESTIONS
%
Intuitively this is the case because acceptance sets treat inputs and
outputs similarly, while in an asynchronous setting only outputs can
be tested.

%In the next subsection we present
%\ila{We next present} a technique that
%lets us nevertheless use~$\asleq$ to characterise~$\testleqS$.
Nevertheless~$\asleq$ characterises~$\testleqS$, if servers
are enhanced as with forwarding. We now introduce this concept.

%% We now \gb{present a} technique that is sufficient
%% for~$\asleq$ to characterise~$\testleqS$.


%% OLD EXPLANATION
%% Intuitively this is true because acceptance sets (1) treat inputs and
%% outputs similarly, while \ila{in an asynchronous setting} only outputs
%% can be tested; and (2) they disregard the role of interactions due to
%% spurious messages in the global mailbox.  \ilacom{The second point is
%%   cryptic. Moreover there has been no mention of the global mailbox so
%%   far.} We now define and discuss a technique that is sufficient to
%% for~$\asleq$ to characterise~$\testleqS$.


{\bfseries Forwarders.}
We say that an LTS~$\genlts$ is of output-buffered agents {\em with forwarding},
for short is \obaFW, if it satisfies all the axioms in
\rfig{axioms} except \outputfeedback, and also the two following axioms:% in \rfig{axioms-forwarders}.

\begin{equation}
  \label{eq:axioms-forwarders}
  \begin{tabular}{cc}%@{\hskip 20pt}c}
    \begin{tikzpicture}
      \node (p) {$p$};
      \node[right=+40pt of p] (p') {$p'$};

      \path[->]
      (p) edge [bend left] node[above] {$a$}  (p')
      (p') edge [bend left] node[above] {$\co{a}$}  (p);
    \end{tikzpicture}
    &
    \begin{tabular}{l@{\hskip 4pt}c@{\hskip 0pt}l}
      \begin{tikzcd}
      p \arrow[r, "\co{\aa}"]
      &
      p' \arrow[d, "\aa"]
      \\
      &
      q
    \end{tikzcd}
    &$\Rightarrow$&\,\,
    $p \st{ \tau } q$ or $p = q$

    \end{tabular}
    \\[5pt]
    \boom
&

    \fwdfeedback
  \end{tabular}
  \end{equation}

The~\boom axiom states a kind of input-enabledness property,
which is however more specific as it
stipulates that the target
state of the input should loop back to the source state via a
complementary output. This is the essence of the behaviour of a
forwarder, whose role is simply to pass on a message and then get
back to its original state.
The~\fwdfeedback axiom is a weak form of Selinger's~\outputfeedback
axiom, which is better understood in conjunction with the~\boom
axiom: if the sequence of transitions $p\st{\out{a}} p'\st{a}q$ in
the~\fwdfeedback axiom is taken to be the sequence of transitions
$p'\st{\out{a}} p\st{a}p'$ in the~\boom axiom, then we see that it
must be $q=p$ in the~\fwdfeedback axiom.  Moreover, no~$\tau$ action is issued when moving from $p$ to $q$, since no
synchronisation occurs in this case: the message is just passed on.


We mechanise all this via the typeclass
\lstinline!LtsObaFW!. The overall structure of our typeclasses to reason on LTSs is thus
  $\text{\lstinline!Lts!} \geq \text{\lstinline!LtsEq!} \geq \text{\lstinline!LtsOba!}$ and
  $\text{\lstinline!LtsOba!}$ is a super-class of both \lstinline!LtsObaFB! and \lstinline!LtsObaFW!.
%% \begin{center}
%%   \scalebox{.9}{%
%%     \begin{tikzpicture}
%%       \node (Lts) {\lstinline!Lts!};
%%       \node (LtsEq) [right =+7pt of Lts]  {\lstinline!LtsEq!};
%%       \node (LtsOba) [right =+7pt of LtsEq]  {\lstinline!LtsOba!};
%%       \node (LtsFb) [above right=+7pt and +1pt of LtsOba]  {\lstinline!LtsObaFb!};
%%       \node (LtsFw) [below right=+7pt and +1pt of LtsOba]  {\lstinline!LtsObaFw!};
%%       \path[->]
%%       (Lts) edge (LtsEq)
%%       (LtsEq) edge (LtsOba)
%%       (LtsOba) edge (LtsFb)
%%       (LtsOba) edge (LtsFw);
%%     \end{tikzpicture}
%%   }
%% \end{center}
%which %
%we show in \rfig{structure-typeclasses-lts},
%along with the overall structure of our typeclasses to reason on LTSs.
We defer the details to \rapp{coq}.


To prove that~$\asleq$ is sound and complete with respect to~$\testleqS$:
\begin{enumerate} \item we define an operation to lift any LTS~$\genlts \in \obaFB$
into a suitable LTS~$\genlts_{\fw} \in \obaFW$, and \item we check the
predicates~$\cnvalong$ and~$\accP{-}{-}{-}$ over the LTS~$\genlts_{\fw}$.
\end{enumerate}
%\begin{figure}



Let~$\MO$ denote the set of all finite multisets of output actions, 
for instance we have $\varnothing, \mset{ \co{ a } }, \mset{ \co{ a },   \co{
    a }  }, \mset{ \co{ a },  \co{ b },  \co{ a },  \co{ b }} \in
\MO$.
%Similarly, let $\MI$ denote the set of all finite multisets of names.
%\ilacom{Why are the elements of $\MI$  required to be finite while those of $\MO$ are not?}
We let %$\I, \J, \ldots$ range over $\MI$ and
$M, N, \ldots$ range over~$\MO$. The symbol~$M$ stands for {\em mailbox}.
We denote with~$\uplus$ the multiset union.
We assume a function $\outputmultisetSym : A \rightarrow MO$ defined for any
LTS $\genlts_A$ of output-buffered agents such that
\begin{enumerate}[(i)]
\item
  $\co{\aa} \in O(\server)$ if and only if $\co{\aa} \in \outputmultiset{\server}$, and
\item
  for every $\server'$, if $\server \st{\co{\aa}} \server'$ then $\outputmultiset{\server} = \mset{\co{\aa}} \uplus \outputmultiset{\server'}$.
\end{enumerate}
Note that by definition $\outputmultiset{ \server }$ is a finite multiset.

\begin{definition}
  \label{def:sta}
  \label{def:liftFW}\coqLTS{lts_a}
  Let $\liftFW{\genlts} = \lts{\States \times \MO}{L}{\sta{}}$ for every LTS $\genlts = \lts{\States}{L}{\st{}}$,
%  For every LTS $\genlts = \lts{\States}{L}{\st{}}$
%  we let
%$$
%\liftFW{\genlts} = \lts{\States \times \FinMultiset{L}}{L}{\sta{}}
%$$
where the states in $\liftFW{\genlts}$ are pairs denoted $p
\triangleright \mailbox{M}$, such that $p \in \States$ and $M \in \MO$,
%is a finite multiset of outputs,
and the transition relation~$\sta{}$ is defined via the rules in
\rfig{rules-liftFW}.\hfill$\blacksquare$
\end{definition}

  



\begin{figure}
\hrulefill
  $$
  \begin{array}{llll}
    \begin{prooftree}
      \server \st{\alpha} \server'
      \justifies
      \server \triangleright M \sta{\alpha} \server' \triangleright M
    \end{prooftree}
  &
  \begin{prooftree}
    \server \st{a} \server'
    \justifies
    \server \triangleright (\mset{\co{a}} \uplus M) \sta{\tau} \server' \triangleright M
  \end{prooftree}
  \\[2em]
  \begin{prooftree}
    \justifies
    \server \triangleright (\mset{\co{a}} \uplus M) \sta{\co{a}} \server \triangleright M
  \end{prooftree}
  &
  \begin{prooftree}
    \justifies
    \server \triangleright M \sta{a} \server  \triangleright (\mset{\co{a}} \uplus M)
  \end{prooftree}
  &&
  \end{array}
  $$
  \caption{Lifting of a transition relation to transitions of forwarders.}
  \label{fig:rules-liftFW}
  \hrulefill
\end{figure}


\begin{example}
  \label{ex:forwarders-in-TACCS}
  If a calculus is fixed, then the function~$\liftFWSym$ may have a
  simpler definition.  For instance Castellani and
    Hennessy~\cite{DBLP:conf/fsttcs/CastellaniH98} define it in their
  calculus~\textsf{TACCS} by letting $\sta{\alpha}$ be the least
  relation over~\textsf{TACCS} such that
  %\item\label{pt:sta-inclusion}
  (1) for every $\alpha\in\Acttau \wehavethat \st{\alpha} {} \subseteq {} \sta{\alpha}$, and 
  %\item\label{pt:sta-asynch-input} 
  (2) for every $\aa \in \Names \wehavethat p \sta{\aa} p \Par \out{\aa}$.\hfill$\qed$
\end{example}



%%%%%%%%%%%%%%%%%%%%%%%%%%%%%%%%%%%%%%%%%%%%%%%%%%%%%%%%%%%%%
%%%%%%%%%%%%%%%%%%%%%%%%%%%%%%%%%%%%%%%%%%%%%%%%%%%%%%%%%%%%%
%%% THIS IS FOR THE THESIS
%%%%%%%%%%%%%%%%%%%%%%%%%%%%%%%%%%%%%%%%%%%%%%%%%%%%%%%%%%%%%
%%%%%%%%%%%%%%%%%%%%%%%%%%%%%%%%%%%%%%%%%%%%%%%%%%%%%%%%%%%%%
\leaveout{
NOTE. In this paragraph I explain the issues raised by defining the lifting $\genlts_\fw$
of an LTS $\genlts$ by its composition with the LTS $\Fwd_L = \lts{\MO}{L}{\st{}_{\mathit{mb}}}$
where
  \[
    M \uplus \{ \bar a \} \st{\co{a}} M
    \qquad\text{and}\qquad
    M \st{a} M \uplus \{ \bar a \}
  \]

From now, we assume that $\genlts_\fw = \genlts \times \Fwd_L$.
I will illustrate the issue using $\genlts = \lts{\ACCS}{\st{}}{\Acttau}$.

The keypoint behind the issue lies in the following question: should we consider
that $\co{a} \triangleright \varnothing \equiv 0 \triangleright \mset{\co{a}}$ ?

Let us first consider the case in which this equation is true.
First, observe that $0 \triangleright \mset{\co{a}} \stable$, which does not hold
for $\co{a} \triangleright \varnothing$ as
we can exhibit the following transition due to an interaction between $\co{a}$ and $\varnothing$.
$$
\begin{prooftree}
  \co{a} \st{\co{a}} 0 \hspace{1em} \varnothing \st{a}_{\mathit{mb}} \mset{\co{a}}
  \justifies
  \co{a} \triangleright \varnothing \sta{\tau} 0 \triangleright \mset{\co{a}}
\end{prooftree}
$$

From these two facts we have a counter example to \rlem{harmony-sta} as
$\co{a} \triangleright \varnothing \equiv 0 \triangleright \mset{\co{a}}$, and
$\co{a} \triangleright \varnothing \sta{\tau}$, but $0 \triangleright \mset{\co{a}} \stable$.
More generally, we have that the equivalence relation does not preserve stability.

We now consider the case in which this equation is false.
The output-commutativity axiom states that if
$\server \sta{\co{a}} \server_1$ and
$\server \sta{\co{a}} \server_2$ then $\server_1 = \server_2$.
We recall that in, our settings, we reason up-to equivalence between states, and thus
it should be the case that $\server_1 \equiv \server_2$.

Note that $\co{a} \triangleright \mset{\co{a}} \sta{\co{a}} 0 \triangleright \mset{\co{a}}$
and $\co{a} \triangleright \mset{\co{a}} \sta{\co{a}} \co{a} \triangleright \varnothing$.
However, by considering $\co{a} \triangleright \varnothing \nequiv 0 \triangleright \mset{\co{a}}$
we have that $\genlts_\fw$ does not obey to the output-commutativity axiom.
}
%%%%%%%%%%%%%%%%%%%%%%%%%%%%%%%%%%%%%%%%%%%%%%%%%%%%%%%%%
%%%%% END OF LEAVEOUT
%%%%%%%%%%%%%%%%%%%%%%%%%%%%%%%%%%%%%%%%%%%%%%%%%%%%%%%%%
%% \leo{%                                                                           %%
%%   to work with LTSs we define the lifting in a more abstract manner, by          %%
%%   composition with an LTS~$\Fwd_L =                                              %%
%%   \lts{\FinMultiset{L}}{L}{\st{}_{\mathit{mb}}}$ of mailboxes, whose states      %%
%%   are finite multisets~$M$ of messages, and with two types of transitions:       %%
%%   %                                                                              %%
%%   \[                                                                             %%
%%     M \uplus \{ \bar a \} \st{\co{a}} M                                          %%
%%     \qquad\text{and}\qquad                                                       %%
%%     M \st{a} M \uplus \{ \bar a \}                                               %%
%%   \]                                                                             %%
%%   %                                                                              %%
%%   The lifting~$\genlts_\fw$ of an LTS $\genlts = \lts{\States}{L}{\st{}}$ is     %%
%%   then defined as the product LTS: $\genlts \times \Fwd_L$, whose states are     %%
%%   pairs which we write $p \triangleright \mailbox{M}$ where $p \in \genlts$ is a %%
%%   server, and $M$ is a finite multiset of outputs.                               %%
%% }                                                                                %%

The transition relation $\sta{}$ is reminiscent of the one introduced
in Definition 8 by Honda and
  Tokoro in~\cite{DBLP:conf/ecoop/HondaT91}.  The construction given in
our \rdef{liftFW}, though, does not yield the LTS of Honda and Tokoro,
as $\sta{}$ adds the forwarding capabilities to the states only at the
top-level, instead of descending structurally into terms. As a
consequence, in the LTS of~\cite{DBLP:conf/ecoop/HondaT91}
$\aa.\Nil + \Nil \st{ \ab } \co{ \ab }$, while
$\aa.\Nil + \Nil \Nsta{ \ab } \co{ \ab }$.


\begin{example}
  As the set~$\Names$ is countable, every process~$\server$ in
  the LTS $\lts{\ACCS \times \MO}{\Acttau}{\sta{}}$ is
  infinitely-branching, %
  for instance for every $\server$ and every input $\mu$ we have %
    $ \server \st{\mu} \server \Par \co{\mu}$, %
    hence $\server \st{\aa_0} \server \Par \co{\aa_0}$,
    $\server \st{\aa_1} \server \Par \co{\aa_1}$,
    $\server \st{\aa_2} \server \Par \co{\aa_2}$, \ldots
  \hfill$\qed$
    %%% FOR THE JOURNAL VERSION
%as illustrated in the picture below,
%%    where we show only the part of the LTS that is added
%%    by the lifting to $\sta{}$.
    %% \ilacom{maybe we should
    %% say that the we show only the part of the LTS that is added
    %% by the lifting to $\sta{}$.}
%% \begin{center}
%%   \scalebox{.7}{%
%%   \begin{tikzpicture}
%%   \node[state](t){$\server \triangleright \varnothing$};

%%   \node[state][below=  of t](p20){$\server \triangleright \mailbox{\co{\aa_2}}$};
%%   \node[state][below=  of p20] (nil){$\server \triangleright \varnothing$};

%%   \node[state][left=of p20] (p10) {$\server \triangleright \mailbox{\co{\aa_1}}$};

%%   \node[state][left =of p10] (p00) {$\server \triangleright \mailbox{\co{\aa_0}}$};

%%   \node[state][right=of p20](p30){$\server \triangleright \mailbox{\co{\aa_3}}$};

%%   \node[][right=of p30](p5){\vdots\ldots\vdots};

%% %Edges
%% \path (t) edge[to] node[action,swap] {$\aa_0$} (p00)
%%       (t) edge[to] node[action] {$\aa_1$} (p10)
%%       (t) edge[to] node[action] {$\aa_2$} (p20)
%%       (t) edge[to] node[action,yshift=-8pt] {$\aa_3$} (p30)
%%       (t) edge[to] node[action] {$\aa_4$} (p5);

%% \path
%% (p00)  edge[to] node[action, swap] {$\co{\aa}_0$} (nil)
%% (p10)  edge[to] node[action] {$\co{\aa}_1$} (nil)
%% (p20)  edge[to] node[action] {$\co{\aa}_2$} (nil)
%% (p30)  edge[to] node[action, right] {$\co{\aa}_3$} (nil)
%% (p5)  edge[to] node[action] {$\co{\aa}_4$} (nil);
%%   \end{tikzpicture}
%%   }
%% \end{center}
    %% \vspace{-12pt}
\end{example}

%\gb{TODO ILARIA, check and possibly improve the explanation ?}
The intuition behind \rdef{sta} %\rdefpt{sta}{sta-asynch-input} \ilacom{I see no point $(ii)$ in this definition}
is that, when a client interacts with a server asynchronously, the
client can send any message it likes, regardless of the inputs that
the server can actually perform. In fact, asynchronous clients behave
as if the server was saturated with \emph{forwarders}, namely
processes of the form $\aa. \co{\aa}$, for any $\aa\in\Names$.


\noindent
We are ready to state two main properties of the
function~$\liftFWSym$: it lifts any LTS of output-buffered agents with feedback
to an LTS of forwarders, and the lifting preserves the~$\opMust$ predicate. We can therefore
reason on~$\testleqS$ using LTSs of forwarders.% (\rcor{testleq-obafb-iff-testleq-obafw}).

\begin{lemma}
  \label{lem:liftFW-works}
  For every LTS~$\genlts \in \obaFB$, $\liftFW{\genlts} \in \obaFW$.
\end{lemma}
%% \begin{proof}
%%   \TODO{Paul, WRITE ME}
%% \end{proof}


\begin{lemma}
  \label{lem:musti-obafb-iff-musti-obafw}
  For every $\genlts_A, \genlts_B, \genlts_C \in \obaFB, \serverA \in A, \serverB \in B, \client \in C$,
  \begin{enumerate}
    \item
      $\musti{\server}{\client}$ if and only if $\musti{\liftFW{\server}}{\client}$,
    \item
      $\serverA \testleqS \serverB$ if and only if $\liftFW{\serverA} \testleqS \liftFW{\serverB}$.
  \end{enumerate}
\end{lemma}

%% \begin{corollary}
%%   \label{cor:testleq-obafb-iff-testleq-obafw}
%%   For every $\genlts_A, \genlts_B  \in \obaFB, \serverA \in A, \serverB \in B$,
%%   $\serverA \testleqS \serverB$ if and only if $\liftFW{\serverA} \testleqS \liftFW{\serverB}$.
%% \end{corollary}


%\TODO{Why are ObaFW useful if must is logically equivalent to obaFB?}


%%%%%%%%%%%%%%%%% DEFINITION ABSTRACTION FOR ALTERNATIVE PREORDER
%% USELESS
%% Surprisingly, two predicates suffice to define our first alternative
%% preoder.  First, we define convergence along traces on the LTS of
%% Honda and Tokoro, \ie the predicate~$\acnvalong$, exactly
%% as~$\cnvalong$ but based on the transition relation~$\sta{}$.

%% The second abstraction that we need is an asynchronous variant
%% of the {\em acceptance sets}
%% of~\cite{DBLP:journals/jacm/Hennessy85}:%
%% (\coqMT{acceptance_sets}):

%% NOW USELESS
%We define weak transitions~$\wta{s}$~in the standard inductive way,% (\coqLTS{weak_a}),
%\ie via the same rules used to define~$\wt{s}$.
We now simplify the definition of acceptance sets to reason on
forwarders: %LTSs that are \obaFW:
for any two LTS $\genlts_\StatesA, \genlts_\StatesB \in \obaFW$ and servers
$ \serverA \in \StatesA$, and $\serverB \in \StatesB$  we let
$\accfwp{ \state }{ \trace }{ \st{} } =  \setof{ O(\state') }{ \state
  \wt{ \trace } \state' \stable }$.
This definition suffices to characterise~$\testleqS$ because in each LTS that is \obaFW\ every state performs
every input, thus comparing inputs has no impact on the
preorder~$\bhvleqtwo$ of \rdef{standard-char}. More formally,
for every $\genlts_\StatesA, \genlts_\StatesB \in \obaFW$ and every $\server \in \StatesA$ and $\serverB \in \StatesB$,
we let %$\serverA \asleqAfw \serverB \text{ iff } \text{ iff }$
$$
\serverA \asleqAfw \serverB \text{ iff } \forall \trace \in \Actfin \wehavethat \serverA \cnvalong \trace \implies \accfwp{ \serverA }{ \trace }{ \st{}_\StatesA } \ll \accfwp{ \serverB }{ \trace }{ \st{}_\StatesB }
$$
Then we have the following logical equivalence.
\begin{lemma}
  \label{lem:conditions-on-accsets-logically-equivalent}
  Let $\genlts_A, \genlts_B \in \obaFW$.
  For every $\serverA \in \StatesA, \serverB \in \StatesB, \serverA \bhvleqtwo \serverB$
  if and only if $\serverA \asleqAfw \serverB$.
\end{lemma}
\begin{proof}
  The {\em only if} implication is trivial, so we discuss the {\em if}
  one. Suppose that $ \serverA \asleqAfw \serverB  $ and that for some
  $\trace$ we have that $R \in \accP{\serverB}{s}{\st{}_B}$. Let~$X$ be
  the possibly empty subset of~$R$ that contains only output actions.
  Note that since~$\genlts_B$ is~\obaFW\ we know by definition that $R =
  X \cup \Names$.
  By definition $X \in \accfwp{\serverB}{s}{\st{}_B}$, and thus by
  hypothesis there exists some set of output actions $Y \in
  \accfwp{\serverA}{s}{\st{}_A}$ such that $Y \subseteq X$.
  It follows that the set $Y \cup \Names \in
  \accP{\serverA}{s}{\st{}_A}$, and trivially $Y \cup \Names \subseteq
  X \cup \Names = R$.
\end{proof}


  %%%%%%%%%
  %%%%%%%%%
  %%%%%%%%% END OF \gb



%%% ACCEPTANCE SETS EXPLAINED EARLIER FOR THE STANDARD ALTERNATIVE
%%% CHAR BY MATTHEW
%% If~$O( \server ) = \emptyset$ then~$\server$ can only wait for an
%% output sent by the client. Acceptance sets then capture, in spite of
%% \nondeterminismT, all the outputs that potential deadlock states in
%% the server offer to the environment (\ie the client).

%%% ARE THESE NEEDED ?
\renewcommand{\serverA}{p}
\renewcommand{\serverB}{q}

%% We are now ready to define our first characterisation.
%% \begin{definition}[Behavioural preorder]%\coqMT{bhv_pre}]
%%   \label{def:bhv-leq}%
%%   We write $ \serverA \bhvleq \serverB $ whenever
%%   $\serverB \bhvleqone
%%   \serverA  \wedge  \serverA \asleqAfw \serverB$.\hfill$\blacksquare$
%% \end{definition}


In view of the second point of \rlem{musti-obafb-iff-musti-obafw},
to prove completeness it suffices to show that~$\asleq$
includes~$\testleqS$ in the LTS of
forwarders. This is indeed true:
\begin{lemma}
  \label{lem:completenessA}
  For every $\genlts_A, \genlts_B \in \obaFW$ and
  servers $\serverA \in \StatesA, \serverB \in \StatesB $,
  if ${ \serverA } \testleqS { \serverB }$
  then ${ \serverA } \asleq { \serverB }$.
\end{lemma}


%{\em Notation:}
%To simplify the statements of our theorems we slightly abuse the
%notation.
%With a slight
By a slight abuse of notation,
given an LTS $\genlts = \lts{\States}{L}{\st{}}$ and a state
$\server \in \States$,
we denote with $\liftFW{
  \server }$ the LTS rooted at $\server \triangleright \mailbox{ \emptyMset }$ in $\liftFW{\genlts}$.


\begin{theorem}%% \coqEq{ctx_iff_bhv} %%
  \label{thm:testleqS-equals-bhvleq}
  \label{thm:testleqS-equals-accleq}
  \label{thm:testleqS-equals-asleq}
  For every $\genlts_A, \genlts_B \in \obaFB$
  and $\serverA \in \States, \serverB \in \StatesB, \,
  \serverA \testleqS \serverB$ if and only if
  $\liftFW{\serverA} \asleq \liftFW{\serverB}$.
\end{theorem}
%% \begin{proof}
%%   We discuss the steps of the proof
%%   in \rsec{bhv-completeness}-\ref{sec:bhv-soundness}.
%% \end{proof}

The proof of completeness is given in \rapp{proof-completeness},
where the main aim is to show \rlem{completenessA}.
The proof of soundness, instead, requires much more
auxiliary machinery than the one used to state \rlem{completenessA},
so we defer it entirely to \rapp{proof-soundness}.
Here we highlight the major novelty with respect to the literature, via a little digression.
All the soundness arguments for behavioural characterisations of
$\testleqS$ in non-deterministic settings,
for instance~\cite{DBLP:journals/tcs/NicolaH84,DBLP:journals/jlp/Hennessy05,DBLP:journals/fac/HennessyI93,DBLP:journals/iandc/BorealeN95,DBLP:journals/corr/BernardiH15}
but to cite a few, are rooted in classical logic, because they
(1) unzip %potentially infinite
maximal computations of 
$\csys{ \server }{ \client }\st{}\cdots$  %(which cannot be done
%constructively)
to produce traces $ \server \wt{ \trace } $ and $ \client
\wt{ \co{ \trace } } $ that may be infinite;
%%  \item\label{pt:excluded-middle}
(2) use the excluded middle on an undecidable property,
namely the infinity of the traces at hand; and 
%\item\label{pt:koenigs}
(3) in case of infinite traces apply \koenigslemma
(see for instance lemmas 4.4.12 and 4.4.13 of~\cite{DBLP:books/daglib/0066919}).
Our proof replaces \koenigslemma with induction and works
on infinite branching STS. This is possible thanks to the \barinduction\ principle,
which we outline in \rsec{barinduction-main-body}.

% An immediate consequence of~\rlem{ACCS-obaFB} and
% ~\rthm{testleqS-equals-accleq} is a characterisation of~$\testleqS$
% for~$\ACCS$:

From~\rlem{ACCS-obaFB} and~\rthm{testleqS-equals-accleq} we
immediately get a characterisation of~$\testleqS$ for~$\ACCS$:
\begin{corollary}
  \label{cor:characterisation-for-aCCS}
For every $\serverA, \serverB \in \modulo{\ACCS}{\equiv}, \serverA \testleqS \serverB$ if and only if $\liftFW{\serverA} \asleq
\liftFW{\serverB}.$
\end{corollary}

In \rapp{normal-form} we present what, to the best of our knowledge,
are the first behavioural characterisations of the \mustpreorder that
fully exploit asynchrony, \ie disregard irrelevant (that is, non-causal)
orders of visible actions in traces. Due to space constraints, here
we omit these additional results.

%This is possible thanks to the \outputcommutativity axiom.
%, and an analogous property for inputs that forwarders enjoy.
%%%% NO LONGER USEFULL
%% These alternative preorders use finite sequences of (pairs of)
%% multisets of inputs and multisets of outputs \ie sequences $
%% (\I_1,M_1) \ldots (\I_n,M_n)$, instead of standard sequences of actions.




\newcommand{\msleqtwo}{\preccurlyeq_{\textsc{m}}}

\subsection{The \MustSet approach}
As first application of \rthm{testleqS-equals-asleq}, we prove that the
second standard way to characterise the preorder~$\testleqS$, \ie
the one based on \MustSets, is indeed sound and complete.

\renewcommand{\after}[3]{ (#1 \, \mathsf{after} \,  #2, #3) }

For every~$X \subseteq_{\mathit{fin}} \Act$, that is for every finite set of visible actions,
with a slightly abuse of notation we write $\server \mathrel{\opMust} X$ whenever $\server \wt{ \varepsilon } \server'$ implies that $\server' \st{ \mu }$ for some $\mu \in X$, and we say that~$X$ is a \MustSet of~$\server$.
Let $\after{ \serverA }{ s }{ \st{} } = \setof{ \serverA' }{ \serverA \wt{s} \serverA' }$.
For every $\genlts_\StatesA, \genlts_\StatesB$ and $\serverA \in \StatesA, \serverB \in \StatesB$,
let $\serverA \msleqtwo \serverB$ whenever
$\forall \trace \in \Actfin$ we have that $\serverA \cnvalong{ \trace
}$ implies that $(\forall X \subseteq_{\mathit{fin}} \Act$ if
$\after{\serverA}{\trace}{ \st{}_\StatesA } \mathrel{\opMust} X$ then
$\after{\serverB}{\trace}{ \st{}_\StatesB } \mathrel{\opMust} X).$



\begin{definition}
  \label{def:denicola-char}
  For every $\genlts_A, \genlts_B \in \obaFB$, and server $\serverA
  \in A$ and $\serverB \in B$ we let $\serverA \msleq \serverB$ whenever
  $\serverA \bhvleqone \serverB \wedge
  \serverA \msleqtwo \serverB$.\hfill$\blacksquare$
\end{definition}


\begin{lemma}
  \label{lem:acceptance-sets-and-must-sets-have-same-expressivity}
  Let $\genlts_A, \genlts_B \in \obaFB$.
  For every $\serverA \in \StatesA, \serverB \in \StatesB $ such that $\liftFW{\serverA} \bhvleqone \liftFW{\serverB}$,
  we have that
  $\liftFW{\serverA} \msleqtwo \liftFW{\serverB}$ if and only if
  $\liftFW{\serverA} \asleqAfw \liftFW{\serverB}$.
\end{lemma}
\noindent
As a direct consequence, we obtain our second result.
\begin{theorem}
  \label{thm:testleqS-equals-mustsetleq}
    Let $\genlts_A, \genlts_B \in \obaFB$.
  For every $\serverA \in \States$  and
  $\serverB \in \StatesB $, we have that
  $\serverA \testleqS \serverB$ if and only if
  $\liftFW{\serverA} \msleq \liftFW{\serverB}$.
\end{theorem}
%% \begin{proof}
%%   It suffices to prove that $ \serverA \preccurlyeq_{\textsc{ms}} \serverB $ if and only if
%%   $ \serverA \preccurlyeq_{\textsc{m}} \serverB $. This is \rlem{acceptance-sets-and-must-sets-have-same-expressivity}.
%% \end{proof}


\newcommand{\failleq}{\mathrel{\leq_{\textsf{fail}}}}

{\bfseries Failure refinement.} %
\MustSets\ have been used mainly
by De Nicola and collaborators, for instance
in~\cite{DBLP:journals/lmcs/NicolaM23,DBLP:journals/iandc/BorealeNP02},
and are closely related to the failure refinement proposed
in~\cite{DBLP:journals/jacm/BrookesHR84} by Hoare, Brookes and Roscoe for TCSP (the process
algebra based on Hoare's language CSP
\cite{DBLP:journals/cacm/Hoare83a,DBLP:conf/icalp/Brookes83}).
Following~\cite{DBLP:journals/jacm/BrookesHR84},  
a {\em failure} of a process $\server$ is a pair $(\trace, X)$
such that $p \wt{ \trace } p'$ and $p' \Nst{\mu}$ for all $\mu \in X$.
Then, failure refinement is defined by letting $\serverA\failleq\serverB$
whenever the failures of~$\serverB$ are also failures of $\serverA$.
%This refinement is the staple of the community focused on 
%its variants have been applied to
%verify the software in the Russian module of the International Space Station 
%\cite{DBLP:conf/birthday/PeleskaB99,DBLP:conf/amast/ButhKPS97,DBLP:conf/amast/ButhPS98},
%Practitioners using failure refinements rely on the tool FDR
%\cite{DBLP:journals/sttt/Gibson-Robinson16}, and a number of
This refinement was designed to give a denotational semantics to
processes, and mechanisations in Isabelle/HOL have been
developed to ensure that the refinement
is well defined~\cite{HOL-CSP-AFP,DBLP:journals/acta/BaxterRC22}.  Both
Hennessy~\cite[pag. 260]{DBLP:books/daglib/0066919}
and~\cite{Castellan2023}
% \cite{DBLP:journals/corr/abs-2108-10558}
highlight that the failure
  model can be justified operationally via the~$\opMust$ testing
  equivalence: it is folklore dating back to~\cite[Section
    4]{DBLP:journals/tcs/NicolaH84} that failure equivalence and~$\testeq$ coincide.
  Thanks to \rthm{testleqS-equals-mustsetleq} we
  conclude that in fact~$\testleqS$ 
  coincides with~$\failleq$ in conjunction with~$\bhvleqone$.\footnote{The preorder becomes 
  then the ``failure divergence'' refinement formalised as
  $\sqsubseteq_{\text{\tt FD}}$ in \url{https://www.isa-afp.org/sessions/hol-csp/\#Process\_Order.html}.}
%%\ilacom{Check that this has not be proven before.}

  
\begin{corollary}
  \label{cor:testleqS-equals-failleq}
  Let $\genlts_A, \genlts_B \in \obaFB$.
  For every $\serverA \in \States$  and
  $\serverB \in \StatesB $, we have that
  $\serverA \testleqS \serverB$ if and only if
  $\liftFW{\serverA} \bhvleqone \liftFW{\serverB}$
  and $\liftFW{\serverA} \failleq \liftFW{\serverB}$.
\end{corollary}
%% \begin{proof}
%%   Suppose that $\serverB$ refuses $(s,X)$, then
%%   $\lnot (\serverB \mathrel{\opMust} (s,X))$,
%%   and thus $\lnot (\serverA \mathrel{\opMust} (s,X))$,
%%   which menas that $\serverA$ refuses $(s,W)$.
%% \end{proof}


%%% FOR LONG VERSION
%% {\bfseries Further applications.}
%% Thanks to our alternative preorders, 
%% we also formalise an intuition
%% that comes from our every-day experience: in an asynchronous world, a
%% server that performs only inputs is useless, \ie it can be replaced by a
%% completely inactive program, like~$\Nil$ in $\ACCS$.

%% \begin{proposition}
%%   \label{prop:only-inputs-imply-uselessness}
%%   For every $\genlts_\StatesA \in \obaFB$ and $\server \in \StatesA$,
%%   if for every $\trace \in \Actfin . \, \server \wt{\trace} \st{
%%     \alpha}$ implies that $\alpha \in \Names$ then $\server \testleqS \Nil$.
%% \end{proposition}





\section{\Barinduction: from \extensional to \intentional definitions}
\label{sec:barinduction-main-body}
Two predicates are crucial to reason on the \mustpreorder,
namely passing a test, \ie $\opMust$, and convergence, \ie $\conv$.
Both predicates are defined in an {\em \extensional} manner,
\ie by requiring that for every infinite sequence
there exists a state that is in some sense good. These are respectively the
predicate $\goodSym$ in the definition of $\opMust$ and the predicate
of stability, \ie $\Nst{}$, in the definition of convergence.

Both \extensional predicates can actually be defined inductively,
following an {\em \intentional} approach.
Let $\mathsf{int}_Q$ be the inductive predicate (least
fixpoint) defined by the following rules:
  $$
  \begin{array}{l@{\hskip 2pt}l@{\hskip 20pt}l@{\hskip 2pt}l}
    \rname{axiom}
&    \begin{prooftree}
      Q(s)
      \justifies
      \mathsf{int}_Q(s)
    \end{prooftree}
    &
    \rname{ind-rule}
    &
    \begin{prooftree}
      s \to
      \qquad
      \forall s' \wehavethat  s \to s' \implies \mathsf{int}_Q(s')
      \justifies
      \mathsf{int}_Q(s)
    \end{prooftree}
  \end{array}
  $$
and we define our inductive predicates via $\mathsf{int}$ by letting
$\state \convi \eqdef  \mathsf{int}_{Q_1}( \state )$
and $\musti{ \server }{\client } \eqdef \mathsf{int}_{Q_2}(\state, \client )$,
where $Q_1(\state) \eqdef \state \Nst{\phantom{\tau}}$ and  %\qquad\qquad
$Q_2(\state, \client) \eqdef \good{\client}$.


While proving that the \intentional predicates ($\opMusti$ and~$\convi$)
imply the \extensional ones ($\opMust$ and~$\conv$) are easy arguments by
induction, proving the converse implications is a known problem.
Its constructive solution rests on either the fan-theorem or the
\barinduction principle. The first applies to finite branching trees,
while the second to countably infinite branching trees. We favour
\barinduction because in calculi like infinitary CCS computations
can form countably branching trees.

\begin{proposition}%% \coqBar{extensional_implies_inductive} %%
  Given a countably branching STS~$\sts{\SysStates}{\to}$, and a decidable predicate~$Q$
  on~$\SysStates$, for all~$s \in \SysStates$, $\mathsf{ext}_Q(s)$ implies
  $\mathsf{int}_Q(s).$
\end{proposition}

\begin{corollary}
  \label{cor:ext-int-eq-conv}
  \label{cor:ext-int-eq-must}
  For every $\server \in \States$,
%  \begin{enumerate}
  % \item
  (1) $\state \conv $ if and only if $\state \convi$,
%\item
  (2) for every $\client$ we have that $\Must{\server}{\client}$ if
    and only if $\musti{\server}{\client}$.
%  \end{enumerate}
\end{corollary}
\noindent

Thanks to this corollary, in the proofs of the characterisations of~$\testleqS$,
and in our code, we use the predicates~$\opMusti$ and~$\convi$. In other terms,
we reason by induction.

The details about \barinduction, our mechanisation,
and the proofs of the above results are deferred to \rapp{bar-induction}.


%% \begin{lemma}[\coqMT{weak_a_congr_up_to_permutation}]
%%     \label{lem:weaka-congr-switch-s}
%%     \label{lem:wta-preserved-by-permutation-inputs}
%%   For every $p, q$, and
%%   $s, t \in \Names^\star \wehavethat \perm{s}{t}$ and $p \wta{s} q$ imply $p \wta{t} \cdot \equiv q$.
%% \end{lemma}
%% \noindent
%% where $\perm{s}{t}$ means that $s$ is a permutation of $t$. An easy consequence is that
%% %\rlem{weak-a-swap} implies that
%% convergence is preserved by permuting inputs:

%% \begin{lemma}[\coqMT{acnv_up_to_permutation}]%{mylemma}{existsacnvaimpliesforallacnvaperm}
%%   \label{lem:exists-acnva-implies-forall-acnva-perm}
%%   $\Forevery p \in \Proc \and s,t \in \Names^\star \wehavethat \perm{s}{t}$ and
%%   $p \acnvalong{s} \imply p \acnvalong{t}$.
%% \end{lemma}

%% If we consider also infinite multisets $\I$
%% in \rdefptNOPAR{ff-good}{ff4-acc-sets},
%% then the preorder is not complete. We discuss this in
%% \rexa{finite-multisets-necessary}.

%% In the next two examples we prove that
%% both conditions of \rdef{ff-good} are necessary to prove the soudness of $\testleqS$.

%% \begin{example}
%%         \gb{Show that \rpt{ff4-conv} of \rdef{ff-good} is complete but not sound.}\hfill$\qed$
%% \end{example}


%% \begin{example}
%%         \gb{Show that \rpt{ff4-acc-sets} of \rdef{ff-good} is complete but not sound.}\hfill$\qed$
%% \end{example}



%% We have alredy seen in \rlem{standard-property} does not hold for input actions in the
%% asynchronous CCS, because \rptlem{standard-property}{standard-property-1} is false.
%% The situation does not improve much in
%% the LTS $\sta{}$, because even though the analogous of \rptlem{standard-property}{standard-property-1} is true,
%% it is the analogous of \rptlem{standard-property}{standard-property-2} that is false.



%% \begin{restatable}[COQ LINK]{mylemma}{testleqSpreservedbyrightactioninput}
%%   \label{lem:testleqS-preserved-by-right-action-input}
%%   For every $p,q$, $\mu \in \Names$,
%%   if $p \testleqS q$ and $q \wt{\mu} q'$
%%   then there exists $p'$ such that
%%   $p \wta{\mu} p'$ and $p' \testleqS q'$.
%% \end{restatable}
%% \begin{proof}
%% \gb{We decompose the reduction $q \wta{\mu} q'$ as follows:
%%   $$
%%   q \wta{\varepsilon} q_0 \st{\mu} q_1 \wta{\varepsilon} q'
%%   $$
%%   \rcor{testleqS-included-by-right-tau} and the transitivity of the preorder
%%   ensure that $q \testleqS q_0$ and $q_1 \testleqS q'$.
%%   The hypothesis imply  $p \testleqS q_0$, and that it
%%   it suffices to prove $p' \testleqS q_1$, for some $p'$
%%   such that $p \wt{\mu} p'$.

%%   If $\mu$ is an input we use \rdefpt{sta}{sta-asynch-input} to prove $p \wta{\mu} p \Par \mailbox{\co{\mu}}$.
%%   We prove $p \Par \mailbox{\co{\mu}} \testleqS q_1$.
%%   Now let $p \Par \mailbox{\co{\mu}} \musti r$ for some $r$.
%%   If $r \st{\ok}$ then trivially $q_1 \musti r$. If $r$ is unsuccesful,
%%   then we apply \rlem{must-i-output-swap} to show that $p  \musti r \Par \mailbox{\co{\mu}}$,
%%   and thus $ q_0 \musti r \Par \mailbox{\co{\mu}} $. The transition $ q_0 \st{\mu} q_1$ and
%%   \rlem{musti-presereved-by-actions-of-unsuccesful-tests} imply $ q_1 \musti r $.


%%   If $\mu$ is an output, then observe that $q \Nmusti \co{\mu}.\Nil \extc \tau.\Unit$,
%%   thus by hypothesis $p \Nmusti \co{\mu}.\Nil \extc \tau.\Unit$, and we obtain $p \wta{\mu} p'$
%%   via \rlem{action-or-pass}. \rlem{output-shape} implies $ p \equiv p' \Par \mu $
%%   and that $q \equiv q' \Par \mu$. We conclude via \rlem{testleqS-output-annhil}.}
%%   %% Thanks to \rdefpt{sta}{sta-asynch-input} we know that
%%   %% $p \wta{\mu} p \Par \co{\mu}$.
%%   %% We must show that $p \Par \co{\mu} \testleqS q'$.
%% %% We decompose the reduction $q \wta{\mu} q'$ as follows:
%% %% $$
%% %% q \wta{\varepsilon} q_0 \st{\mu} q_1 \wta{\varepsilon} q'
%% %% $$
%% %% \rcor{testleqS-included-by-wta-epsilon} ensures that
%% %% $p \testleqS q_0$.
%% %% \rlem{must-i-output-swap} ensures that
%% %% $p \Par \co{\mu} \testleqS q_1$ (be more explicit here) and
%% %% \rcor{testleqS-included-by-wta-epsilon}
%% %% allows us to conclude that $p \Par \co{\mu} \testleqS q'$.
%% \end{proof}
%% \gb{Is the hypothesis of $\mu$ being an input necessary ?}


%% \begin{example}
%%   \label{ex:partB-of-standard-lemma-fails}
%% Take $\mu = a$, $p' = \tau.\Nil \extc c.\co{b} $ and $p = p' \Par \mailbox{\co{a} \Par \co{c}}$.
%% By reflexivity $p \testleqS p$, and by definition there exist the transitions
%% $$
%% \begin{array}{l}
%% p \st{\co{a}} \equiv p' \Par \mailbox{\co{c}}
%% \\
%% p \wta{ \co{a} } \st{\tau} \mailbox{\co{b}} \Par \Nil
%% \end{array}
%% $$
%% thus $ \mailbox{\co{b}} \in ( p \aftera \co{a} )$.
%% We have that $\co{b} \NtestleqS p' \Par \co{c}$
%% because $\co{b} \musti b.\Unit$, while the maximal computation
%% $$
%% \svr[ p' \Par \mailbox{\co{c}} ] \Par \clt[  b.\Unit ] \st{\tau}
%% \svr[ \Nil \Par \mailbox{\co{c}} ] \Par \clt[  b.\Unit ] \stable
%% $$
%% witnesses that
%% $p' \Par \mailbox{\co{c}} \Nmusti b.\Unit$. It follows that $  ( p \aftera \co{a} ) \NtestleqS q'$.\qed
%% %  It is not true that for every $p$, $q$, and $q'$, $\mu \in \Names$,
%% %  if $q \st{\co{\mu}} q'$ then for every $p'$ such that
%% %  $p \wta{\co{\mu}} p'$, then $p' \testleqS q'$.
%%   %% Take $\mu = a$,
%%   %% $p = q = (\tau.\Nil \extc \tau.\co{b}) \Par \co{a}$.
%%   %% It is clear that $p \testleqS q$, $q \st{\co{a}} (\tau.\Nil \extc \tau.\co{b}) \Par \Nil$
%%   %% and $p \wta{\co{\mu}} \co{b} \Par \Nil$, but $\co{b} \NtestleqS \tau.\Nil \extc \tau.\co{b}$
%%   %% as $\co{b} \NtestleqS \tau.\Nil \extc \tau.\co{b}$.
%%   %% As a consequence we need a more precise Lemma to catch the desired $p'$,
%%   %% as we try in \rlem{testleqS-output-inversion}.
%% \end{example}
