\section{Further related works}
\label{sec:bisimulation}
\label{sec:detailed-related-works}
%%\subsection{Related and future work}
%%\label{sec:stateoftheart}
%%\label{sec:related-works}
%%\label{sec:future-works}



{\bfseries Contextual preorders in functional languages.}
Morris preorder is actively studied in
the pure $\lambda$-calculus \cite{DBLP:conf/rta/BreuvartMPR16,DBLP:journals/lmcs/BreuvartMR18,DBLP:journals/lmcs/IntrigilaMP19,DBLP:books/cp/BarendregtM22}, %
$\lambda$-calculus with references \cite{DBLP:conf/icalp/Prebet22,DBLP:conf/fossacs/HirschkoffJP23}, %
in PCF \cite{koutavas13} %
as well as in languages supporting shared memory concurrency \cite{DBLP:conf/popl/TuronTABD13}, and mutable references \cite{DBLP:journals/jfp/DreyerNB12}.
%
The more sophisticated the languages, the more intricate and larger the proofs.
The need for mechanisation became thus apparent, in particular to prove that
complex logical relations defined in the framework Iris
  (implemented in Coq) % a metalogic defined in Coq,
are sound, \ie included in the preorder \cite{DBLP:conf/popl/KrebbersTB17,DBLP:conf/lics/FruminKB18}.
%% TODO study the paper and describe better what they do.
\cite{DBLP:journals/jlap/AubertV22} provide a framework to study
contextual equivalences in the setting of process calculi.
It is worth noting, though, that as argued by
\cite{DBLP:journals/lisp/Boudol98} (in Section~$3$ of that paper),
Morris equivalence coincides with $\May$-equivalence,
at least if the operational semantics at hand enjoys the Church-Rosser
property. In fact \cite{DBLP:conf/caap/BoudolL96} define Morris
preorder literally as a testing one, via tests for convergence.
The studies of the \mustpreorder in process calculi can thus be seen
as providing proof methods to adapt Morris equivalence to
\nondeterministic settings, and using contexts that are really
external observers. To sum up, one may say that
  Morris equivalence coincides with $\May$-equivalence when
  \nondeterminism\ is confluent and all states are viewed as accepting states,
  while it coincides with $\opMust$ equivalence in the presence of true
  \nondeterminism\ and when only successful states are viewed as accepting states.

  In the setting of nondeterministic and possibly concurrent
  applicative programming languages
  \cite{DBLP:conf/ppdp/Schmidt-Schauss18,DBLP:journals/corr/abs-2008-13359,
  birkedal-non-determinism},
  also a contextual preorder based on may and must-termination has
  been studied \cite{DBLP:journals/mscs/SabelS08,birkedal-non-determinism}.
  Our preorder $\bhvleqone$ is essentially a
  generalisation of the must-termination preorder of
  \cite{DBLP:journals/mscs/SabelS08} to traces of visible actions.
  
  %% The work of \cite{DBLP:conf/fossacs/HirschkoffJP23} provides a similar
  %% example for the Morris preorder itself. \leo{This reference should be better
  %% motivated IMO.}

  
  %%%%%%%%%%%%%%%%%%%%%%%%%%%%%
%%% Models of asynchrony
%%%%%%%%%%%%%%%%%%%%%%%%%%%%%
{\bfseries Models of asynchrony.}
While synchronous (binary) communication requires
the simultaneous occurrence of a send and a receive action,
asynchronous communication allows a delay between a send action and
the corresponding receive action.  Different models of asynchrony
exist, depending on which medium is assumed for storing messages in
transit. In this paper, following the early work on the asynchronous
$\pi$-calculus~\cite{DBLP:conf/ecoop/HondaT91,boudol:inria-00076939,
  ACS96},
we assume the medium to be an unbounded
unordered mailbox, shared by all processes. Thus, no process needs to
wait to send a message, namely the send action is non-blocking.
  This model of communication is best captured via the output-buffered
  agents with feedback of \cite{DBLP:conf/concur/Selinger97}. 
  % LTSs have been shown to be morphisms of a traced monoidal category
  % of buffers by \cite{DBLP:journals/entcs/Selinger99}.
  The early style LTS of the asynchronous $\pi$-calculus is a concrete example of this
  kind of LTSs.
A similar global unordered mailbox is used %for instance
also in Chapter~$5$ of \cite{DBLP:phd/us/Thati03}, %
by \cite{Brookes2002DeconstructingCA}, which relies explicitly on a mutable
global state, and by \cite{palamidessi_2003}, which
manipulates it via two functions {\em get} and {\em set}.
%\ilacom{In the bib file, the ref to Palamidessi should be updated to}
%\gb{IS the problem fixed?}


% one unordered buffer for every name
  %Instead,
More deterministic models of asynchrony are obtained assigning a data
structure to every channel. %
%% For example \cite{josephs} model
%% asynchrony by assigning a separate {\em unordered} buffer to each
%% channel.  \ilacom{I haven't read \cite{josephs}, but if channels are as
%%   in CCS or TCSP I don't see how assigning an {\em unordered} buffer
%%   to each channel can make the model more deterministic. If messages
%%   are unordered no matter the channel they were sent along, where does
%%   the additional determinism come from?}
%
For example \cite{HYC08,HYC16} use an even more deterministic model
in which each ordered pair of processes is assigned a dedicated
channel, equipped with an {\em ordered} queue. Hence, messages along
such channels are received in the same order in which they were sent.
% \ilacom{Aparté. One should be careful when comparing different models, because
%   the communication mode may vary: in CCS (resp., the $\pi$-calculus)
%   all channels are shared among all processes, and one must use
%   restriction (resp., the $\nu$ construct of the $\pi$-calculus) to
%   ``privatise'' communication on a given channel.  Instead, in session
%   calculi communication is point-to-point and there is a dedicated
%   channel for each ordered pair of processes. Moreover, usually in
%   session calculi each process is assumed to have sequential
%   behaviour (\ie there is only top level parallelism). So indeed, their
%   communication pattern is more restricted. In fact, the LTSs of
%   sessions are fully deterministic, in the sense that there are no two
%   transitions from the same state with equal labels.}  
  This model is used for asynchronous session calculi, and mimics the
  communication mode of the TCP/IP protocol.  The obvious research
  question here is how to adapt our results to the different
  communication mechanisms and different classes of LTSs. For
  instance, both \cite{Tanti2015TowardsSR} and \cite{caruana19}
  define LTSs for \textsc{Erlang}. We will study whether at least one
  of these LTSs is an instance of output-buffered agents with
  feedback. If this is not the case, we will first try to adapt our
  results to \textsc{Erlang} LTSs.


  
  {\bfseries Mutable state.}
  Prebet \cite{DBLP:conf/icalp/Prebet22} has recently shown an 
  encoding of the asynchronous $\pi$-calculus into a $\lambda$-calculus
  with references, which captures Morris equivalence via a bisimulation.
  This renders vividly the intuition that output-buffered agents
  manipulate a shared common state.
  We therefore see our work also as an analysis of the \mustpreorder for
  a language in which programs manipulate a global mutable store.
  Since the store is what contains output messages, and our formal
  development shows that only outputs are observable, our results
  suggest that characterisations of testing preorders for impure
  programming languages should predicate over the content of the mutable
  store, \ie the values written by programs.
  Another account of $\pi$-calculus synchronisation via a
    functional programming language is provided in
    \cite{DBLP:journals/corr/abs-2008-13359}, that explains how to
    use Haskell \textsc{M-var}s to implement $\pi$-calculus
    message passing.


%% \subsection{Comparision with bisimulation}


%% \gb{First, the must preorder does not require an LTS to be defined. A reduction semantics suffices.}


%% \newcommand{\failleq}{\leq_{\textsf{fail}}}
%% \newcommand{\faileq}{=_{\textsf{fail}}}


%% %% {\bfseries Software correctness via refinements:}
%% %% %% WHY REFINEMENTS ?
%% %% The highest precision when reasoning on software correctness can be otained %
%% %% %To reason appropriately on software correctness, it is paramount to achieve the highest precision, which can be attained
%% %% via {\em formal methods}, i.e.~techniques that employ mathematical means to {\em prove} that the software at hand behaves as expected.
%% %% According to \cite{DBLP:conf/amast/ButhKPS97}, the formal
%% %% notion of {\em refinement} fits this purpose, and captures
%% %% % is particularly useful in 
%% %% many forms of engineering activity.  A refinement is a
%% %% relation $\leq$ over programs, such that if $p \leq q$ then the
%% %%   program~$q$ can be used in place of~$p$ safely, i.e. without
%% %%   breaking any existing functionality.
%% %% %that \ila{specifies when a program may  safely 
%% %% %% %  replace another one.}
%% %% %% \ila{Accordingly, this relation
%% %% %%   is required to be} reflexive, i.e. every program
%% %% %% % can
%% %% %% \ila{may} replace itself ($p \leq p$), and transitive, i.e.
%% %% %% if~$q$ %can
%% %% %% \ila{may} safely replace~$p$ ($p \leq q$) and~$s$ %can
%% %% %% \ila{may} safely replace~$q$ ($q \leq s$) then~$s$ %can
%% %% %% \ila{may} replace~$p$ ($p \leq s$).
%% %% In practice refinements are useful because they reduce the task of 
%% %% proving the correctness of a program~$q$ to two smaller ones, namely
%% %% \begin{inparaenum}[(1)]
%% %% \item finding a correct~$p$,
%% %% \item and proving that~$p \leq q$.
%% %% \end{inparaenum}
%% %% In practice, often either~$p$ already exists, or it is simpler than~$q$.
%% %% %
%% %% %% EXAMPLES OF USE OF REFINEMENTS
%% %% %Refinements play a crucial role in the development of software,
%% %% %its verification, and in its management.
%% %% Refinements let us write correct software %via
%% %% by \emph{iterative development} \cite[Section 1.1.6]{sommerville2011software},
%% %% that is, in successive cycles. To ensure that the program $p_{i+1}$
%% %% written at cycle $i+1$ improves on the program $p_i$, it suffices to
%% %% prove that $p_i \leq p_{i+1}$.
%% %% %A representative use case was given in
%% %% %% NO: the code there was given. The verification took place via refinements.

%% %% Refinements have been applied in software verification. For
%% %%   instance the authors of
%% %%   \cite{DBLP:conf/birthday/PeleskaB99,DBLP:conf/amast/ButhKPS97,DBLP:conf/amast/ButhPS98},
%% %%   used
%% %% % give a use case:
%% %% %\ila{where}
%% %% the so-called {\em failure refinement}, here denoted~$p \failleq q$,
%% %% to verify the deadlock freedom of the
%% %% software in the Russian module of the International Space Station.
%% %% % , they relied on the so-called {\em failure refinement}.
%% %% In fact, failure refinement has been successfully
%% %% employed in the design of security protocols \cite{ryan2001modelling,DBLP:conf/fmoods/Schneider02},
%% %% % of 
%% %% railway systems \cite{DBLP:conf/fm/ButhS99},
%% %% % of 
%% %% electronic payment systems \cite{DBLP:conf/tacas/IsobeR05},
%% %% and 
%% %% % in 
%% %% model-based engineering of
%% %% robotics~\cite{DBLP:journals/sosym/MiyazawaRLCTW19}.
%% %% It has also been used
%% %% % as well as 
%% %% for simulation~\cite{DBLP:journals/scp/CavalcantiSMRFD19}.



%% \gb{This, once LTS are introduced, explain differences with bisimulation.}


%% \begin{figure}[t]
%%   \hrulefill
%%   \begin{center}
%%   \begin{tabular}{c@{\hskip20pt}c}%@{\hskip20pt}c@{\hskip20pt}c}
%% %    \scalebox{.9}{
%%       \begin{tikzpicture}
%%         \node[state,scale=0.8] (s1) at (6,0) {$\server_1$};
%%         \node[state,scale=0.8,below of=s1,left of=s1] (s2) {$$}; 
%%         \node[state,scale=0.8,below of =s1,right  of=s1] (s3) {$$}; 
%%         \node[state,scale=0.8,below of=s2] (s4) {$$}; 
%%         \node[state,scale=0.8,below of=s3] (s5) {$$}; 
        
%%         \node[scale=0.8, below of = s5] (dummy) {$$};
        
%%         \path[->]
%%         (s1) edge node [above left,scale=0.8] {$\texttt{str}$} (s2)
%%         (s1) edge node [above right,scale=0.8] {$\texttt{str}$} (s3)
%%         (s2) edge node [left,scale=0.8] {$\texttt{int}$} (s4)
%%         (s3) edge node [right,scale=0.8] {$\texttt{long}$} (s5);
%%       \end{tikzpicture}
%% %    }
%%       &
%%      \begin{tikzpicture}
%%         \node[state,scale=0.8] (s0) at (0,0) {$\server_2$};
%%         \node[state,scale=0.8,below of=s0] (s1) {$$}; 
%%         \node[state,scale=0.8,below of =s2,left  of=s1] (s2) {$$}; 
%%         \node[state,scale=0.8,below of =s1,right  of=s1] (s3) {$$}; 
%%         \node[state,scale=0.8,below of=s2] (s4) {$$}; 
%%         \node[state,scale=0.8,below of=s3] (s5) {$$}; 
        
        
%%         \path[->]
%%         (s0) edge node [left,scale=0.8] {$\texttt{str}$} (s1)
%%         (s1) edge node [above left,scale=0.8] {$\tau$} (s2)
%%         (s1) edge node [above right,scale=0.8] {$\tau$} (s3)
%%         (s2) edge node [left,scale=0.8] {$\texttt{int}$} (s4)
%%         (s3) edge node [right,scale=0.8] {$\texttt{long}$} (s5);
%%       \end{tikzpicture}
%%      \\
%%      (a) & (b)
%%      \\[10pt]
%%       \begin{tikzpicture}
%%         \node[state,scale=0.8] (s0) at (0,0) {$q_1$};
%%         \node[state,scale=0.8,below of =s0] (s1) {$$}; 
        
%%         \path[->]
%%         (s0) edge node [left,scale=0.8] {$\texttt{str}$} (s1)
%%         % terrible terrible hack.
%%         (s1) edge  [white, out=225, in=-45, loop] node[below, scale=0.8] {$\tau$} (s1);
%%       \end{tikzpicture}
%%       &
%%       \begin{tikzpicture}
%%         \node[state,scale=0.8] (s0) at (0,0) {$q_2$};
%%         \node[state,scale=0.8,below of=s0] (s1) {$\Omega$};
        
%%         \path[->]
%%         (s0) edge node [left,scale=0.8] {$\texttt{str}$} (s1)
%%         (s1) edge  [out=225, in=-45, loop] node[below, scale=0.8] {$\tau$} (s1);
%%       \end{tikzpicture}
%%       \\
%%       (c) & (d)
%%   \end{tabular}
%%   \end{center}
%%   % \caption{Servers to prove that weak \ila{bisimilarity} and the
%%   %   equivalence induced by the \mustpreorder are incomparable.}
%%   \caption{Servers showing that
%%       $\,\testeqS~~\not\subseteq~~\approx\,$ and
%%       $\,\approx~~\not\subseteq~~\testeqS$.
%%      }
%%   \label{fig:distinguishing-processes}
%%   \hrulefill
%% \end{figure}


%% %% Examples and explanations
%% \gb{{\bfseries Equating non-determinisms, distinguishing termination and divergence}
%% %To \gb{recall} %\ila{start} discussing
%% %the features of the \mustpreorder,
%% Consider the behaviours in \rfig{distinguishing-processes}, we have that
%% \begin{equation}
%%   \label{eq:features-must}
%%   \server_1 \testeqS \server_2, \qquad q_1 \NtestleqS q_2
%% \end{equation}
%% where~$\testeqS$~denotes the obvious equivalence induced by~$\testleqS$.}
%% %Consider the behaviours~$\server_1$ and~$\server_2$ in
%% %\rfig{distinguishing-processes}.
%% The server~$\server_1$ is ready to
%% input a string, and in doing so
%% % by performing %said input
%% % \ila{such input}
%% it nondeterministically
%% chooses whether to input next an integer or a long.
%% %Similarly to
%% Like  the server~$\server$ of \rfig{first-example},~$\server_1$ is an
%% external choice, but in this case
%% % , though,
%% the environment cannot
%% influence the result of the choice %by~$\server_1$,
%% as both branches of~$\server_1$ match the same offer. % from the environment.
%% % , which therefore is non-deterministic.
%% % This form of non-determisim is often called {\em
%% %   external} for it depends on an interaction between the server and
%% % the environment.  
%% The behaviour of~$\server_2$ is %only
%% slightly different: it first inputs a string, and then takes an internal decision,
%% represented by the two $\tau$ transitions, akin to an \texttt{if then
%%   else}. In one branch it becomes ready to input an integer, while in
%% the other it becomes ready to input a long. In this case the server
%% shows some {\em internal} nondeterminism.
%% \cite{DBLP:journals/acta/Nicola87} argues that there is no way to
%% distinguish~$\server_1$ and~$\server_2$, \ie external and internal
%% nondeterminism, if one can only observe the visible actions they
%% may perform and the way they react to external stimuli.
%% % only the visible actions they may perform and the
%% % way they may react to external stimuli are considered. 
%% This is exactly
%% the standpoint of clients, which indeed cannot distinguish the servers
%% at hand, hence $\server_1 \testeqS \server_2$.

%% %As another example,
%% Consider now the servers~$q_1$ and~$q_2$.
%% Since parallel composition models an unfair scheduler they
%% are distinguished by the client $r = \co{\texttt{str}} \Par
%% \tau.\Unit$, depicted in \rfig{first-example}, where $\Unit$
%% denotes syntatically a good state, depicted using dashed borders.



%% This client will always reach a good
%% (\ie dashed) state when run together with~$q_1$, while it could be
%% prevented from reaching success when run together with~$q_2$, because
%% the scheduler allows the infinite computation:
%% $$
%% \cpar{q_2}{r} \st{\tau} \cpar{ \Omega }{r} \st{\tau}  \cpar{
%%   \Omega }{r}  \st{\tau} \ldots
%% $$
%% in which~$r$ starves. This implies~$q_1 \NtestleqS q_2$.



%% {\bfseries Advantages over weak bisimilarity.}
%% %Summing up, we have $\server_1 \testeqS \server_2$ and $q_1 \NtestleqS q_2$.
%% %This
%% \gb{\req{features-must} puts the \mustpreorder in sharp contrast with the
%%   weak bisimilarity equivalence. First of all, we have that~$\server_1 \not\approx
%% \server_2$. To prove this, thanks to the Hennessy-Milner theorem it suffices to show a formula 
%% %not
%% %\ila{weakly bisimilar if and only if they satisfy the same formulae
%% %are distinguished by some formula
%% in the recursive %extension of
%% Hennessy-Milner logic\footnote{Also known as the modal $\mu$-calculus.}
%% }%% GB
%% that is satisfied by $\server_2$ and not by $\server_1$.
%% %It is thus easy to see that~$\server_1 \not\approx \server_2$,
%% The discriminating formula is $\dmd{\texttt{str}}(\dmd{\texttt{int}}\true 
%% \wedge \dmd{\texttt{float}}\true)$.
%% %, which is satisfied by~\ila{$\server_2$} but not by~\ila{$\server_1$.} 
%% %\gb{I suggest to discuss elsewhere the test associated to
%% %  $\phi$, in order to stick to the point here.}
%% Essentially, weak bisimilarity distinguishes the external nondeterminism of~$\server_1$
%% and the internal nondeterminism of~$\server_2$, while the
%% \mustpreorder does not.
%% Indeed, the client to test the above formula is
%% $\Unit \Par \co{\texttt{str}} \Par \co{\texttt{int}} \Par
%% \co{\texttt{float}}$ and it is satisfied by both $q_1$ and $q_2$,
%% \ie does not distinguish them.
%% Aware of this phenomenon, %first
%% \cite{DBLP:journals/acta/Nicola87}, and later %then
%% \cite{DBLP:journals/tcs/BorealeG06} and \cite{DBLP:journals/iandc/RensinkV07},
%% argued that weak bisimilarity is over-discriminating, \ie that its
%% resolution power can be undesirable in practice.
%% %% NOT REALLY ARGUED BY THOSE RensinkV07
%% %and that
%% %equivalences à la~$\testeqS$ offer a more faithful picture of
%% %reality.

%% It is also immediate to see that~$q_1 \approx q_2$, since
%% weak bisimilarity does not distinguish termination and
%% divergence\footnote{Divergence-sensitive variants of weak bisimilarity
%%   have been considered, but they lose some of the
%%   advantages of the original definition.}.
%% Again, this seems a worse model of reality than the one afforded by
%% the \mustpreorder.

%% \newcommand{\spec}{{\mathit{spec}}}
%% \newcommand{\impl}{{\mathit{impl}}}

%% \cite{DBLP:journals/iandc/RensinkV07} suggest also that
%% preorders have an intrinsic advantage over equivalences,
%% in that they allow comparing specifications and implementations.
%% \gb{For instance, the inequality~$\server_\spec~\testleqS~\server_\impl$
%% %we can  consider
%% %view~$\server_2$ as an
%% %{\em implementation} of the {\em specification}~$\server_1$.
%% %the definition
%%   %of the preorder
%%   ensures that any client satisfied by the specification
%% will run correctly also with the implementation, hence a proof that
%% $\server_\spec \testleqS \server_\impl$
%% %coincides with proving the
%% ensures that the implementation is correct. This is indeed a standard
%% usage of \textsc{CSP} failure refinement
%% \cite{DBLP:conf/icalp/Brookes83,DBLP:conf/tacas/Gibson-RobinsonABR14}.}
%% In the world of refinements, the \mustpreorder has %again
%% an advantage also over \emph{weak similarity}, denoted~$\lesssim$,
%% for a theory for~$\testleqS $ sheds light %by definition
%% also on the  equivalence~$\testeqS$, since $
%% {\testeqS} = {\testleqS} \cap
%% (\testleqS)^{-1}$. This is not the case for weak similarity, since
%% ${\approx} \neq {\lesssim} \cap (\lesssim)^{-1}$ (see
%% \cite{DBLP:books/daglib/0098267}, Exercise 3.10).
%% % is the standard one to
%% % prove that ${\approx} \neq {\lesssim} \cap (\lesssim)^{-1}$.


%% A more subtle advantage over weak bisimilarity is that~$\testleqS$ and
%% the equivalence it induces can be parameterised with respect to
%% different notions of fairness, as explained by \cite{DBLP:conf/fossacs/Glabbeek23}:
%% it suffices to change the semantics of parallel composition.
%% A solid understanding of the \mustpreorder can thus shed light also
%% on fair refinements. This is exemplified by
%% \cite{DBLP:journals/jfp/AghaMST97}, who point out explicitly
%% the impact of the fairness assumptions in their proofs.

%% \gb{%
%% A final asset of the \mustpreorder in particular, and testing ones
%%   in general, is that it is equipped with fully abstract denotational models
%% (\cite{DBLP:books/daglib/0066919,DBLP:journals/iandc/BorealeN95,DBLP:journals/tcs/Hennessy02,DBLP:journals/corr/Hirschowitz14}).}
%% %while bisimilarity lacks a natural denotational model.




%%%%%%%%%%%%%%%%%%%%%%%%%%%%%%%%%%%%%%%%%%%%%%%%%%%%%%%%%%%%%%%%%%%%%%%%%%%%%%%%%%%%%%%
%%%%%%%%%%%%%%%%%%%%%%%%%%%%%%%%%%%%%%%%%%%%%%%%%%%%%%%%%%%%%%%%%%%%%%%%%%%%%%%%%%%%%%%
%%%%%%%%%%%%%%%%%%%%%%%%%%%%%%%%%%%%%%%%%%%%%%%%%%%%%%%%%%%% SHORT COMPARISON

  %%
  %% BETTER OMIT THIS.
  %% THE PAPER IS ALREADY TOO LONG, AND ANY SERIOUS COMPARISONG WITH BISIMULATION OPENS A OF WORMS.
  %%
%% {\bfseries \mustpreorder and bisimilarity}
%% The differences between the \mustpreorder %must testing preorder
%% and bisimilarity in a synchronous setting are discussed at length by
%%   \cite{DBLP:conf/icalp/Pnueli85,DBLP:journals/acta/Nicola87,DBLP:journals/tcs/BorealeG06,DBLP:journals/iandc/RensinkV07,sangiorgi}.
%% Essentially~$\testleqS$~distinguishes termination and divergence,
%% while bisimilarity does not; and~$\testleqS$~does not distinguish external and
%% internal non-determinism, while bisimilarity does.
%% Similar differences may be found between their asynchronous
%% counterparts.

%% \gb{Why do we care about about common features ?}
%% Let us instead point out some common
%%   features of our asynchronous \mustpreorder and of the
%%   asynchronous bisimilarity of~\cite{ACS96,ACS98}:
%%   \begin{inparaenum}
%%   \item the definition of asynchronous bisimilarity essentially mimics the ordinary
%%     bisimilarity in the LTS by Honda and Tokoro, the same LTS which is
%%     the basis for our alternative characterisations of the asynchronous
%%     \mustpreorder, and
%%   \item
%%     the way in which our characterisations relate stable states
%%     %our asynchronous acceptance sets
%%     differ from the classical one \gb{No it doesn't. It is exactly the
%%     same definition.} in two regards, by %ones by
%%     the use of traces in the LTS by Honda and Tokoro (rather than
%%     traces in the ordinary LTS), and by considering only output
%%       actions instead of all the actions, %restricting the final ready
%%     %set to outputs,
%%     much in the same way as the barbed congruence characterisation of
%%     asynchronous bisimilarity given by \cite{ACS96,ACS98} differs
%%     from the standard one by restricting the set of barbs to the output barbs.
%%   \end{inparaenum}





%%%%%%%%%%%%%%%%%%%%%%
%% SYNCHRONOUS SETTING
%%%%%%%%%%%%%%%%%%%%%%
{\bfseries Theories for synchronous semantics.}
%%%%% HIDDEN FOR BREVITY
%%%%% DO NOT EREASE
%% The \mustpreorder has been studied and/or employed in more than~$30$
%% papers, as well as in a book by~\cite{DBLP:books/daglib/0066919},
%% always in settings where both inputs and outputs are blocking.
%% Its theory has thus informed various other formal models that
%% we breafly recall.
%%%%
%%%%
%% Web services
Both \cite{DBLP:conf/concur/LaneveP07} and
\cite{DBLP:journals/toplas/CastagnaGP09} employed LTSs as a model of
contracts for web-services (\ie \texttt{WSCL}), and the \mustpreorder
as refinement for contracts.  The idea is that a search engine asked
to look for a service described by a contract~$\server_1$ can
actually return a service that has a contract~$\server_2$, provided
that~$\server_1 \testleqS \server_2$.

%% Monitors
The \mustpreorder for {\em clients} proposed by
\cite{DBLP:journals/corr/BernardiH15} has partly informed the
theory of monitors by \cite{DBLP:journals/pacmpl/AcetoAFIL19},
in particular the study of preorders for monitors by
\cite{DBLP:journals/iandc/Francalanza21}.
Our results concern LTSs that are more general than those of monitors,
and thus %we expect that
our code could provide the basis to mechanise
the results of \cite{DBLP:journals/pacmpl/AcetoAFIL19}.

%% Session types
The first subtyping relation for binary session types was presented
in~\cite{DBLP:journals/acta/GayH05} using a syntax-oriented definition.
The semantic model of that subtyping is a refinement %that is
very similar to the \mustpreorder. The idea is to treat types as \textsc{CCS} terms,
assign them an LTS
\cite{DBLP:conf/ppdp/CastagnaDGP09,DBLP:conf/ppdp/Barbanerad10,DBLP:journals/iandc/RavaraRV12,DBLP:journals/mscs/BernardiH16},
and use the resulting testing preorders as semantic models of the
subtyping.
%this operational semantics to define a semantic subtyping.
In the setting of coinductively defined higher-order session types, the
correspondence is implicitly addressed
in~\cite{DBLP:conf/ppdp/CastagnaDGP09}.
In the setting of recursive higher-order session types, %this correspondence is
it is given by Theorem 4.10 of~\cite{DBLP:journals/corr/BernardiH13}.

%{\bfseries Semantic models of subtyping for session types.}
%%% SLIGHT REPETITION WITH PREVIOUS PARAGRAPH
%Testing preorders provide semantic models of subtyping for binary
%session types, both in synchronous and asynchronous settings
%\cite{DBLP:journals/corr/BernardiH13,DBLP:journals/mscs/BernardiH16,DBLP:journals/lmcs/BravettiCLYZ21}.
We would like to mechanise in our framework these results,
in particular the ones about asynchronous semantics, and contrast and
compare the various testing preorders used in the literature.
More in general, given the practical relevance of asynchronous
communication, it seems crucial %
%natural \ilacom{utiliser un adjectif plus percutant que ``natural''? Crucial?}
not only to adapt the large body of theory
outlined above to the asynchronous setting but also to resort to
machine supported reasoning to do it. This paper is meant to be a step
forward in this direction.


{\bfseries Must-preorder and asynchrony.}
The first investigation on the \mustpreorder in an asynchronous
setting was put forth by \cite{DBLP:conf/fsttcs/CastellaniH98}.
While their very clear examples shed light on the preorder, their
alternative preorder (Definition 6 in that paper) is more complicated
than necessary: it uses the standard LTS of \ACCS, the LTS of
forwarders, a somewhat ad-hoc predicate $\rr{I}$, and a condition on
multisets of inputs, that we do not use. Moreover that preorder is not
complete because of a glitch in the treatment of divergence.
The details of the counter-example we found to that completeness
result are in \rapp{counterexample}.

In~\cite{DBLP:journals/jlp/Hennessy05} Hennessy outlines how to adapt the
approach of \cite{DBLP:conf/fsttcs/CastellaniH98} to a typed
asynchronous $\pi$-calculus. While the LTS of forwarders is
replaced by a Context LTS, the predicates to define the alternative
preorder are essentially the same used in the preceding work with
Castellani. Acceptance sets are given in Definition 3.19 there, and
the predicate~$\rr{}$ is denoted~$\searrow$, while the generalised
acceptance sets of \cite{DBLP:conf/fsttcs/CastellaniH98} are given in
Definition 3.20.  Owing to the glitch in the completeness of
\cite{DBLP:conf/fsttcs/CastellaniH98}, it is not clear that Theorem
3.28 of~\cite{DBLP:journals/jlp/Hennessy05} is correct either.

%% TAKEN FROM INTRO
%%   and the subsequent paper
%% while adapting the results of \cite{DBLP:conf/fsttcs/CastellaniH98} to the
%% asynchronous $\pi$-calculus, leaves to the reader
%% some parts of the completeness proof (lemmas 3.26 and 3.27 there).

Also the authors of~\cite{DBLP:journals/iandc/BorealeNP02} the \mustpreorder in \ACCS.
There is a major difference between their approach and ours.
When studying theories for asynchronous programs, one can either
\begin{enumerate}[(1)]
  \item keep the definitions used for synchronous programs,
    and enhance the LTS with forwarders;
    or
  \item adapt the definitions, and keep the standard LTS.
\end{enumerate}
In the first case, the complexity is moved into the LTS, which
becomes infinite-branching and infinite-state.
In the second case, the complexity is moved into the definitions
used to reason on the LTS (i.e. in the meta-language), and in
particular in the definition of the alternative preorder, which
deviates from the standard one. The authors of
\cite{DBLP:journals/iandc/BorealeNP02} follow the second approach. %
This essentially explains why they employ the standard LTS of \CCS
and to tackle asynchrony they reason on traces via
\begin{enumerate}[(i)]
\item a preorder~$\preceq$ (Table 2 of that paper) that defines on {\em input} actions
  the phenomena due to asynchrony, for instance their {\em
    annihilation} rule (i.e. TO3) is analogous to the \outputfeedback axiom, and
  their {\em postponement} (i.e. TO2) is analogous to the
  \outputcommutativity axiom; and
%  In fact \rlem{acnv-drop-in-the-middle} shows that the relation~$\acnvalong$ is preserved by
%  two operations on traces that are reminiscent of {\em deletion} and
%  {\em annihilation} of \cite{DBLP:journals/iandc/BorealeNP02}; and
\item a rather technical operation on traces, namely %$s,s'$, namely
  $s \ominus s' = ( \mset{s}_{i} \setminus
  \mset{s'}_{i}) \setminus \co{( \mset{s}_{o} \setminus  \mset{s'}_{o}
    )}$.
\end{enumerate}
We favour instead the first approach, for, as we already argued,
it helps us achieve a modular mechanisation.
%% Using definitions of alternative preorders that are parametrised over the LTS,
%% we can treat uniformly the characterisations of the \mustpreorder for
%% both the asynchronous semantics and the synchronous one.
%% In fact the arguments developed for the asynchronous case, which
%% features both blocking and non-blocking actions, carry over for free
%% to the synchronous case, which features only blocking actions.

%study both the $\May$ and $\opMust$ preorders for
%\ACCS. The definitions of our alternative preorders are simpler than
%their Definition 7.4.

%% WE ALSO USE MUST-SETS NOW
%% Another difference between our approach and the one by \cite{DBLP:journals/iandc/BorealeNP02}
%% is that our alternative preorder uses acceptance sets, \ie compares (output) actions of stable states.
%% Instead, their Definition $7.4$ compares the so-called \MustSets.
%% More precisely, if~$L$ is a set of actions then they write $\server \mathrel{\opMust} L$
%% whenever $\server \wt{} \server'$ implies that $\server \st{ \mu }$ for some $\mu
%% \in L$, and say that~$L$ is a \MustSet of $\server$.
%% In their alternative preorder then  $\serverA \leq \serverB$
%% if every \MustSet of $\serverA$ is a \MustSet of $\serverB$.



The authors of~\cite{DBLP:journals/tcs/NicolaP00} give yet another account of the \mustpreorder.
Even though non-blocking outputs can be written in their calculus,
they use a left-merge operator that allows writing {\em blocking} outputs.
%Inequalities such as $ {\tt str}.\Nil \testleqS \Nil$
%are thus false in their setting, because clients can be written that
%distinguish the two servers.
The contexts that they use to prove the completeness of their
alternative preorder use such blocking outputs, consequently
their arguments need not tackle the asymmetric treatment of input and
output actions. This explains why they can use smoothly a standard
LTS, while \cite{DBLP:conf/fsttcs/CastellaniH98} and
\cite{DBLP:journals/iandc/BorealeNP02} have to resort to more
complicated structures.


Theorem 5.3 of the PhD thesis by \cite{DBLP:phd/us/Thati03} states an
alternative characterisation of the \mustpreorder, but it is given
with no proof.  The alternative preorder given in Definition 5.8 of
that thesis turns out to be a mix of the ones by
\cite{DBLP:conf/fsttcs/CastellaniH98} and
\cite{DBLP:journals/iandc/BorealeNP02}.  In particular, the
definition of the alternative preorder relies on the LTS of
forwarders, there denoted~$\st{}_A$ (Point 1. in Definition 5.1
defines exactly the input transitions that forward messages into the
global buffer).  The condition that compares convergence of processes
is the same as in \cite{DBLP:conf/fsttcs/CastellaniH98}, while
server actions are compared using \MustSets, and not \AcceptanceSets.
In fact, Definition 5.7 there is titled ``\AcceptanceSets'' but it
actually defines \MustSets.





{\bfseries May-preorder.} %
  $\May$ testing and the $\May$-preorder, have been widely studied in 
  asynchronous settings. The first characterisation for \ACCS appeared 
  in \cite{DBLP:conf/fsttcs/CastellaniH98} and relies on comparing traces
  and asynchronous traces of servers. Shortly after
  \cite{DBLP:journals/iandc/BorealeNP02} presented a characterisation based
  on operation on traces.
  A third characterisation appeared in \cite{DBLP:conf/birthday/BaldanBGV15},
  where the saturated LTS $\st{}_s$ is essentially out $\sta{}$.
  That characterisation supports our claim that results about synchronous
  semantics are true also for asynchronous ones, modulo forwarding.
  Compositionality of trace inclusion, i.e. the alternative
  characterisation of the $\May$-preorder, 
  has been partly investigated in Coq by \cite{aathalye-meng} in the
  setting of IO-automata.
  The $\May$-preorder has also been studied in the setting of actor languages
  by \cite{caruana19,Tanti2015TowardsSR}.
  %% The conclusion of \cite{Tanti2015TowardsSR}
  %% suggests that characterising a liveness preserving preorder for
  %% asynchronous semantics remains an open problem. This paper closes it.

%%%
{\bfseries Fairness.} 
Van Glabbeek \cite{DBLP:conf/fossacs/Glabbeek23} argues that by amending the
semantics of parallel composition (i.e. the scheduler) different
notions of fairness can be embedded in the \mustpreorder. We would like to investigate
which notion of fairness makes the \mustpreorder coincide with the 
\textsc{Fair}-preorder of \cite{DBLP:journals/iandc/RensinkV07}.



{\bfseries Bar-induction.}
A mainstay in the literature on the \mustpreorder is \koenigslemma, see
for example Theorem 2.3.3 in \cite{TCD-CS-2010-13}, and Theorem 1 in
\cite{DBLP:journals/scp/BernardiF18}. \cite{dummett2000elements}, though,
explains in detail why \koenigslemma is not constructive.
%
Instead, we use in this paper the constructive \barinduction principle, whose
fundamental use is to prove that if every path in a tree~$T$ is finite, then~$T$
is well-founded, as discussed by
\cite{DBLP:conf/aplas/NakataUB11,DBLP:conf/lics/BredeH21} and
\cite{kleene1965foundations}.
%
Unfortunately, while it is a constructive principle, mainstream proof assistants
do not support it, which is why we had to postulate it as a proof principle that
we proved using the Excluded Middle axiom. One consequence of using an axiom is
that they do not have computational content.
%
Developing a type theory with a principle of \barinduction is the subject of
recent and ongoing works~\cite{DBLP:conf/types/Fridlender98,DBLP:journals/jacm/RahliBCC19}.


%%%%%%%%%%%%%%%%%%%%%%%%%%%
%% Asynchronous setting
%%%%%%%%%%%%%%%%%%%%%%%%%%%
%% {\bfseries Semantics of Asynchrony}

%% % This leads to the presence
%% This amounts to view outputs as sent messages and to assume
%% the presence of a shared global unordered mailbox, \ie a {\em buffer},
%% which all processes use to send messages to and read messages from.
%% Indeed the LTS we consider are morphisms of a traced monoidal category
%% of buffers presented by \cite{DBLP:journals/entcs/Selinger99}.
%% For instance, consider the term
%% $$
%%  q \Par \mailbox{ \co{a}
%%   \Par \co{ b }} \qquad \text{ where } \qquad q = a . \Nil \extc \tau . b.(\co{c} \Par e.\Nil)
%% $$
%% Here the mailbox is highlighted, and the process~$q$
%% is ready to either do an input $a$, thereby consuming the message %output
%% $\co{a}$ from the mailbox, or internally decide (for instance
%% due a timeout) to move to a state in which it can consume the message
%% $\co{b}$. In the latter case, %after $b$ has consumed $\co{b}$ %the interaction of $b$ and $\co{b}$
%% the process will then put a message $\co{c}$ in the mailbox and remain locked
%% waiting for a message $\co{e}$.




%% \TODO{Possibly omit the following sentence ?}
%% None of the above works relies on the trace normalisation
%% that we used to define our preorder~$\asynleq$ \gb{(\rapp{normal-form})}.







%% %%%%%%%%%%%%%%%%%%%%%%
%% %% SYNCHRONOUS SETTING
%% %%%%%%%%%%%%%%%%%%%%%%
%% {\bfseries Theories for synchronous semantics.}
%% %%%%% HIDDEN FOR BREVITY
%% %%%%% DO NOT EREASE
%% %% The \mustpreorder has been studied and/or employed in more than~$30$
%% %% papers, as well as in a book by~\cite{DBLP:books/daglib/0066919},
%% %% always in settings where both inputs and outputs are blocking.
%% %% Its theory has thus informed various other formal models that
%% %% we breafly recall.
%% %%%%
%% %%%%
%% %% Web services
%% \cite{DBLP:conf/concur/LaneveP07} and
%% \cite{DBLP:journals/toplas/CastagnaGP09} employed LTSs as a model of
%% contracts for web-services (\ie \texttt{WSCL}), and the \mustpreorder
%% as refinement for contracts.  The idea is that a search engine asked
%% to look for a service described by a contract~$\server_1$ can
%% actually return a service that has a contract~$\server_2$, provided
%% that~$\server_1 \testleqS \server_2$.

%% %% Monitors
%% The \mustpreorder for {\em clients} proposed by
%% \cite{DBLP:journals/corr/BernardiH15} has partly informed the
%% theory of monitors by \cite{DBLP:journals/pacmpl/AcetoAFIL19},
%% in particular the study of preorders for monitors by
%% \cite{DBLP:journals/iandc/Francalanza21}.
%% Our results concern LTSs that are more general than those of monitors,
%% and thus %we expect that
%% our code could provide the basis to mechanise
%% the results of \cite{DBLP:journals/pacmpl/AcetoAFIL19}.

%% %% Session types
%% \cite{DBLP:journals/acta/GayH05} presented the first subtyping
%% relation for binary session types using a syntax-oriented definition.
%% The semantic model of that subtyping is a refinement %that is
%% very similar to the \mustpreorder. The idea is to treat types as \textsc{CCS} terms,
%% assign them an LTS
%% \cite{DBLP:conf/ppdp/CastagnaDGP09,DBLP:conf/ppdp/Barbanerad10,DBLP:journals/iandc/RavaraRV12,DBLP:journals/mscs/BernardiH16},
%% and use the resulting testing preorders as semantic models of the
%% subtyping.
%% %this operational semantics to define a semantic subtyping.
%% In the setting of coinductively defined higher-order session types, the
%% correspondence is implicitly addressed in 
%% \cite{DBLP:conf/ppdp/CastagnaDGP09}.  In the setting of recursive
%% higher-order session types, %this correspondence is
%% it is given by Theorem 4.10 of
%% \cite{DBLP:journals/corr/BernardiH13}.

%% %{\bfseries Semantic models of subtyping for session types.}
%% %%% SLIGHT REPETITION WITH PREVIOUS PARAGRAPH
%% %Testing preorders provide semantic models of subtyping for binary
%% %session types, both in synchronous and asynchronous settings
%% %\cite{DBLP:journals/corr/BernardiH13,DBLP:journals/mscs/BernardiH16,DBLP:journals/lmcs/BravettiCLYZ21}.
%% We would like to mechanise in our framework these results,
%% in particular the ones about asynchronous semantics, and contrast and
%% compare the various testing preorders used in the literature.
%% More in general, given the practical relevance of asynchronous
%% communication, it seems crucial %
%% %natural \ilacom{utiliser un adjectif plus percutant que ``natural''? Crucial?}
%% not only to adapt the large body of theory
%% outlined above to the asynchronous setting but also to resort to
%% machine supported reasoning to do it. This paper is meant to be a step
%% forward in this direction.




%%% COMMENTED TO SAVE SPACE
%%%{\bfseries Comparison with failure refinement}

%% \cite{DBLP:journals/iandc/BorealeN95} provide a fully-abstract
%% denotational model and an axiomatisation of $\testleqS$ in the
%% synchronous $\pi$-calculus. Once again, the present work paves
%% the way to adapt those results to asychrony. Similarly for the
%% fully-abstract model of \cite{DBLP:journals/tcs/Hennessy02}.



%%% COMMENTED TO SAVE SPACE
%%%{\bfseries Comparison with failure refinement}

%% \cite{DBLP:journals/iandc/BorealeN95} provide a fully-abstract
%% denotational model and an axiomatisation of $\testleqS$ in the
%% synchronous $\pi$-calculus. Once again, the present work paves
%% the way to adapt those results to asychrony. Similarly for the
%% fully-abstract model of \cite{DBLP:journals/tcs/Hennessy02}.
