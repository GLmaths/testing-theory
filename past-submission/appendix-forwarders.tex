\section{Forwarders}
\label{sec:appendix-forxarders}

The intuition behind forwarders, quoting \cite{DBLP:conf/ecoop/HondaT91},
is that ``any message can come into the configuration, regardless of the forms of
inner receptors. [\ldots] As the experimenter is not synchronously
interacting with the configuration [\ldots], he may send any message
as he likes.''

In this appendix we give the technical results to ensure that
the function $\liftFW{-}$ builds an LTS that satisfies
the axioms of the class \texttt{LtsEq}.

  \newcommand{\stripSym}{\mathsf{strip}}
  \newcommand{\strip}[1]{\stripSym(#1)}

  \begin{definition}\label{def:strip-def}
    We define the function $\stripSym : \States \longrightarrow \States$ by induction on $
    \outputmultiset{ \server }$ as follows: if
    $\outputmultiset{\server}  = \varnothing $ then
    $\strip{\server} = \server$, while
    if $\exists \co{\aa} \in \outputmultiset{\server}$ and $\server \st{ \co{\aa} } \server'$ then
    $ \strip{ \server } =  \strip{ \server' } $.
    Note that $\strip{ \server }$ is well-defined thanks to the \outputdeterminacy
    and the \outputcommutativity axioms.\hfill$\blacksquare$
  \end{definition}


    %%   and $$.
    %% $$
    %% \strip{ \server } =
    %% \begin{cases}
    %%   \server & \text{ if } \outputmultiset{\server} = \varnothing\\
    %%   \strip{ \server' } & \text{ if } \exists  \mu \in \outputmultiset{\server} \text{ and }\server \st{ \mu } \server'
    %% \end{cases}
    %% $$
    

%% \begin{lemma}
%%   \label{lem:stout-uniqueness}
%%   Let $\genlts_A \in \oba$ and $\serverA, \serverB_1, \serverB_2 \in A$.
%%   If
%%   $\outnorm{\serverA}{\serverB_1}$ and $\outnorm{\serverA}{\serverB_2}$ then
%%   $\serverB_1 \simeq \serverB_2$.
%% \end{lemma}
%% \begin{proof}
%%   By induction on the cardinality of $\outputmultiset{\serverA}$,
%%   together with an application of the \outputdeterminacy axiom.
%% \end{proof}



We now wish to show that $\liftFW{\genlts} \in \obaFW$ for any LTS
$\genlts$ of output-buffered agents with feedback.
Owing to the structure of our typeclasses, we have first to construct an
equivalence~$\doteq$ over $\liftFW{\genlts}$ that is compatible with the
transition relation, \ie satisfies the axiom in \rfig{Axiom-LtsEq}.
We do this in the obvious manner, \ie by combining the equivalence $\simeq$
over the states of~$\genlts$ with an equivalence over mailboxes.

\begin{definition}
  \label{def:fw-eq}
  For any LTS $\genlts$, two states $\serverA \triangleright M$
    and $\serverB \triangleright N$ of $\liftFW{\genlts}$ are
  equivalent, denoted
  $\serverA \triangleright M \doteq \serverB \triangleright N$, if
  $ \strip{ \serverA } \simeq \strip{ \serverB }$ and
  $M \uplus \outputmultiset{\serverA} = N \uplus
  \outputmultiset{\serverB}$.\hfill$\blacksquare$
\end{definition}



\begin{lemma}
  \label{lem:harmony-sta}\coqLTS{harmony_a}
  %%%% THIS IS NOT THE HARMONY LEMMA
  For every $\genlts_\StatesA$ and every
  $\serverA \triangleright M, \serverB \triangleright N \in \StatesA \times MO$,
  and every $\alpha \in L$, if
  $
  \serverA \triangleright M \mathrel{({\doteq} \cdot {\sta{\alpha}})}
  \serverB \triangleright N
  $ then
  $
  \serverA  \triangleright M \mathrel{({\sta{\alpha}} \cdot {\doteq})} \serverB' \triangleright N'.
  $
  %%%% OLD STATEMENT %%%
  %% For every $\genlts = \lts{\States}{L}{\st{}}$, %
  %% every $\serverA, \serverB' \in \States$, %
  %% every $M,N' \in MO$ and %
  %% every $\alpha \in L$, if
  %% $
  %% \serverA \triangleright M \mathrel{({\doteq}  \cdot {\sta{\alpha}})} \serverB'
  %% \triangleright N'
  %% $ then %  \text{ implies }
  %% $
  %% \serverA  \triangleright M \mathrel{({\sta{\alpha}} \cdot {\doteq})} \serverB' \triangleright N'.
  %% $
\end{lemma}



%\ilacom{To be more general, in the above paragraph we could replace ``client'' by
%  ``process'' and ``server'' by ``environment''.}

%% \pl{

%% \begin{figure}
%% \hrulefill
%%   $$
%%   \begin{array}{llll}
%%     \begin{prooftree}
%%       \justifies
%%       \server \stout{\varnothing} \server
%%     \end{prooftree}
%%     &
%%     \begin{prooftree}
%%       \server_1 \st{\co{\aa}} \server_2 \qquad
%%       \server_2 \stout{M} \server_3
%%       \justifies
%%       \server_1 \stout{\mset{\aa} \uplus M} \server_3
%%     \end{prooftree}
%%   \end{array}
%%   $$
%%   \caption{The transition $\serverA \stout{M} \serverB$}
%%   \label{fig:rules-stout}
%%   \hrulefill
%% \end{figure}

%% \begin{definition}
%%   Given an LTS $\genlts_{\StatesA}$% $\lts{\States}{L}{\st{}}$
%%   and two states $\serverA, \serverB \in \StatesA$,
%% we define the transition relation $\server \st{M} \serverB$,
%% where $M$ is a multiset of actions, via the following rules
%% \begin{description}
%% \item[\rname{st-m-refl}]
%%   $\server \st{\varnothing} \server$
%% \item[\rname{st-m-mu}]
%%   $\server \st{\mset{\mu} \uplus M} \serverB$ \quad if $\server \st{\mu} \server'$
%%   and $\server' \st{M} \serverB$
%% \end{description}
%% \end{definition}

%% \TODO{QUESTION: should we skip the definition of strip and inline its definition
%%   where it is used ? It would avoid another indirection.}

%% \begin{definition}
%%   Let $\genlts_A \in \oba$.
%%   Given two servers $\serverA$ and $\serverB$ in $A$,
%%   let $\outnorm{\serverA}{\serverB}$ be the relation defined as
%%   $$
%%   \outnorm{\serverA}{\serverB} = \serverA \st{\outputmultiset{\serverA}} \serverB
%%   $$
%% \end{definition}
