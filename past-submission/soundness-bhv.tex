\section{Soundness}

\renewcommand{\mustset}[2]{#1 \mathrel{\opMustset} #2}

\newcommand{\msetnow}{\rname{Mset-now}}
\newcommand{\msetstep}{\rname{Mset-step}}

\newcommand{\cnvleqset}{\mathrel{\preccurlyeq^{\mathsf{set}}_{\mathsf{cnv}}}}
\newcommand{\accleqset}{\mathrel{\preccurlyeq^{\mathsf{set}}_{\mathsf{acc}}}}
\newcommand{\Naccleqset}{\mathrel{{\not\preccurlyeq}^{\mathsf{set}}_{\mathsf{acc}}}}
\newcommand{\asleqset}{\mathrel{\preccurlyeq^{\mathsf{set}}_{\mathsf{AS}}}}
\newcommand{\Nasleqset}{\mathrel{\not\preccurlyeq^{\mathsf{set}}_{\mathsf{AS}}}}

%\onecolumn

%\begin{scriptsize}
\begin{figure}
 \hrulefill
  $$
  \begin{array}{l}%*{c@{\hskip 2em}l}
    \msetnow
    \\[1pt]
    \begin{prooftree}
      \good{\client}
      \justifies
      \mustset{ X }{r}
    \end{prooftree}
    \\[10pt]
    \msetstep
    \\[1pt]
    \begin{prooftree}
      \begin{array}{lr}
        \lnot \good{\client} & \forall X' \wehavethat X \st{ \tau } X' %\text{ and }  X'
        \implies \mustset{X'}{\client}\\
      \forall\ \serverA \in X \wehavethat \csys{ \serverA }{ \client } \st{\tau} & \forall \client' \wehavethat \client \st{ \tau } \client' \implies \mustset{X}{\client'}
      \\
      \multicolumn{2}{r}{        \forall X', \mu \in \Actfin  %% p \stable
        \wehavethat X \wt{\co{\mu}} X' %\text{ and }  X'
        \text{ and }  \client \st{\mu} \client' \imply %% \der{p}{\co{\aa}}
        \mustset{ X' }{ \client'}}
      \end{array}
      \justifies
      \mustset{ X }{ \client }
    \end{prooftree}
  \end{array}
  $$
  \vspace{-10pt}
  \caption{Rules to define inductively the predicate $\opMustset$.}
%    (\coqSou{must__set}).}
  \label{fig:rules-mustset}
\hrulefill
\end{figure}
%\end{scriptsize}


\label{sec:bhv-soundness}\label{sec:soundness-bhv}

To prove the converse of \rprop{bhv-completeness}, a naïve proof does not work:
suppose given servers $\server$ and $\serverB$, assume $\serverA \asleq \serverB$.
We need to prove that, for every client $\client$, if
$\musti{\server}{\client}$ then $\musti{\serverA}{\client}$.
The reasonable first attempt consisting in proceeding by induction on
$\musti{\server}{\client}$ fails, as demonstrated by the following example.

\begin{example}
  \label{ex:must-set-is-helpful}
%  Recall the state $\server$ of \rexa{set-transitions}
  Consider the two servers $\server = \tau.(\co{\aa} \Par \co{b}) \extc
  \tau.(\co{\aa} \Par \co{c})$ and $\serverB = \co{\aa} \Par (\tau.\co{b} \extc
  \tau.\co{c})$ of \req{mailbox-hoisting}.
  Fix a client $\client$ such that $\musti{\serverA}{\client}$.
  Rule induction %on this fact
  yields the following inductive hypothesis:
  %on $\musti{\serverA}{\client}   tells us the
  \begin{center}
    $\forall \serverA', \serverB' \suchthat\;
    \csys{\serverA}{\client} \st{\tau} \csys{\serverA'}{\client'} \;\land\;$
    $\serverA' \asleq \serverB' \;\Rightarrow\; \musti{\serverB'}{\client'}$.
  \end{center}
  %% Showing that $\serverA \asleq \serverB$ implies $\serverA \testleqS \serverB$,
  %% \ie the soundness direction,
  %would among us to pick a client $\client$ such that $\musti{\serverA}{\client}$
  %and prove $\musti{\serverB}{\client}$.
  In the proof of
  $\musti{\serverB}{\client}$ we have to consider the case where there
  is a communication between $\serverB$ and
  $\client$ such that, for instance, $\serverB \st{\co{\aa}}
  \tau.\co{b} \extc \tau.\co{c}$ and $\client \st{\aa}
  \client'$.  In that case, we need to show that $ \musti{ \tau.\co{b}
    \extc \tau.\co{c}
  }{\client'}$. Ideally, we would like to use the inductive
  hypothesis. This requires us to exhibit a $\server'$ such that $
  \csys{\server}{\client} \st{\tau} \csys{\server'}{\client'}$ and $
  \server' \bhvleqtwo \tau.\co{b} \extc \tau.\co{c}$.
%\ilacom{What is $\accleqset$? I see now it is defined below.}
  However, note that there is no way to derive
  $\csys{\server}{\client} \st{\tau} \csys{\server'}{\client'}
  $, because $\server
  \Nst{\co{\aa}}$.  The inductive hypothesis thus cannot be applied,
  and the naïve proof does not go through.\hfill$\qed$
  %%% RATIONALE
  %% Note that, as~$\opMusti$ is defined on strong transition relations,
  %% the inductive hypothesis works \textit{one transition at a time}.
\end{example}
\noindent
This example suggests that defining an auxiliary predicate~$\opMustset$ in some sense
equivalent to~$\opMusti$, but that uses explicitly {\em weak} outputs
of servers, should be enough to prove that~$\asleq$ is sound wrt~$\testleqS$.
Unfortunately, though, there is an additional nuisance to tackle: server
non-determinism.
\begin{example}
  %  Following the intuition outlined thus far,
  %% Observe, though, that $ \server \wt{ \co{\aa}}$. This intuition
  %% motivates the use of the weak transition $X \wt{\co{\mu}} X'$ in \rfig{rules-mustset}.
  %% need for relying on weak transitions instead of strong transitions.
  %% To show why the use of sets is necessary,
  Assume that we defined the predicate~$\opMusti$
  using weak transitions on the server side for the case of
  communications. Recall the argument %unfolded
  put forward in the previous example.
  The inductive hypothesis now becomes the following:
  \begin{center}
    For every $\serverA', \serverB', \mu$ such that
    $\serverA \wt{\mu} \serverA'$ and $\client \st{\mu} \client'$,
    $\serverA' \asleq \serverB'$ implies $\musti{\serverB'}{\client'}$.
  \end{center}
  To use the inductive hypothesis we have to choose a $\server'$ such
  that $\server \wt{\co{\aa}} \server'$ and $\server' \asleq
  \tau.\co{b} \extc \tau.\co{c}$. This is still not enough for the
  entire proof to go through, because (modulo further $\tau$-moves)
  the particular $\server'$ we pick has to be related also to either
  $\co{b}$ or $\co{c}$. It is not possible to find such a
    $\server'$, because
    the two possible candidates
  %choose such a $\server'$, because the particular $\server'$ we can
  %choose
  are either $\co{b}$
  or $\co{c}$; neither of which can satisfy $\server' \asleq
  \tau.\co{b} \extc \tau.\co{c}$, as the right-hand side has not
  committed to a branch yet.

  %% the server $\tau.\co{b} \extc \tau.\co{c}$ has
  %% not commited yet to any branch.%  did not already decide which
  %% %  branch it would take.
  %% \leo{I think I understand by it's not super clear: is the following correct?
  %%   ... the particular $\server'$ we choose will either correspond to $\co{a} \Par \co{b}$
  %%   or to $\co{a} \Par \co{c}$; neither of which can satisfy $\server' \asleq
  %% \tau.\co{b} \extc \tau.\co{c}$, as the right-hand side has not commited to a
  %% branch yet.}

  If instead of a single state $\server$ in the novel definition of
  $\opMusti$ we used a set of %servers
  states and a suitable
  transition relation, the choice of either $\co{b}$ or $\co{c}$ will be
  suitably delayed. It suffices for instance to have the following states and transitions:
  %$X = \set{\co{b}, \co{c}}$ we have that
  $\set{\serverA} \wt{\co{\aa}} \set{\co{b}, \co{c}}.$\hfill$\qed$
  %and
%  $X \asleqset \tau.\co{b} \extc \tau.\co{c}$.
%  , for instance $X' =
%  \set{\co{b}, \co{c}}$
  %% and we defined an LTS such that $X \wt{
  %%   \co{\mu}} X'$ then in the proof the choice of either $\co{b}$ or $
  %% \co{c}$ would be suitably delayed.
  %% %In contrary, the use of sets allows to
  %% delay this choice. Indeed, by taking
  %% $X = \set{\co{b}, \co{c}}$ we have that $\set{\serverA} \wt{\co{\aa}} X$ and
  %% $X \asleqset \tau.\co{b} \extc \tau.\co{c}$.\hfill$\qed$
\end{example}

Now that we have motivated the main intuitions behind the definition of our
novel auxiliary predicate~$\opMustset$, we proceed with the formal definitions.

{\bfseries The LTS of sets.}
Let~$\pparts{ Z }$ be the set of  {\em non-empty} parts of~$Z$.
Given an LTS~$\lts{\States}{ L }{ \st{} }$, we define
for every $ X \in \pparts{ \States } $ and every $\alpha$ the set
$D{(\alpha, X)} = \setof{ \stateA }{ \exists \state \in X \suchthat \state \st{\alpha} \stateA }$
\pl{and $WD{(\alpha, X)} = \setof{ \stateA }{ \exists \state \in X \suchthat \state \wt{\alpha} \stateA }$}.
Essentially we lift the standard notion of state derivative to sets of states.
We construct the LTS $\lts{\pparts{ \States }}{ \Acttau }{ \st{} }$
by letting $ X \st{ \alpha } D{(\alpha, X)}$ whenever $D{(\alpha, X)} \neq \emptyset$.
\pl{Similarly, we have $X \wt{ \alpha } WD{(\alpha, X)}$ whenever $WD{(\alpha, X)} \neq \emptyset$}.
%, \ie whenever
%and there exists a $\server \in X$ such that $\server \st{ \alpha }$.

%%%%%%%%%%%%%%%%%%%%%%%%%%%%%%%%%%%%%%%%%%%%%%%%%%%%%%%%%%%%%%%%%%%%%%%%%%%%
%%%%%%%%%%%%%%%%%%%%%%%%%%%%%%%%%%%%%%%%%%%%%%%%%%%%%%%%%%%%%%%%%%%%%%%%%%%%
%%%%%%%%%%%%%%%%%%%%%%%%%%%%%%%%%%%%%%%%%%%%%%%%%%%%%%%%%%%%%%%%%%%%%%%%%%%%
\leaveout{
\begin{example}
  \label{ex:set-transitions}
    Let $\server = \tau.(\co{\aa} \Par \co{b}) \extc \tau.(\co{\aa} \Par \co{c})$.
    We have
    $$
    \begin{array}{l}
      \set{ \server } \st{ \tau} \set{ \co{\aa} \Par \co{b} }\\
      \set{ \server } \st{ \tau} \set{ \co{\aa} \Par \co{c} }\\
      \set{ \server } \st{ \tau} \set{ \co{\aa} \Par \co{b}, \co{\aa} \Par \co{c} }
      \end{array}
    $$
    \hfill$\qed$
  \end{example}
  In our arguments we will reason on the fact for a given $X \in \pparts{ \States }$
  all the states $p \in X$ must pass a given client $r$.
  The predicate $\opMusti$ is not directly usable to this aim, because
  in general $ \musti{X}{\client} $ does not imply that $\forall
  \server \musti{X}{\client}$. We show this in the next example.
  \begin{example}
    \label{ex:musti-not-directly-liftable-LTS-of-sets}
    Let $X = \set{\Nil, \co{\aa}}$ and $\client = \aa.\Unit$.
    We have that $\musti{X}{\client}$, because $X \st{\co{\aa}}$ and $X
    \st{\co{\aa}} Y$ implies that $Y = \set{ \Nil }$  and
    $$
    \begin{prooftree}
      {\good{\Unit}}
      \justifies
          {\begin{prooftree}
              \musti{\Nil}{\Unit}
              \justifies
                  { \musti{ \set{\Nil, \co{\aa}} }{\aa.\Unit} }
            \end{prooftree}%
          }
    \end{prooftree}
    $$
    On the other hand $\Nil \in X$  and $\Nmusti{\Nil}{\client}
    $.\hfill$\qed$
  \end{example}
%%   It seems natural that a server $\server$ and the singleton
%%   $\set{\server}$ be interchangeable when reasoning on~$\opMusti$,
%%   \ie $\musti{\server}{\client}$ if and only if
%%   $\musti{\set{\server}}{\client}$ for every~$\client$.

%% \TODO{If this is true, then there is the more general question:
%%   is it true that for every $X\in\pparts{A}$ .
%%   ($\forall \server \in X . \musti{\server}{\client}$) if and only if $\musti{X}{ \client}$ ?
\noindent

}
%%%%%%%%%%%%%%%%%%%%%%%%%%%%%%%%%%%%%%%%%%%%%%%%%%%%%%%%%%%%%%%%%%%%%%%%%%%%
%%%%%%%%%%%%%%%%%%%%%%%%%%%%%%%%%%%%%%%%%%%%%%%%%%%%%%%%%%%%%%%%%%%%%%%%%%%%
%%%%%%%%%%%%%%%%%%%%%%%%%%%%%%%%%%%%%%%%%%%%%%%%%%%%%%%%%%%%%%%%%%%%%%%%%%%%
Let $\opMustset$ be defined via the rules in \rfig{rules-mustset}.
This predicate let us reason on~$\opMusti$ via sets of servers,
in the following sense.
\begin{lemma}
  \label{lem:musti-if-mustset-helper}
  For every LTS $\genlts_A, \genlts_B$ and every
  $X \in \pparts{\StatesA}$, we have that
  $\mustset{X}{\client}$ if and only if for every $\serverA \in X
  \wehavethat \musti{\serverA}{\client}$.
\end{lemma}
%%%%% HIDDEN BECAUSE PERHAPS THE LEMMA IS NOT WORTH BEING PRESENTED
%%%%%%%%%%%%%%%%%%%%%% DO NOT EREASE
%%%%%%%%%%%%%%%%%%%%%% DO NOT EREASE
%% \begin{proof}
%% We proceed by rule induction on the derivation of the hypothesis $\mustset{X}{\client}$.

%% In the base case, $\mustset{X}{\client}$ has been derived using the rule \msetnow
%% such that $\good{\client}$ and $\musti{\server}{\client}$ follows from
%% an application of \mnow. %
%% %
%% %
%% In the inductive case $\mustset{X}{\client}$ has been derived using
%% rule \msetstep, and thus
%% \begin{enumerate}[(a)]
%% \item\label{musti-if-mustset-helper-h0} $\lnot \good{\client}$,
%% \item\label{musti-if-mustset-helper-h1} for every $\serverA \in X$ we have that
%%   $\csys{ \serverA }{ \client } \st{\tau}$,
%% \item\label{musti-if-mustset-helper-h2}
%%   $\forall X' \wehavethat X \st{ \tau } X' \neq \emptyset \implies \mustset{X'}{\client}$
%% \item\label{musti-if-mustset-helper-h3}
%%   $\forall \client' \wehavethat \client \st{ \tau } \client' \implies \mustset{X}{\client'}$
%% \item\label{musti-if-mustset-helper-h4}
%%   $\forall X', \mu \in \Actfin  \wehavethat X \wt{\co{\mu}} X' \neq \emptyset$
%%   and $\client \st{\mu} \client'$ imply that $\mustset{ X' }{ \client'}$
%% \end{enumerate}

%% Pick a $\server \in X$.
%% Since $\lnot \good{\client}$ and thanks to
%% (\ref{musti-if-mustset-helper-h1}), to show
%% $\musti{\serverA}{\client}$ we can apply the rule \mstep,
%% if we prove that
%% %%\begin{enumerate}[(i)]
%% %  \item $\lnot \good{\client}$,
%% %\item $\csys{ \serverA }{ \client } \st{\tau}$, and that
%% %%\item
%% \begin{equation*}
%%   \forall \serverA',
%%   \client' \wehavethat \csys{\serverA}{\client} \st{\tau}
%%   \csys{\serverA'}{\client'} \text{ imply that }\mustset{X'}{\client'}.
%%   \end{equation*}
%% %%\end{enumerate}
%% %The first requirement is a direct consequence of (\ref{musti-if-mustset-helper-h0}).
%% %% The first requirement follows from the fact that $\serverA \in X$ together with
%% %% (\ref{musti-if-mustset-helper-h1}).
%% %%To show the third requirement
%% Fix a silent transition $\csys{\serverA}{\client} \st{\tau} \csys{\serverA'}{\client'}$. We proceed by case analysis on how the transition
%% $\csys{\serverA}{\client} \st{\tau} \csys{\serverA'}{\client'}$ has been derived.
%% There are three cases, either the transition has been derived. There are the three following cases.
%% \begin{enumerate}[(1)]
%% \item[\stauserver]
%%   a $\tau$-transition performed by the server such that $\serverA \st{\tau} \serverA'$
%%   and that $\client' = \client$, or
%% \item[\stauclient]
%%   a $\tau$-transition performed by the client such that $\client \st{\tau} \client'$
%%   and that $\server' = \server$, or
%% \item[\scom]
%%   an interaction between the server $\serverA$ and the client $\client$ such that
%%   $\serverA \st{\mu} \serverA'$ and $\client \st{\co{\mu}} \client'$.
%% \end{enumerate}

%% \TODO{What does the inductive hypothesis state ?}

%% In case \stauserver we apply the inductive hypothesis. To do so we
%% have to provide a set $X'$ such that $\serverA' \in X'$ and $X \st{\tau} X'$.
%% We take $X' = \setof{\hat{\server}}{\serverA \st{\tau} \hat{\server}}$.
%% As $\serverA \st{\tau} \serverA'$, it must be the case that $\serverA' \in X'$ and we are done.
%% In case \stauclient, then conclusion is a direct consequence of the inductive hypothesis.
%% The argument for case \scom is similar to the one for case
%% \stauserver, the only difference being that we take $X' = \setof{\hat{\server}}{\serverA \st{\mu} \hat{\server}}$.
%% \end{proof}
%%%%%%%%%%%%%%%%%%%%%% DO NOT EREASE
%%%%%%%%%%%%%%%%%%%%%% DO NOT EREASE


%% \TODO{What about the converse implication ? Is the following lemma true ?
%% \begin{lemma}
%%   \label{lem:musti-if-mustset-helper-inverse}
%%   For every LTS $\genlts_A, \genlts_B$, %
%%   every $X \in \pparts{\StatesA}$, %
%%   and every $\client \in \StatesB$,
%%   if $\musti{\serverA}{\client}$ for every $\serverA \in X$,
%%   then $\mustset{X}{\client}$.
%% \end{lemma}
%% }


%%%%%%%%%%%%%%%%%%%%%% THE PREVIOUS LEMMA IS THE INTERESTING ONE
%%%%%%%%%%%%%%%%%%%%%%
%% \begin{lemma}
%%   \label{lem:musti-equals-mustset}
%%   For every LTS $\genlts_A, \genlts_B$, every
%%   $\server \in \StatesA$ and every $\client \in \StatesB$,
%%   $\musti{\server}{\client}$ if and only if $\mustset{\set{\server}}{\client}$.
%% \end{lemma}
%%%%%%%%%%%%%%%%%%%%%% DO NOT EREASE
%%%%%%%%%%%%%%%%%%%%%% DO NOT EREASE
%% \pl{
%% \begin{proof}
%% We first prove that
%% $\musti{\server}{\client}$ implies $\mustset{\set{\server}}{\client}$.
%% We proceed by rule induction on the derivation of $\mustset{\server}{\client}$.
%% In the base case it has been derived using \mnow, and thus $\good{\client}$,
%% which lets us conclude the base case by an applicaton of \mnow.
%% In the inductive case we show $\mustset{\set{\server}}{\client}$ by applying the rule \mstep.
%% We need to show the following.
%% \begin{enumerate}[(i)]
%% \item\label{musti-if-mustset-helper-g0} $\lnot \good{\client}$,
%% \item\label{musti-if-mustset-helper-g1} $\forall \serverA \in \set{\serverA}$ we have that
%%   $\csys{ \serverA }{ \client } \st{\tau}$,
%% \item\label{musti-if-mustset-helper-g2}
%%   $\forall X'$ such that $\set{\serverA} \st{ \tau } X' \neq \emptyset$, $\mustset{X'}{\client}$
%% \item\label{musti-if-mustset-helper-g3}
%%   $\forall \client'$, such that $\client \st{ \tau } \client'$, $\mustset{\set{\serverA}}{\client'}$
%% \item\label{musti-if-mustset-helper-g4}
%%   $\forall X', \mu \in \Act$,  $\set{\serverA} \wt{\co{\mu}} X' \neq \emptyset$
%%   and $\client \st{\mu} \client'$ imply $\mustset{ X' }{ \client'}$
%% \end{enumerate}

%% The first and the second requirements are a consequence of
%% the derivation of $\mustset{\server}{\client}$ by the rule \mstep.
%% To show (\ref{musti-if-mustset-helper-g2}) we apply \rlem{mx-forall} which requires us to show
%% that for each $\serverA'$, $\serverA \in X'$ implies $\mustset{\set{\serverA'}}{\client}$,
%% which is ensured by the inductive hypothesis.
%% The third requirement follows directly from an application of the inductive hypothesis.
%% The last requirement requires a bit more work.
%% We apply \rlem{mx-forall} which requires us to show
%% that for each $\serverA'$, $\serverA \in X'$ we have that
%% $\mustset{\set{\serverA'}}{\client'}$.
%% Let us pick such $\serverA'$. From the hypothesis $\set{\serverA} \wt{\co{\mu}} X'$
%% we know that $\serverA \wt{\mu} \serverA'$.
%% We continue by case analysis on the derivation of the reduction
%% $\serverA \wt{\mu} \server'$. There are two different cases to consider.
%% \begin{enumerate}
%% \item[\rname{wt-tau}]
%%   An application of \rptlem{mx-preservation}{mx-preservation-wt-mu} together
%%   with the inductive hypothesis ensures $\mustset{\set{\server'}}{\client'}$
%%   as required.
%% \item[\rname{wt-mu}]
%%   An application of \rptlem{mx-preservation}{mx-preservation-wt-nil} together
%%   with the inductive hypothesis ensures $\mustset{\set{\server'}}{\client'}$
%%   as required.
%% \end{enumerate}

%% The second direction, \ie $\mustset{\set{\server}}{\client}$ implies $\musti{\server}{\client}$,
%% is a direct consequence of \rlem{musti-if-mustset-helper}.
%% \end{proof}
%%%%%%%%%%%%%%%%%%%%%% DO NOT EREASE
%%%%%%%%%%%%%%%%%%%%%% DO NOT EREASE


  %% This definition of $\opMustset$ is notably helpful for our proof of soundness
  %% (\rlem{soundness-set}), in particular when we have to tackle
  %% the case where there is a communication between the server $\serverB$
  %% and the client $\client$ (\rptlem{soundness-set}{aim-soundness-3}).
%  We illustrate our reasoning through the following example to motivate the
%  use of the weak reduction as well as the use of sets.

  %% The next example illustrates why in rule \msetstep\ we use the weak
  %% transitions $X \wt{\co{ \mu }} X'$, and how the use of sets of
  %% servers help us deal with non-determinism.





To lift the predicates~$\bhvleqone$ and~$\bhvleqtwo$ to sets of servers,
we let $\accht{ X }{ \trace } = \setof{ O }{ \exists \server \in X . O
  \in \accht{\server}{ \trace }}$, and for every finite $X \in
\pparts{ \States }$, we write $X \conv$ to mean $\forall \state \in X \suchthat\state \conv$, we write $  X \cnvalong \trace$ to mean $\forall \state \in X \suchthat \state \cnvalong \trace$, and let

\begin{itemize}
\item $ X \cnvleqset \serverB$ to mean $\forall \trace \in \Actfin,
  \text{ if } X \cnvalong \trace
  \text{ then } \serverB \cnvalong \trace$,

\item
  $X \accleqset \serverB$ to mean $\forall \trace \in \Actfin,
  X \cnvalong{\trace} \implies \accht{ X }{ \trace } \ll \accht{ \serverB }{ \trace }$,

\item
  $X \asleqset \serverB$ to mean $X \cnvleqset \serverB \wedge X \accleqset \serverB$.
\end{itemize}
%%$$
%% \begin{array}{lcl}
%%   X \cnvleqset \serverB &\text{to mean}& \forall \trace \in \Actfin,
%%   \text{ if } X \cnvalong \trace
%%   \text{ then } \serverB \cnvalong \trace
%%   \\[5pt]
%%   %% X \accleqset \serverB & \text{to mean}&
%%   %% \forall \trace \in \Actfin,
%%   %% X \cnvalong{\trace} \implies\\
%%   %% && \forall O \in \accht{ \serverB }{ \trace } \wehavethat \\
%%   %% && \exists \serverA \in X \suchthat \\
%%   %% &&\exists \widehat{O} \in \accht{ \serverA }{ \trace } . \widehat{ O } \subseteq O
%%   %% \\[5pt]
%%   X \accleqset \serverB & \text{to mean}& \forall \trace \in \Actfin,
%%   X \cnvalong{\trace} \implies\\
%% &&  \accht{ X }{ \trace } \ll \accht{ \serverB }{ \trace }
%%   \\[5pt]
%%   X \asleqset \serverB & \text{to mean}& X \cnvleqset \serverB \wedge X \accleqset \serverB
%% \end{array}
%% $$

These definitions imply immediately the following equivalences,
  $
\set{\serverA} \accleqset \serverB \Longleftrightarrow p \bhvleqone \serverB$,
$\set{\serverA} \cnvleqset \serverB \Longleftrightarrow p \bhvleqtwo \serverB $
and thereby the following lemma.
\begin{lemma}
  \label{lem:alt-set-singleton-iff}
  For every LTS $\genlts_A, \genlts_B, \serverA \in \StatesA$,
  $\serverB \in \StatesB$,
  $\serverA \asleq \serverB$ if and only if $\set{\serverA} \asleqset \serverB$.
\end{lemma}
%% \begin{proof}
%%   The results follow directly from the following facts.
%% \end{proof}


%% \TODO{
%%   In \rlem{bhvleqone-preserved} and \rlem{bhvleqtwo-preserved}
%%   we have a stronger hypothesis than
%%   $X \wt{\mu} X'$ in the code.
%%   We say that for each
%%   $\serverA \in X$ if $\serverA \wt{\mu} \serverA'$ then $\serverA' \in X'$.
%%   I fix this tomorrow.
%% }


\newcommand{\completewrt}[2]{\ensuremath{\mathsf{cwrt} \, (#1,#2)}}

The preorder~$\asleqset$ is preserved by $\tau$-transitions on
its right-hand side, and by visible transitions
%actions performed
on both sides.
We reason separately on the two auxiliary
preorders $\cnvleqset$ and $\accleqset$.
We need one
further notion.
%% \gb{This should no longer be necessary.                                             %%
%% We say that a set $X'$ is {\em complete with respect                                %%
%%   to} a set $X$ and an action $\mu$, denoted                                        %%
%% $X' \completewrt{X}{\mu}$, if two properties hold: % are true:                      %%
%% \begin{enumerate}[i)]                                                               %%
%% \item $X \wt{ \mu } X'$ and                                                         %%
%% \item                                                                               %%
%%   for every $\server \in X$, if $\server \wt{\mu} \server'$ then $\server' \in X'$. %%
%% \end{enumerate}                                                                     %%
%% Note that if $X' \completewrt{X}{\mu}$ and $X''                                     %%
%% \completewrt{X}{\mu}$ then $X' = X''$, because the LTSs at hand are image finite.   %%
%% }                                                                                   %%
\begin{lemma}
  \label{lem:bhvleqone-preserved}
  Let $\genlts_\StatesA, \genlts_\StatesB \in \obaFW$.
  For every set $X \in \pparts{ \StatesA }$, and
  $\serverB \in \StatesB$, such that
  $X \cnvleqset \serverB$,
  \begin{enumerate}
  \item\label{pt:bhvleqone-preserved-by-tau}
    $\serverB \st{ \tau } \serverB'$ implies $X \cnvleqset \serverB'$,
  \item\label{pt:bhvleqone-preserved-by-mu}
    $X \convi$, $X \wt{\mu} X'$ and $\serverB \st{\mu} \serverB'$ imply $X' \cnvleqset \serverB'$.
  \end{enumerate}
\end{lemma}


\begin{lemma}
  \label{lem:bhvleqtwo-preserved}
  Let $\genlts_\StatesA, \genlts_\StatesB \in \obaFW$.
  For every
  $X, X' \in \pparts{ \StatesA }$ and $ \serverB \in \StatesB$,
  such that $X \accleqset \serverB$, then
  \begin{enumerate}
  \item\label{pt:bhvleqtwo-preserved-by-tau}
    $\serverB \st{ \tau } \serverB'$ implies $X \accleqset \serverB'$,
  \item\label{pt:bhvleqtwo-preserved-by-mu}
    \pl{if $X \convi$ then for every $\mu \in \Act$, every $\serverB'$ and $X'$ such that $\serverB \st{\mu} \serverB'$ and $X \wt{\mu} X'$
    we have $X' \accleqset \serverB'$.}
  \end{enumerate}
\end{lemma}
%% \begin{proof}
%%   We first prove \rptlem{bhvleqtwo-preserved}{bhvleqtwo-preserved-by-tau}.
%%   We must show $X \accleqset \serverB'$.
%%   Let us fix a server $\serverB''$ such that
%%   $X \cnvalong \trace$ and $\serverB' \wt{\trace} \serverB'' \stable$.
%%   An application of the hypothesis $X \accleqset \serverB$
%%   provides us with two servers $\serverA' \in X$ and $\serverA''$ such that
%%   $\serverA' \wt{s} \serverA'' \stable$ and $O(\serverA'') \subseteq O(\serverB'')$
%%   as required.

%%   We now prove \rptlem{bhvleqtwo-preserved}{bhvleqtwo-preserved-by-mu}.
%%   Let us fix a server $\serverB''$ such that
%%   $X' \cnvalong \trace$ and $\serverB' \wt{\trace} \serverB'' \stable$.
%%   An application of the hypothesis $X \accleqset \serverB$
%%   requires us to show $X \cnvalong \mu.\trace$ which follows
%%   from $X \convi$ and $X' \cnvalong \trace$.
%%   We then obtain two servers $\serverA \in X$ and $\serverA''$ such that
%%   $\serverA \wt{\mu.\trace} \serverA'' \stable$ and $O(\serverA'') \subseteq O(\serverB'')$
%%   We decompose the reduction $\serverA \wt{\mu.\trace} \serverA''$ and obtain
%%   a server $\serverA'$ such that
%%   $\serverA \wt{\mu} \serverA' \wt{\trace} \serverA''$ and allows us
%%   to conclude this case.
%% \end{proof}



%%%%%%%%%%%%%%%%%%% HIDDEN FOR SPACE REASONS
%%%%%%%%%%%%%%%%%%%
%% We now gather sufficient conditions for the servers on the
%% right-hand side of~$\asleqset$ to perform a silent move,
%% and for servers on the left-hand side of~$\asleqset$ to perform
%% weakly visible actions.

%% \begin{lemma}%[\coqSou{unhappy_must_st_nleqx}]
%%   \label{lem:stability-Nbhvleqtwo}
%%   Let $\genlts_\StatesA, \genlts_\StatesB \in \obaFW$ and $\genlts_\StatesC \in \obaFB$.
%%   For every $X \in \pparts{ \StatesA }$ and
%%   $\serverB \in \StatesB $ such that
%%   $X \asleqset \serverB$, for every $\client \in \StatesC$
%%   if $\lnot \good{\client}$ and $\mustset{X}{\client}$
%%   then $\csys{\serverB}{\client} \st{\tau}$.
%% \end{lemma}


%% \gb{
%% \begin{lemma}
%%   \label{lem:empty-nleqx}
%%   Let $\genlts_\StatesA, \genlts_\StatesB \in \obaFW$.
%%   For every $X \in \pparts{ \StatesA }$ and
%%   $\serverB, \serverB' \in \StatesB$,
%%   such that $X \cnvleqset \serverB$, then
%%   for every $\mu \in \Act$, if
%%   $X \convi$, $X \accleqset \serverB$ and  $\serverB \st{\mu}
%%   \serverB'$
%%   then $X \Nwt{\mu}$.
%% \end{lemma}
%% \begin{proof}
%%   First note that $X \convi$ and $X \Nwt{\mu}$ imply $X \cnvalong \mu$.
%%   Then, from $X \cnvleqset \serverB$ and $X \cnvalong \mu$ we have that
%%   $\serverB \cnvalong \mu$ and thus $\serverB' \convi$.
%%   As $\serverB'$ terminates, it must be the case that
%%   there exists a $\serverB''$ such that
%%   $\serverB \wt{\mu} \serverB' \wt{\varepsilon} \serverB'' \stable$.
%%   We then apply the hypothesis $X \accleqset \serverB$ to obtain
%%   two servers $\serverA' \in X$ and $\serverA''$ such that
%%   $\serverA' \wt{\mu} \serverA'' \stable$ which contradicts the hypothesis
%%   $X \Nwt{\mu}$ and we are done.
%% \end{proof}
%% }


%% \begin{lemma}
%%   \label{lem:empty-nleqx}
%%   \label{lem:X-must-perform-visible-action}
%%   Let $\genlts_\StatesA, \genlts_\StatesB \in \obaFW$.
%%   For every $X \in \pparts{ \StatesA }$ and
%%   $\serverB, \serverB' \in \StatesB$,
%%   such that $X \cnvleqset \serverB$, then
%%   for every $\mu \in \Act$, if
%%   $X \convi$,  $X \Nwt{\mu}$ and $\serverB \st{\mu} \serverB'$ then $\lnot (X \accleqset \serverB)$.
%% \end{lemma}
%%%%%%%%%%%%%%%%%%%%%%%%%%%%%%%%%%%%%%%%%%%%%%% OLD LEMMA PROVEN VIA A CONTRADICTION
%%%%%%%%%%%%%%%%%%%%%%%%%%%%%%%%%%%%%%%%%%%%%%% OLD LEMMA PROVEN VIA A CONTRADICTION
%%%%%%%%%%%%%%%%%%%%%%%%%%%%%%%%%%%%%%%%%%%%%%% OLD LEMMA PROVEN VIA A CONTRADICTION
%% \begin{lemma}%[\coqSou{unhappy_must_st_nleqx}]
%%   \label{lem:stability-Nbhvleqtwo}
%%   Let $\genlts_\StatesA$, $\genlts_\StatesB$ be $\obaFW$ and $\genlts_\StatesC$ be $\obaFB$.
%%   For every $X \in \pparts{ \StatesA }$, $\serverB \in \pparts{ \StatesB }$ and $\client \in \StatesC$,
%%   if $\lnot \good{\client}$ and $\mustset{X}{\client}$ then
%%   $\csys{\serverB}{\client} \stable$ implies $\lnot (X \asleqset \serverB)$.
%% \end{lemma}
%% \begin{proof}
%%   We show that assuming $X \asleqset \serverB$ leads to a contradiction and thus a derivation of $\bot$.
%%   First, from the hypothesis $\csys{\serverB}{\client} \stable$ we know the following statements hold.
%%   \begin{enumerate}[(a)]
%%   \item\label{stability-Nbhvleqtwo-h1}
%%     $\serverB \stable$
%%   \item\label{stability-Nbhvleqtwo-h2}
%%     $\client \stable$
%%   \item\label{stability-Nbhvleqtwo-h3}
%%     there is no $\mu \in \Act$ such that $\serverB \st{\mu}$ and $\client \st{\co{\mu}}$
%%   \end{enumerate}

%%   The hypotheses $\lnot \good{\client}$ and $ \mustset{X}{\client}$ together with
%%   \rlem{mustx-terminate-ungood} imply $X \convi$ and thus $X \cnvalong \varepsilon$.
%%   An application of $X \accleqset q$ with $\trace = \varepsilon$, $\stateB' = \stateB$,
%%   and (\ref{stability-Nbhvleqtwo-h1}) gives us a $\serverA'$ such that
%%   $\serverA \wt{\varepsilon} \serverA' \stable$ and $O(\serverA') \subseteq O(\serverB)$.
%%   An application of $\mustset{X}{\client}$, \rlem{wt-nil-mx} and \rlem{mx-sub} ensures that
%%   $\mustset{\set{\serverA'}}{\client}$.
%%   As $\lnot \good{\client}$, it must be the case that $\mustset{\set{\serverA'}}{\client}$
%%   is derived using the rule \mstep which implies that $\csys{\serverA'}{\client} \st{\tau}$.
%%   We contine with a case analysis on how this transition has been derived.
%%   As $\serverA'$ is stable, and from (\ref{stability-Nbhvleqtwo-h2}) we know that
%%   it must be derived using \rname{s-com} such that
%%   $\serverA' \st{\mu}$ and $\client \st{\co{\mu}}$ for some $\mu \in \Act$.
%%   Finally, we distinguish whether $\mu$ is an input or an output.
%%   In the first case $\mu$ is an input.
%%   From $O(\serverA') \subseteq O(\serverB)$ it must be the case that $\serverB \st{\co{\mu}}$
%%   holds, which contradicts the hypothesis that $\csys{\serverB}{\client} \stable$.
%%   In the second case $\mu$ is an output.
%%   As $\serverB$ is $\obaFW$, we can apply the axiom for forwarders and get a server $\stateB'$
%%   such that $\serverB \st{\co{\mu}}$. This implies the existence of a communication
%%   between $\serverB$ and $\client$, which contradicts
%%   the hypothesis that $\csys{\serverB}{\client} \stable$ and lets us conclude this case.
%% \end{proof}
%%%%%%%%%%%%%%%%%%%%%%%%%%%%%%%%%%%%%%%%%%%%%%%%%%%%%%%%%%%%%%%%%%%%%%%%%%%%%%%%%%%%%%%%%%%%%%%%%%%
%%%%%%%%%%%%%%%%%%%%%%%%%%%%%%%%%%%%%%%%%%%%%%%%%%%%%%%%%%%%% END OLD LEMMA

%%%%%%%%%%%%%%%%%%% HIDDEN FOR SPACE REASONS
%%%%%%%%%%%%%%%%%%%


The main technical work for the proof of soundness is carried out by the next lemma.
\begin{lemma}
  \label{lem:soundness-set}
  Let $\genlts_\StatesA, \genlts_\StatesB \in \obaFW$ and
  $\genlts_\StatesC \in \obaFB$.
  For every set of servers $X \in \pparts{ \StatesA }$,
  server $\serverB \in \StatesB$ and client $\client \in \StatesC$,
  if $\mustset{X}{\client}$ and $X \asleqset \serverB$ then $\musti{\serverB}{\client}$.
\end{lemma}



\begin{proposition}[Soundness]
  \label{prop:bhv-soundness}
  For every $\genlts_A, \genlts_B \in \obaFB$ and
  servers $\serverA \in \StatesA, \serverB \in \StatesB $,
  if $\liftFW{ \serverA } \asleq \liftFW{ \serverB }$ then $\serverA \testleqS \serverB$.
\end{proposition}
\begin{proof}
\rcor{testleq-obafb-iff-testleq-obafw} ensures that the result follows if we prove that
$\liftFW{\serverA} \testleqS \liftFW{\serverB}.$
Fix a client $\client$ such that $\musti{\liftFW{\serverA}}{\client}$.
\rlem{soundness-set} implies the required
$\musti{\liftFW{\serverB}}{\client}$, if we show that
\begin{enumerate}[(i)]
  \item $\mustset{ \set{ \liftFW{\serverA} } }{ \client }$, and that
  \item $\set{ \liftFW{ \serverA } } \mathrel{\preccurlyeq^{\mathsf{set}}_{\mathsf{AS}}} \liftFW{\serverB}$.
\end{enumerate}
The first fact follows from the assumption that $\musti{\liftFW{\serverA}}{\client}$
and \rlem{musti-if-mustset-helper} applied to the singleton
$\set{\liftFW{\serverA}}$.
The second fact follows from the hypothesis that $\liftFW{ \serverA }
\asleq \liftFW{ \serverB }$ and \rlem{alt-set-singleton-iff}.
%%%% ORIGINAL PROOF
  %% Fix a client $\client$ such that $\musti{\liftFW{\serverA}}{\client}$.
  %% The required $\musti{\liftFW{\serverB}}{\client}$ follows from
  %% \rlem{soundness-set}, if we show that
  %% $\set{\serverA} \asleq \serverB$.
  %% We obtain $\set{\serverA} \asleq \serverB$ by
  %% an application of \rlem{alt-set-singleton-iff} in the hypothesis $\serverA \asleq \serverB$.
  %% The requirement $\musti{\set{\liftFW{\serverA}}}{\client}$ is a consequence of
  %% an application of \rlem{musti-equals-mustset} together with the hypothesis that
  %% $\musti{\liftFW{\serverA}}{\client}$.
\end{proof}
