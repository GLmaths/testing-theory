\section{Completeness}
\label{sec:proof-completeness}
\label{sec:bhv-completeness}


\begin{table*}
  \hrulefill\\

  % \begin{tabular}{l}
    \begin{minipage}{300pt}%
      $\forall \trace \in \Actfin, \forall \aa \in \Names,$
      \begin{enumerate}[(1)]
      \item
        \label{gen-spec-ungood}
        $ \lnot \good{f(\trace)}$
      \item
        \label{gen-spec-mu-lts-co}
        $ \forall \mu \in \Act, f (\mu.\trace) \st{\co{\mu}} f(\trace)$
      \item
        \label{gen-spec-mu-out-ex-tau}
        $ f (\co{a}.\trace) \st{\tau} $
      \item
        \label{gen-spec-out-good}
        $ \forall \client \in \States, f (\co{a}.\trace)
        \st{\tau} \client$ implies $\good{ \client }$
      \item
        \label{gen-spec-out-mu-inp}
        $ \forall \client \in \States, \mu \in \Act,$
        $f (\co{a}.\trace) \st{\mu} \client$ implies $\mu = \aa$ and
        $\client = f(s)$
      \end{enumerate}
    \end{minipage}
    \\[2em]
    \begin{minipage}{300pt}
      $\forall E \subseteq \Names$,
      \begin{enumerate}[(t1)]
      \item\label{gen-spec-acc-nil-stable-tau}
        $\testacc{\varepsilon}{E} \Nst{\tau}$
      \item\label{gen-spec-acc-nil-stable-out}
        $\forall \aa \in \Names, \testacc{\varepsilon}{E} \Nst{\co{\aa}}$
      \item\label{gen-spec-acc-nil-mem-lts-inp}
        $\forall \aa \in \Names, \testacc{\varepsilon}{E} \st{\aa}$ if and only if
        $\aa \in E$
      \item\label{gen-spec-acc-nil-lts-inp-good}
        $\forall \mu \in \Act, \client \in \States,
        \testacc{\varepsilon}{E} \st{\mu} \client$ implies $\good{\client}$
      \end{enumerate}
    \end{minipage}
  % \end{tabular}
  \\[2em]
   %%%%%%%%
  %%%%%%%% HORIZONTAL LAYOUT BY GIO
 %%%%%%%%
%% %% \multicolumn{3}{l}{%
%%   \begin{enumerate}[(c1)]
%% \item
%%   $\forall \mu \in \Act, \testconv{\varepsilon} \Nst{\mu}$ \qquad\qquad
%% \item
%%   $\testconv{\varepsilon} \st{\tau} $ \qquad\qquad
%% \item
%%   $\forall \client, \testconv{\varepsilon} \st{\tau} \client$ implies
%%   $\good{ \client }$
%%   \end{enumerate}
%%   \\[7pt]
%% %%  }
  %%%%%%%%
  %%%%%%%% VERTICAL LAYOUT BY ILARIA
  %%%%%%%%
    % \begin{tabular}{c}
      \begin{minipage}{300pt}
      \begin{enumerate}[(c1)]
      \item
        $\forall \mu \in \Act, \testconv{\varepsilon} \Nst{\tau}$ \qquad
      \item
        $\exists \client, \testconv{\varepsilon} \st{\tau} \client$ \qquad
      \item
        $\forall \client, \testconv{\varepsilon} \st{\tau} \client$ implies
        $\good{ \client }$
        \\
      \end{enumerate}
      \end{minipage}
    % \end{tabular}
  %%%%%%%%
%%%%%%%% VERTICAL LAYOUT BY ILARIA
%%%%%%%%
  \caption{Properties of the functions that generate clients.}
\hrulefill
\label{tab:properties-functions-to-generate-clients}
\end{table*}





This section is devoted to the proof that the alternative preorder
given in \rdef{accset-leq} includes the \mustpreorder.
First we present a general outline of the main technical results to obtain
the proof we are after. Afterwards, in Subsection~(\ref{sec:appendix-completeness})
we discuss in detail on all the technicalities.


Proofs of completeness of characterisations of contextual preorders usually
require using, as the name suggests, syntactic contexts.
Our calculus-independent setting, though, does not allow us to
define them.  Instead we phrase our arguments using two functions
$
\testconvSym :  \Actfin \rightarrow \States,$ and
$\testaccSym :  \Actfin \times \parts{\Names} \rightarrow \States$
where \lts{ \States }{L}{ \st{} } is some LTS of \obaFB.
In \rtab{properties-functions-to-generate-clients} we gather all the
{\em properties} of~$\testconvSym$ and~$\testaccSym$ that are sufficient to give our
arguments. The properties (1) - (5) must hold for both~$\testconvSym$ and
$\testacc{\varepsilon}{-}$ for every set of names~$O$, the properties (c1) -
(c2) must hold for~$\testconvSym$, and (t1) - (t4) must hold for~$\testaccSym$.

We use the function~$\testconvSym$ to test the convergence of servers, and the
function~$\testaccSym$ to test the acceptance sets of servers.

A natural question is whether such~$\testconvSym$ and~$\testaccSym$ can actually exist.
The answer depends on the LTS at hand. In \rapp{client-generators},
and in particular \rfig{client-generators}, we define these functions for
the standard LTS of~\ACCS, and it should be obvious how to adapt those
definitions to the asynchronous $\pi$-calculus \cite{DBLP:journals/jlp/Hennessy05}.
%\pl{Two times obvious in this par. Maybe we could simply link
%  papers that define test generators for the $\pi$-calculus}.

In short, our proofs show that~$\asleq$ is complete with respect to~$\testleqS$
in any LTS of output-buffered agents with feedback wherein the
functions~$\testconvSym$ and~$\testaccSym$ enjoying the properties in
\rtab{properties-functions-to-generate-clients} can be defined.


%%Instead, we define sets of properties the test generators must obey to.
%% \begin{definition}
%% We first define the properties that must be respected by the test generators
%% we rely on in this section.

%% Given a test generator $f : \Actfin \rightarrow \States$,
%% we write $\mathcal{M} \Vdash f(\trace}$ if


%% We now define the properties that must be respected
%% by the test generator used to test the convergence of
%% a server along a trace.

%% Given a test generator $c : \Actfin \rightarrow \States$,
%% we write $\mathcal{C} \Vdash \testconv{\trace}$ if
%% $\mathcal{M} \Vdash \testconv{\trace}$ and $c$ obey to these properties.

%% We now define the properties that must be respected by the test generator
%% used to test the acceptance-sets of a server.

%% Given a test generator $a : \set{\Names} \times \Actfin \rightarrow \States$,
%% we write $\mathcal{A} \Vdash \testacc{O}{\trace}$ if for any $O \subseteq \Names$ and $s \in \Actfin$,
%% $\mathcal{M} \Vdash \testacc{O}{\trace}$ and $a$ respects the following properties.



%% For the sake of clarity, we first provide the reader                     %%
%% with two test generators that obey the axioms and can be                 %%
%% are usually defined in the litterature                                   %%
%% when proving completeness in the context of ACCS. \rfig{test-generators} %%
%% \begin{example}                                                          %%
%% \testconv{\trace} = ...                                                               %%
%% a(s,O) = ...                                                             %%
%% \end{example}                                                            %%
%%                                                                          %%
%% \begin{lemma}                                                            %%
%% The two test generators obey to the axioms                               %%
%% \end{lemma}                                                              %%


%% This section is then parameterized by two test generators
%% $\testconv{\trace}$ and $a(O,\trace}$ such that $\mathcal{C} \Vdash \testconv{\trace}$
%% and $\mathcal{A} \Vdash a(s, O)$ hold.


\renewcommand{\States}{\ensuremath{A}}


First, converging along a finite trace~$\trace$~is logically
equivalent to passing the client~$\testconv{ \trace}$.  In other
words, there exists a bijection between the proofs (i.e. finite
derivation trees of~$\musti{ \server }{\testconv{ \trace}}$) and
the ones of~$ \server \cnvalong{ \trace }$. We first give the
proposition, and then discuss the auxiliary lemmas to prove it.

\begin{proposition}\coqCom{must_iff_acnv}%{myproposition}{mustiffacnv}
  \label{prop:must-iff-acnv}
  For every $\genlts_{\States} \in \obaFW$,
  $\server \in \States$, and
  $\trace \in \Actfin$ we have that $\musti{\server}{ \testconv{ \trace} }$
  if and only if~$\server \cnvalong \trace$.
\end{proposition}
\noindent
%\begin{proof}
The {\em if} implication is \rlem{acnv-must} and the {\em only if}
implication is \rlem{must-cnv}.
  %\end{proof}
\noindent
The hypothesis that $\genlts_{\States} \in \obaFW$,
\ie the use of forwarders, is necessary to show that convergence
implies passing a client, as shown by the next example.
\begin{example}
    \label{ex:forwarders-necessary}
    Consider a server~$\server$ in an LTS $\genlts \in \obaFB$
%    with, in the set $\StatesA$,
%    a state~$\server$, % that can input on~$\ab$ and then diverges.
    whose behaviour amounts to the following transitions:
    $\server \st{ \ab } \Omega \st{ \tau }  \Omega \st{ \tau } \ldots$
    Note that this entails that $\genlts$ does not
    {\em not} enjoy the axioms of forwarders.

    Now let $ \trace = \aa.\ab $. Since $\server \conv$ and $\server
    \Nwt{\aa}$ we know that $ \server \cnvalong \aa.\ab$.
    On the other hand \rtab{properties-functions-to-generate-clients}(\ref{gen-spec-mu-lts-co})
    implies that the client $\testconv{\trace}$ performs the transitions
    $\testconv{ \trace } \st{ \co{\aa}} \testconv{ \ab }  \st{ \co{\ab}} \testconv{ \varepsilon }$.
    Thanks to the \outputcommutativity axiom we obtain $\testconv{ \trace } \st{ \co{\ab}} \st{ \co{\aa}} \testconv{ \varepsilon }$.
    \rtab{properties-functions-to-generate-clients}(\ref{gen-spec-ungood}) implies that the states
    reached by the client are unsuccessful, and so by zipping the traces performed
    by $\server$ and by $\testconv{\trace}$
    we build a maximal computation of
    $\csys{\server}{\testconv{\trace}}$ that is unsuccessful,
    and thus $\Nmusti{\server}{\testconv{\trace}}$.\hfill$\qed$
  \end{example}
  \noindent
  This example explains why in spite of \rlem{musti-obafb-iff-musti-obafw}
  output-buffered agents with feedback do not suffice to use the
  standard characterisations of the \mustpreorder.


%%%%%%%%%%%%%%%%%%%%%%%%%%%%%%%%%%%%%%%%%%%%%%%%%%%%%%%%%%%%%%%%
%%%%%%%%%%%%%%%%%%%%%%%%%%%%%%%%%%%%%%%%%%%%%%%%%%%%%%%%%%%%%%%%
%%%%%%%%%%%%%%%%%%%%%%%%%%%%%%%%%%%%%%%%%%%%%%%%%%%%%%%%%%%%%%%%
%% ARGUMENTS ABOUT ACCEPTANCE SETS
%%%%%%%%%%%%%%%%%%%%%%%%%%%%%%%%%%%%%%%%%%%%%%%%%%%%%%%%%%%%%%%%
%%%%%%%%%%%%%%%%%%%%%%%%%%%%%%%%%%%%%%%%%%%%%%%%%%%%%%%%%%%%%%%%
%%%%%%%%%%%%%%%%%%%%%%%%%%%%%%%%%%%%%%%%%%%%%%%%%%%%%%%%%%%%%%%%

We move on to the more involved technical results, \ie the next
three lemmas, that we use to reason on acceptance sets of servers.
%  The proofs are deferred to \rapp{bar-induction} and \rapp{appendix-completeness},
We wish to stress \rlem{must-output-swap-l-fw}:
it states that, when reasoning on~$\opMusti$,
outputs can be freely moved from the client to the server side of
systems, if servers %are forwarders.
have the forwarding ability.
Its proof uses {\em all} the axioms for output-buffered agents with
feedback, and the lemma itself is used in the proof of the main
result on acceptance sets, namely \rlem{completeness-part-2.2-auxiliary}.

\begin{lemma}[ Output swap ]
  \label{lem:must-output-swap-l-fw}
  Let $\genlts_A \in \obaFW$ and
  $\genlts_B \in \obaFB$.
  $\Forevery \serverA_1, \serverA_2 \in \StatesA$,
  every $\client_1, \client_2 \in \StatesB$ and name $\aa \in \Names$ such that
  $\serverA_1 \st{\co{\aa}} \serverA_2$ and
  $\client_1 \st{\co{\aa}} \client_2$,
  if $\musti{\serverA_1}{\client_2}$ then $\musti{\serverA_2}{\client_1}$.
\end{lemma}

\begin{lemma}
  \label{lem:completeness-part-2.2-diff-outputs}
  Let $\genlts_A \in \obaFW$.
  For every $\server \in \States$, $\trace \in \Actfin$,
  and every $L, E \subseteq \Names$, if
  $\co{L} \in \accht{ \server }{ \trace }$
  then $\Nmusti{ \server }{ \testacc{\trace}{E \setminus L}}$.
\end{lemma}



\begin{lemma}\coqCom{must_gen_a_with_s}
  \label{lem:completeness-part-2.2-auxiliary}
  Let $\genlts_A \in \obaFW$.
  $\Forevery \server \in \States, \trace \in \Actfin$,
  and every finite set $\ohmy \subseteq \co{\Names}$,
  if $\server \cnvalong s$ then either
  \begin{enumerate}[(i)]
      \item
    %% \item\label{pt:completeness-crux-move-1} %%
    $\musti{\server}{\testacc{ \trace }{ \bigcup \co{ \accht{p}{s}
          \setminus \ohmy }}}$, or
  \item
   %% \item\label{pt:completeness-crux-move-2} %%
    there exists $\widehat{\ohmy} \in \accht{ \server }{ \trace }$ such that $\widehat{\ohmy} \subseteq \ohmy$.
  \end{enumerate}
\end{lemma}

\renewcommand{\traceA}{s_1}
\renewcommand{\traceB}{s_2}

%% \begin{lemma}%[\coqConv{acnv_weak_a_congr}]%{mylemma}{acnvsplits}                 %%
%%  \label{lem:acnv-split-s}                                                         %%
%%  For every $\traceA, \traceB \in \Actfin,$ and                                    %%
%%  $\serverA, \serverB  \in \States$,                                               %%
%%  if $\serverA \cnvalong \traceA.\traceB$ and $ \serverA \wta{ \traceA} \serverB$ %%
%% then $\serverB \cnvalong \traceB$.                                               %%
%% % $ (p \aftera \traceA)  \cnvalong \traceB$.                                     %%
%% \end{lemma}                                                                       %%

\renewcommand{\stateB}{q}
\renewcommand{\traceB}{\traceC}

\renewcommand{\traceA}{s_1}
\renewcommand{\traceB}{s_2}
\renewcommand{\traceC}{s_3}

%% Now we gather the properties of the clients generated by the function~$g$~and~$c$. %%
%% \begin{lemma}%[\coqCom{gen_test_reduces_if}]                                        %%
%%   \label{lem:gen-reduces-if}                                                        %%
%%   \label{lem:gs-reduces}                                                            %%
%%   For every $\server \in \Proc$ and                                                 %%
%%   $\trace \in \Actfin$, if $p \st{\tau}$ then $g(\trace, \server) \st{\tau}$.       %%
%% \end{lemma}                                                                         %%
%%%% ORIGINAL ARGUMENT
%%%% DO NOT EREASE
%% \begin{proof}
%% The proof is by induction on the sequence $s$.
%% In the base case $s = \varepsilon$. The test generated by $g$ is $p$,
%% which reduces by hypothesis, and so does $g(\varepsilon,p)$.
%% In the inductive case $s = \mu.s'$.
%% We proceed by case-analysis on $\mu$.
%% If $\mu$ is an output then $g(\mu.s', p) = \co{\mu}.(g(s', p)) \extc
%% \tau.\Unit$ which reduces to $\Unit$ using
%% the transition rule \extR.
%% If $\mu$ is an input then $g(\mu.s', p) = \co{\mu} \Par g(s', p)$
%% which reduces using the transition rule
%% \parR and the inductive hypothesis, which ensures that
%% $g(s', p)$ reduces.
%% \leaveout{%
%% and the hypothesis tells us that it reduces.
%% In the inductive case $s = \mu.s'$ and our inductive hypothesis tells
%% us that $g(s', p)$ reduces.
%% We proceed by case-analysis on $\mu$.
%% If $\mu$ is an input then $g(\mu.s', p) = \co{\mu} \Par g(s', p)$
%% which reduces using the transition rule
%% \parR and the inductive hypothesis.
%% If $\mu$ is an output then $g(\mu.s', p) = \co{\mu}.(g(s', p)) \extc
%% \tau.\Unit$ which reduces to $\Unit$ using
%% the transition rule \extR.}
%% \end{proof}
%%%% ORIGINAL ARGUMENT
%%%% DO NOT EREASE
%%%%%%%%%%%%%%%%%%%%%%%%%%%%%%%%%%%%%%%%%%%%%%%%%%%%%%%%%


\renewcommand{\traceA}{s_1}
\renewcommand{\traceB}{s_2}
\renewcommand{\traceC}{s_3}





%% \begin{lemma}%[\coqCom{must_gen_conv_wta_mu}]                                               %%
%%   \label{lem:must-weak-a-mu}                                                                %%
%%   For every $\mu \in \Act, \trace \in \Actfin$ and                                          %%
%%   $\server \in \Proc \wehavethat  \musti{\server}{\testconv{\mu.\trace)}$                           %%
%%   and $\server \wt{\mu} \server'$ imply that %$\musti{ (\server \aftera \mu) }{\testconv{\trace)}$. %%
%%   $\musti{ \server' }{\testconv{\trace)}$.                                                          %%
%% \end{lemma}                                                                                 %%
%%%%%%%%%%%%%%%%%%%%%%%%%%%%%%%%%%%%%%%%%%%%%%%%%
%%%%%%%%%%%%%%%%%%%%%%%%%%%%%%%%%%%%%%%%%%%%%%%%%
%%%%%%%% ORIGINAL ARGUMENT
%%%%%%%% DO NOT DELETE
%% \begin{proof}
%%   We have to show that $q \musti \testconv{\trace}$ for every $q$ such that $p \wta{\mu} q$.
%%   Pick such $q$, and decompose the weak transition $p \wta{\mu} q$ as
%%   follows,
%%   $$
%%   p \wta{\varepsilon} \widehat{p} \sta{\mu} \widehat{q} \wta{\varepsilon} q
%%   $$

%%   We proceed by induction on the derivation of $p \wta{\varepsilon} \widehat{p}$.

%%   In the base case $p = \widehat{p}$.
%%   Since $\testconv{ \mu.s ) \Nst{\ok}$ and  $\testconv{\mu.\trace} \st{\co{\mu}} \testconv{\trace}$\footnote{This fact follows directly from the definition of $\testconv{\trace}$, and formally we use \rlem{inversion-gen-mu}},
%%   \rlem{musti-presereved-by-actions-of-unsuccesful-tests}
%%   ensures that $\widehat{q} \musti \testconv{\trace}$.
%%   \rlem{must-lts-a-tau} together with an induction on
%%   the derivation of $\widehat{q} \wta{\varepsilon} q$
%%   ensure that $\tmusti{q}{\testconv{\trace}}$ as required.


%%   In the inductive case there exists $p'$ such that
%%   $p \sta{\tau} p' \wta{\varepsilon} \widehat{p}$ and that
%%   $$
%%   p \sta{\tau} p' \wta{\varepsilon} \widehat{p} \sta{\mu} \widehat{q} \wta{\varepsilon} q
%%   $$
%%   The inductive hypothesis esures that
%%   $\tmusti{p'}{\testconv{\mu.\trace}}$ implies $\tmusti{q}{\testconv{\trace}}$,
%%   and thus we obtain $\tmusti{q}{\testconv{\trace}}$ via an application of
%%   \rlem{must-lts-a-tau}.
%% \end{proof}
%%%%%%%% DO NOT DELETE
%%%%%%%%%%%%%%%%%%%%%%%%%%%%%%%%%%%%%%%%%%%%%%%%%
%%%%%%%%%%%%%%%%%%%%%%%%%%%%%%%%%%%%%%%%%%%%%%%%%





%% First we give sufficient conditions for a server                                                                         %%
%% not to satisfy a client generated by the function $\testacc{-}{-}$, then                                                    %%
%% we discuss \rlem{completeness-part-2.2-diff-outputs}. The other proofs                                                   %%
%% are in \rapp{appendix}.                                                                                                  %%
%%                                                                                                                          %%
%%                                                                                                                          %%
%% \begin{lemma}                                                                                                            %%
%%   \label{lem:gen-test-unhappy-if}                                                                                        %%
%%     \label{lem:gen-test-lts-co-mu}                                                                                       %%
%%   For every $\client \in \Proc$ and $\trace \in \Actfin$,                                                                %%
%%   \begin{enumerate}                                                                                                      %%
%%   \item %(\coqCom{gen_test_unhappy_if})                                                                                  %%
%%     \label{pt:gen-test-unhappy-if}   if $\lnot \good{\client}$ then                                                      %%
%%   $\lnot \good{g( \trace , \client )}$,                                                                                  %%
%%   \item %(\coqCom{gen_test_lts_mu})                                                                                      %%
%%     \label{pt:gen-test-lts-co-mu} for every $\mu \in \Act$,  $g(\mu.\trace, \client) \st{\co{\mu}} g(\trace , \client)$. %%
%%   \end{enumerate}                                                                                                        %%
%% \end{lemma}                                                                                                              %%


We can now show that the alternative preorder~$\asleq$
includes~$\testleqS$ when used over LTSs of forwarders.
\begin{lemma}%, \coqCom{completeness}]
  \label{lem:completeness}
  For every $\genlts_A, \genlts_B \in \obaFW$ and
  servers $\serverA \in \StatesA, \serverB \in \StatesB $,
  if ${ \serverA } \testleqS { \serverB }$
  then ${ \serverA } \asleq { \serverB }$.
\end{lemma}
\begin{proof}
  Let ${ \serverA } \testleqS { \serverB }$.
  To prove that ${ \serverA } \bhvleqone { \serverB }$,
  suppose
  ${\serverA} \cnvalong \trace$ for some trace $\trace$.
  \rprop{must-iff-acnv} implies $\musti{ {\serverA} }{\testconv{ \trace} }$,
  and so by hypothesis $ \musti{ { \serverB } }{\testconv{ \trace }}$.
  \rprop{must-iff-acnv} ensures that ${\serverB} \cnvalong \trace$.

  We now show that ${ \serverA } \bhvleqtwo { \serverB }$.
  Thanks to
  \rlem{conditions-on-accsets-logically-equivalent}, it is enough to prove
  that $\serverA \asleqAfw \serverB$. So, we show that
  for every trace $\trace \in \Actfin$, if ${ \serverA } \acnvalong \trace$
  then $\accht{{ \serverA }}{\trace} \ll
  \accht{{ \serverB }}{\trace} $.
  Fix an $O \in \accht{ {\serverB} }{ \trace }$.
  We have to exhibit a set $\widehat{O} \in \accht{{ \serverA
  }}{\trace}$ such that $\widehat{O} \subseteq O$.


  By definition $O \in \accht{ {\serverB} }{ \trace }$
  means that for some $q'$ we have ${\serverB} \wt{ \trace } q' \stable$
  and $O(\serverB') = O$.
  Let $E = \bigcup \accht{ { \serverA }}{\trace}$ and $X = E \setminus O $.
  The hypothesis that ${\serverA}~\cnvalong~\trace$,
  and the construction of the set~$X$
  let us apply \rlem{completeness-part-2.2-auxiliary}, which implies that
  either \begin{enumerate}[(a)] \item
  $\musti{{\serverA}}{\testacc{\trace}{ \co{ X }}}$, or
  \item there exists
  a $\widehat{O} \in \accht{{ \serverA }}{\trace}$ such that $\widehat{O}
  \subseteq O(\serverB')$.
  \end{enumerate}
  Since (b) is exactly what we are after, to conclude the
  argument it suffices to prove that (a) is false.
  This follows from \rlem{completeness-part-2.2-diff-outputs}, which proves
  $\Nmusti{ {\serverB}  }{ \testacc{ \trace }{ \co{ X }} }$,
  and the hypothesis ${ \serverA } \testleqS { \serverB }$,
  which ensures  $\Nmusti{ {\serverA}  }{ \testacc{ \trace }{ \co{ X }} }$.
\end{proof}

The fact that the \mustpreorder can be captured via the function $\liftFW{-}$
and $\asleq$ is a direct consequence of \rlem{musti-obafb-iff-musti-obafw}
and \rlem{completeness}.
\begin{proposition}[Completeness]%, \coqCom{completeness}]
  \label{prop:bhv-completeness}
  For every $\genlts_A, \genlts_B \in \obaFB$ and
  servers $\serverA \in \StatesA, \serverB \in \StatesB $,
  if $\serverA \testleqS \serverB$ then $\liftFW{ \serverA } \asleq \liftFW{ \serverB }$.
\end{proposition}
%%%%%
  %%%%%%%%%%%%%%%%%%%%%%%%% OLD PROOF FOR LICS. DO NOT EREASE BEFORE DEADLINE
  %%%%%%%%%%%%%%%%%%%%%%%%%
%%\begin{proof}
  %% We have to prove the following set inclusions
  %% \begin{align}
  %%   {\testleqS} & \subseteq {\bhvleqone} \tag{cmp-1 \coqCom{completeness1}}\\
  %%   {\testleqS} & \subseteq {\bhvleqtwo} \tag{cmp-2 \coqCom{completeness2}}
  %% \end{align}
  %% To prove the first inclusion, suppose
  %% $\serverA \testleqS \serverB$ and $\liftFW{\serverA} \cnvalong \trace$ for some trace $\trace$.
  %% \rprop{must-iff-acnv} implies
  %% $\musti{ \liftFW{\serverA} }{\testconv{ \trace}}$.
  %% \rlem{musti-obafb-iff-musti-obafw} implies that $ \musti{ \serverA }{\testconv{ \trace }} $,
  %% thus the hypothesis implies that $\musti{ \serverB }{\testconv{ \trace }}$.
  %% Applying again \rlem{musti-obafb-iff-musti-obafw} and
  %% \rprop{must-iff-acnv} we obtain $\liftFW{ \serverB } \cnvalong \trace$.
  %%
  %% To prove the second inclusion, thanks to
  %% \rlem{conditions-on-accsets-logically-equivalent} we have to explain
  %% why for every trace $\trace \in \Actfin$ if $\liftFW{ \serverA } \acnvalong \trace$
  %% then it is the case that $\accht{\liftFW{ \serverA }}{\trace} \ll
  %% \accht{\liftFW{ \serverB }}{\trace} $.
  %% Fix an $O \in \accht{ \liftFW{\serverB} }{ \trace }$.
  %% We have to exhibit a set $\widehat{O} \in \accht{\liftFW{ \serverA
  %% }}{\trace}$ such that $\widehat{O} \subseteq O$.
  %%
  %% By definition $O \in \accht{ \liftFW{\serverB} }{ \trace }$
  %% means that for some $q'$ we have $\liftFW{\serverB} \wt{ \trace } q' \stable$
  %% and $O(\serverB') = O$.
  %% Let
  %% %$ \mathcal{O}_{\serverA} =  \bigcup \setof{ O }{ O \in \accht{ \serverA }{ \trace } }$
  %% $E = \bigcup \accht{ \liftFW{ \serverA }}{\trace}$
  %% and $X = E \setminus O $.
  %% The hypothesis of convergence, \ie that $\liftFW{\serverA}~\cnvalong~\trace$,
  %% and the construction of the set~$X$
  %% let us apply \rlem{completeness-part-2.2-auxiliary} which implies that
  %% either \begin{enumerate}[(a)] \item
  %% $\musti{\liftFW{\serverA}}{\testacc{\trace}{ \co{ X }}}$, or
  %% \item there exists
  %% a $\widehat{O} \in \accht{\liftFW{ \serverA }}{\trace}$ such that $\widehat{O}
  %% \subseteq O(\serverB')$.
  %% \end{enumerate}
  %% Since (b) is exactly what we are after, to conclude the
  %% argument it suffices to prove that (a) is false. This is done in
  %% four steps:
  %% we apply \rlem{completeness-part-2.2-diff-outputs} to prove
  %% $\Nmusti{ \liftFW{\serverB}  }{ \testacc{ \trace }{ \co{ X }} }$,
  %% then we apply
  %% \rlem{musti-obafb-iff-musti-obafw} to obtain
  %% $\Nmusti{\serverB}{ \testacc{ \trace }{ \co{ X }} }$. This fact together
  %% with $\serverA \testleqS \serverB$ imply
  %% $\Nmusti{ \serverA }{ \testacc{ \trace }{ \co{ X }} }$.
  %% Applying again \rlem{musti-obafb-iff-musti-obafw} we conclude
  %% $\Nmusti{ \liftFW{\serverA} }{ \testacc{ \trace }{ \co{ X }} }$.
%%%\end{proof}



\label{completeness-part-2.2-auxiliary-proof}
\label{sec:appendix-completeness}

  We now gather all the technical auxiliary lemmas and then discuss the
  proofs of the main ones.

By assumption, outputs preserve the predicate~$\goodSym$.
For
stable clients, they also preserve the negation of this predicate.
\begin{lemma}%(\coqMT{stsad_wtout_sad})
    \label{lem:st-wtout-Nok}
  For all $\client, \client' \in \States$ and trace $\trace \in \co{\Names}^\star$,
  $\client \stable$, $\lnot \good{\client}$ and $\client \wt{ \trace }
  \client'$ implies $\lnot \good{\client'} $.
\end{lemma}


%% %% THE FOLLOWING LEMMA SEEMS USELESS WRT THE CURRENT EXPOSITION
%% %% \gb{Is the following lemma used anywhere in the explanation ?
%% %% For stable clients, they also preserve the negation of this predicate.

%% \begin{lemma}%(\coqMT{stsad_wtout_sad})

%%   For all $\client, \client' \in \States$ and trace $\trace \in \co{\Names}^\star$,
%%   $\client \stable$, $\lnot \good{\client}$ and $\client \wt{ \trace }
%%   \client'$ implies $\lnot \good{\client'} $.
%% \end{lemma}

%% %% }





\subsection{Testing convergence}
We start with preliminary facts, in particular two lemmas that follow
from the properties in \rtab{properties-functions-to-generate-clients}.

A process $\state$ {\em converges along} a trace $\trace$ if for every
$\state'$ reached by $\state$ performing any prefix of $\trace$, the
process $\state'$ converges.

\begin{lemma}%[\coqConv{cnv_iff_prefix_terminate}]%{mylemma}{cnvalongiffprefix}
  \label{lem:cnvalong-iff-prefix}\label{lem:acnvalong-iff-prefix}
  For every $\lts{\States}{L}{\st{}}$, $\state \in \States$, and $\trace \in \Actfin$,
  $ \state \cnvalong \trace$ if and only if
  $ \state \wt{\trace'} \stateA $ implies $\stateA \conv$
  for every $\trace'$ prefix of $\trace$.
\end{lemma}


Traces of output actions impact neither the stability of servers,
nor their input actions.
\begin{lemma}
  \label{lem:st-wtout-st}  \label{lem:st-wtout-inp}
  For every $\genlts_\StatesA$,
  every $\server, \server' \in \StatesA$ and every trace $\trace \in \co{\Names}^\star$,
  \begin{enumerate}
    \item %(\coqLTS{st_wtout_st})
      $\server \stable$ and $\server \wt{ \trace } \server'$ implies $\server' \stable$.
    \item %(\coqLTS{st_wtout_inp})
      $\server \stable$ and $\server \wt{ \trace } \server'$ implies $I(\server) = I(\server')$.
  \end{enumerate}
\end{lemma}



\begin{lemma}
  \label{lem:testconv-always-reduces}
  For every $\trace \in\Actfin$, $\testconv{ \trace} \st{\tau}$.
\end{lemma}


The \outputdeterminacyinv axiom is used in the proof of the next lemma.
\begin{lemma}\coqCom{inversion_gen_test_mu}
  \label{lem:inversion-gen-mu}
  For every $\trace \in \Actfin$, %and $\client \in \States$,
  if $\testconv{\trace} \st{ \mu } \client $
  then either
  \begin{enumerate}[(a)]
  \item\label{pt:inversion-gen-mu-left}
    $\good{ \client }$, or
  \item\label{pt:inversion-gen-mu-right}
    $s = \traceA. \co{\mu}. \traceB$ for some $\traceA \in \Names^{\star}$ and $\traceB \in \Actfin$
    such that $\client \simeq \testconv{\traceA.\traceB}$.
  \end{enumerate}
\end{lemma}


\begin{lemma}\coqCom{inversion_gen_conv_tau}
  \label{lem:inversion-feeder-tau}
  For every $\trace \in \Actfin$,
  %and $\client \in \States$,
  if $\testconv{\trace} \st{\tau} \client$
    then either:
  \begin{enumerate}[(a)]
  \item\label{inversion-feeder-tau-ok} $\good{ \client }$, or
  \item\label{inversion-feeder-tau-split}
    there exist $\ab$, $\traceA, \traceB$ and $\traceC$ with
    $\traceA.\ab.\traceB \in \Names^\star$ such that
    $s = \traceA.\ab.\traceB.\co{\ab}.\traceC$ and
    $\client \simeq \testconv{\traceA.\traceB.\traceC}$.
    \end{enumerate}
\end{lemma}


\renewcommand{\stateB}{q}
\renewcommand{\traceB}{\traceC}


\renewcommand{\traceA}{s_1}
\renewcommand{\traceB}{s_2}
\renewcommand{\traceC}{s_3}



\begin{lemma}
  \label{lem:must-gen-conv-wt-mu}
  Let $\genlts_A \in \obaFW$. For every server $\server, \server' \in \States$,
  trace $\trace \in \Actfin$ and
  action $\mu \in \Act$ such that $\server \wt{\mu} \server'$
  we have that
  $\musti{\server}{\testconv{\mu.\trace}} \implies \server' \musti{\server'}{\testconv{\trace}}$.
\end{lemma}
\begin{proof}
  By rule induction on the reduction $\server \wt{\mu} \server'$ together with
  \rlem{musti-preserved-by-left-tau} and \rlem{musti-presereved-by-actions-of-unsuccesful-tests}.
\end{proof}

\begin{lemma}
  \label{lem:must-cnv}
  Let $\genlts_A \in \obaFW$. For every server $\server \in \States$, trace $\trace \in \Actfin$
  we have that
  $\musti{\server}{\testconv{\trace}} \implies \server \cnvalong \trace$.
\end{lemma}
\begin{proof}
  We proceed by induction on the trace $\trace$.
  In the base case $\trace$ is $\varepsilon$.
  \rtab{properties-functions-to-generate-clients}(\ref{gen-spec-ungood}) states that
  $\lnot \good{\testconv{\varepsilon}}$ and we apply \rlem{must-terminate} to obtain $\server \convi$,
  and thus $\server \cnvalong \varepsilon$.
  In the inductive case $\trace$ is $\mu.\trace'$ for some $\mu \in \Act$ and $\trace' \in \Actfin$.
  We must show the following properties,
  \begin{enumerate}
  \item $\server \convi$, and
  \item for every $\server'$ such that $\server \wt{\mu} \server'$, $\server' \cnvalong \trace'$.
  \end{enumerate}
  We prove the first property as we did in the base case,
  and we apply \rlem{must-gen-conv-wt-mu} to prove the second property.
  %%%%% ORIGINAL ARGUMENT
  %%%%% DO NOT DELETE
  %% In the base case $\trace = \varepsilon$,
  %% and we must show $p \acnvalong \varepsilon$.
  %% Since $p \convi$ we use the rule \acnvepsilon.
  %% In the inductive case  $\trace= \mu.s'$ for some $\mu \in \Act$, and
  %% $s' \in \Actfin$. We must derive the judgement $p \acnvalong \mu.s'$,
  %% and since~$s$ is not empty we have to apply rule \acnvmu.
  %% As $p \convi$, it remains to explain why $( p \aftera \mu ) \acnvalong s' $.
  %% If $( p \aftera \mu )$ is empty this is trivially true, so suppose
  %% $( p \aftera \mu ) \neq \emptyset $.
  %% The hypothesis that $\musti \csys{p}{c(\mu.s')}$ and \rlem{must-weak-a-mu}
  %% imply that $\musti \csys{(p \aftera \mu)}{c(s')}$.
  %% We obtain $(p \aftera \mu) \acnvalong s' $ from the the inductive hypothesis,
  %% which ensures that for every
  %% $p \in \Proc \wehavethat \musti \csys{p}{c(s')} \implies p \acnvalong s'.$
  %%%%% DO NOT DELETE
  %%%%% ORIGINAL ARGUMENT
\end{proof}

\begin{lemma}%[\coqCom{acnv_drop_in_the_middle}, \coqCom{acnv_annhil}]
  \label{lem:acnvalong-preserved-by-operations}
  \label{lem:acnvalong-preserved-by-annihilation}
  \label{lem:acnv-drop-in-the-middle}
  Let $\genlts_A \in \obaFW$. For every $\server \in \States$, $\traceA \in \Names^\star$ and $\traceC \in \Actfin$ we have that
  \begin{enumerate}
  \item\label{pt:acnvalong-preserved-by-drop-in-the-middle}
        for every $\mu \in \Act$,
        if $\state \acnvalong \traceA.\mu.\traceC$ and $\state \st{\mu} \stateB$ then $\stateB \acnvalong \traceA.\traceC$,
  \item\label{pt:acnvalong-preserved-by-annihilation}
  for every $\aa.\traceB \in \Names^\star$ if
  $\state \acnvalong \traceA.\aa.\traceB.\co{\aa}.\traceC$ then
  $\state \acnvalong \traceA.\traceB.\traceC$.
  \end{enumerate}
\end{lemma}
%% \begin{proof} %%
  %% This is a consequence of \rlem{acnv-unroll} and of \inputreceptivity. %%
%% \end{proof} %%
%%%%%%%%%%%%%%%%%%%%%%%%%%%%%%%%%%%%%%%%%%%
%%%%%%%%%%%%%%%%%%%%%%%%%%%%%%%%%%%%%%%%%%%
%%% ORIGINAL ARGUMENT
%%% DO NOT EREASE
%% \begin{proof}
%%   \rPt{acnvalong-preserved-by-drop-in-the-middle}
%%   %% The hypothesis $ \state \acnvalong \traceA.\aa.\traceB $ together with \rlem{acnv-split-s} ensure
%%   %% that
%%   %% $$
%%   %% (\state \aftera \traceA.\aa) \acnvalong \traceB
%%   %% $$
%%   %% The hypotheses $\traceA.\aa \in \Names^\star$ let us derive
%%   %% the following transitions,
%%   %% $$
%%   %% \state \wta{\traceA.\aa} \state \Par \mailbox{\Pi\co{\traceA.\aa}} %\wt{\aa} \stateB \Par \mailbox{\Pi\co{\traceA}}
%%   %% $$
%%   %% which witness that $ \state \Par \mailbox{\Pi \co{\traceA.\aa}} \in (\state \aftera \traceA.\aa) $. %(SEE \rlem{trivial-yet-ubiquitous})
%%   %% It follows that $ ( \state \Par \mailbox{\Pi \co{\traceA.\aa}} ) \acnvalong \traceB $, and thus
%%   %% $\traceA.\aa \in \Names^\star$ together with \rlem{acnv-unroll} imply  that
%%   %% $( \state \Par \mailbox{\co{\aa}} ) \acnvalong \traceA.\traceB$. \rlem{acnvalong-Par} ensures that $\state \acnvalong \traceA.\traceB$.
%%   %%
%%   \renewcommand{\traceB}{\traceC}
%%   The hypothesis $ \state \acnvalong \traceA.\mu.\traceB $ and \rlem{acnv-split-s} ensure
%%   that
%%   $$
%%   ( \state \aftera \traceA.\mu) \acnvalong \traceB
%%   $$
%%   The hypotheses $\traceA \in \Names^\star$ and $ \state \wt{\mu} \stateB$ let us derive
%%   the following transitions,
%%   $$
%%   \state \wta{\traceA} \state \Par \mailbox{\Pi\co{\traceA}} \wt{\mu} \stateB \Par \mailbox{\Pi\co{\traceA}}
%%   $$
%%   which witness that $  \stateB \Par \mailbox{\Pi \co{\traceA}} \in (\state \aftera \traceA.\mu) $.
%%   It follows that $ ( \stateB \Par \mailbox{\Pi \co{\traceA}} ) \acnvalong \traceB $, and thus
%%   $\traceA \in \Names^\star$ together with \rlem{acnv-unroll} imply  $\stateB \acnvalong \traceA.\traceB$.%

%%   \rPt{acnvalong-preserved-by-annihilation}.
%%   \renewcommand{\traceB}{s_2}
%%   The hypothesis  $p \acnvalong \traceA.\aa.\traceB.\co{\aa}.\traceC$ and \rlem{acnv-split-s} imply that
%%   $$
%%   (p \aftera \traceA.\aa.\traceB.\co{\aa}) \acnvalong \traceC
%%   $$
%%   The hypothesis $\traceA.\aa.\traceB \in \Names^\star$ ensure the following transitions exist,
%%   $$
%%   p \wta{\traceA.\aa.\traceB} p \Par \mailbox{\Pi\co{\traceA}.\co{\aa}.\co{\traceB}} \sta{\co{\aa}} p \Par \mailbox{\Pi\co{\traceA}.\co{\traceB}}
%%   $$
%%   and so $ p \Par \mailbox{\Pi\co{\traceA}.\co{\traceB}} \in (p \aftera \traceA.\aa.\traceB.\co{\aa})$.
%%   Now $p \acnvalong \traceA.\traceB.\traceC$ follows from \rlem{acnv-unroll} and the hypothesis $\traceA.\traceB \in \Names^\star$.%
%% \end{proof}
%%% DO NOT EREASE
%%%%%%%%%%%%%%%%%%%%%%%%%%%%%%%%%%%%%%%%%
%%%%%%%%%%%%%%%%%%%%%%%%%%%%%%%%%%%%%%%%%



\begin{lemma}\coqCom{terminate_must_gen_conv_nil}
  \label{lem:terminate-must-gen-conv-nil}
  For every LTS $\genlts_{\States}$ and
  every $\server \in \States$,
  $\server \convi$ implies $\musti{\server}{\testconv{\varepsilon}}$.
\end{lemma}
%%%%%%%%%%%%%%%%%%%%%%%%%%%%%%%%%%%%%%%%%%%%%%%%%%%%%%%%%%%%%%%
%%%%%%%%%%%%%%%%%%%%%%%%%%%%%%%%%%%%%%%%%%%%%%%%%%%%%%%%%%%%%%%
%%%%%%%%%%%%%%%%% ORIGINAL ARGUMENT
%%%%%%%%%%%%%%%%% DO NOT EREASE
\begin{proof}
Rule induction on the derivation of $p \convi$.
%% By construction we know that $\lnot \testconv{ \trace} \good$.
%% which implies that the only way to prove
%% $ \musti{ \server  }{\testconv{ \trace}}$ is
%% by applying the rule \mstep.
%% We have therefore to show that $\csys{ \server }{ \testconv{ \trace}} \st{\tau}$,
%% which is a consequence of \rlem{gen-reduces-if} together with the rule~\stau,
%% and that
%% $$
%% \forevery q,r \in \Proc
%% \wehavethat
%% \csys{p}{\testconv{ \trace}} \st{\tau} \csys{ q }{ r } \implies \musti{ q }{ r }
%% $$
%% To prove this fact,
%% fix a reduction
%% \begin{equation}
%% \label{eq:acnv-must-reduction-nil}
%% \csys{p}{\testconv{ \trace}} \st{\tau} \csys{ q }{ r }
%% \end{equation}
%
%% By definition $\testconv{\varepsilon) = \tau.\Unit$,
%% and so the test cannot interact with $ \server $:
%% the reduction in \req{acnv-must-reduction-nil}
%% must be due to a reduction of
%% either~$ \server$~or~$\tau.\Unit$.
%
%% We now reason by rule induction on the derivation of $p \convi$.
%% In the base case $p \convi$ because~$p \Nst{\tau}$,
%% and hence $p = q$ and $r = \Unit$. Since $r \good$
%% we conclude by applying \mnow. %the axiom \mnow,
%% %% $$
%% %% \begin{prooftree}
%% %%   \Unit \good
%% %%   \justifies
%% %%   \musti{ q }{ \Unit }
%% %% \end{prooftree}
%% %% $$
%% In the inductive case $p \convi$ because $p' \convi$ for every process $p'$ such that $p \st{\tau} p'$. The inductive hypothesis ensures that
%% $$
%%   \Forevery  \widehat{p} \in \Proc \wehavethat
%%   p \st{\tau} \widehat{p}
%%   \implies  \musti{\widehat{p}}{ \testconv{\varepsilon) }
%% $$
%% If the reduction in \req{acnv-must-reduction} is due to $\tau.\Unit$,
%% we apply \mnow as in the base case; if it is due to a
%% reduction in $p$, then $p \st{\tau} q$ and $r = \testconv{\varepsilon)$,
%% so we have to prove that $\musti{q}{\testconv{\varepsilon)}$.
%% This follows from the inductive hypothesis.
\end{proof}
%%%%%%%%%%%%%%%%% ORIGINAL ARGUMENT
%%%%%%%%%%%%%%%%% DO NOT EREASE
%%%%%%%%%%%%%%%%%%%%%%%%%%%%%%%%%%%%%%%%%%%%%%%%%%%%%%%%%%%%%%%
%%%%%%%%%%%%%%%%%%%%%%%%%%%%%%%%%%%%%%%%%%%%%%%%%%%%%%%%%%%%%%%


\begin{lemma}\coqCom{acnv_must}
  \label{lem:acnv-must}
  For every $\genlts_A \in \obaFW$, every $\server \in \StatesA$, and $\trace \in\Actfin$,
  if $\server \cnvalong \trace$ then $\musti{\server}{\testconv{ \trace}}$.
\end{lemma}

\begin{proof}
The hypothesis $\server \cnvalong \trace$ ensures $p \convi$.
%\rtab{properties-functions-to-generate-clients}(\ref{gen-spec-ungood}) states that $\lnot \good{ \testconv{ \trace} }$.
%The last fact means that the only way to prove
%$\musti{ \server  }{\testconv{ \trace}}$ is by applying the rule \mstep.
%We have therefore to show that $\csys{ \server }{ \testconv{ \trace}} \st{\tau}$,
%which is a consequence of \rlem{testconv-always-reduces}.
We show that $\musti{\server}{\testconv{ \trace}}$ reasoning by complete
induction on the length of the trace~$\trace$.
The base case is \rlem{terminate-must-gen-conv-nil} % together with the fact that $\convi p$.
and here we discuss the inductive case, i.e. when $\len{ \trace } = n + 1$ for some $n \in \N$.

We proceed by rule induction on $p \convi$.
In the base case $p \Nst{\tau}$,
and the reduction at hand % in \req{acnv-must-reduction}
is due to either a $\tau$ transition in~$\testconv{ \trace}$, or
a communication between~$p$ and~$\testconv{ \trace}$.

In the first case~$\testconv{ \trace} \st{\tau} r$, and so \rlem{inversion-feeder-tau} ensures
that one of the following conditions holds,
\begin{enumerate}
\item $\good{ \client }$, or
\item there exist $\aa \in \Names$, $\traceA, \traceB$ and $\traceC$ with
  $\trace = \traceA.\aa.\traceB.\co{\aa}.\traceC$ and
  $r \simeq \testconv{\traceA.\traceB.\traceC}$.
\end{enumerate}

If $\good{\client}$ then we conclude via rule \mnow; %
otherwise \rptlem{acnvalong-preserved-by-operations}{acnvalong-preserved-by-annihilation}
and the hypothesis that $\server \cnvalong \trace$ imply $\server \cnvalong \traceA.\traceB.\traceC$,
thus prove $\musti{\serverA}{\client}$ %$\musti{q}}{r}$
via the inductive hypothesis of the complete induction on $\trace$.%\req{ind-hyp-length}.

We now consider the case when the transition is due to a communication,
\ie $\serverA \st{\mu} \serverA'$ and $\testconv{ \trace} \st{\co{\mu}} \client$.
\rlem{inversion-gen-mu}
tells us that either $\good{\client}$
or there exist $\traceA$ and $\traceB$ such that
$\trace = \traceA.\mu.\traceB$ and $\client \simeq \testconv{\traceA.\traceB}$.
In the first case we conclude via rule \mnow.
In the second case we apply
\rptlem{acnvalong-preserved-by-operations}{acnvalong-preserved-by-drop-in-the-middle}
to prove $\server' \cnvalong \traceA.\traceB$, and thus~$\musti{\server'}{\client}$ follows
from the inductive hypothesis of the complete induction.
In the inductive case of the rule induction on~$\server \convi$, we know that
$\server \st{\tau} \server'$ for some process $\server'$.
We reason again by case analysis on how the reduction
we fixed %in \req{acnv-must-reduction}
has been derived, \ie either via a~$\tau$ transition in~$\testconv{ \trace}$,
or via a communication between~$p$ and~$\testconv{ \trace}$, or via
a~$\tau$ transition in~$p$.
In the first two cases we reason as we did
for the base case of the rule induction.
In the third case~$\server \cnvalong{s}$ and
$\server \st{\tau} \server'$ imply $\server' \cnvalong{s}$, we
thus obtain $\musti{\server'}{\testconv{ \trace}}$ thanks to
the inductive hypothesis of the rule induction
which we can apply because the
tree to derive~$\server' \convi$ is smaller than the tree
to derive that~$\server \convi$.
\end{proof}



\iffalse{
\TODO{This lemma uses syntax. Where is this lemma used ?}
\begin{lemma}
  \label{lem:aftera-musti-implies-musti}
  $\Forevery \server, \client \in \States$ the following facts
   hold: %it is true that
  \begin{enumerate}[(a)]
  \item\label{pt:aftera-musti-implies-musti-input} %(\coqMT{after_a_must_input})
    for every $\aa \in \Names$ and every
    $p' \suchthat \server \wta{\aa}  \server' \wehavethat \musti{\server'}{ \client }$ implies
    $\musti{ \server }{(\client \Par \mailbox{\co{\aa}})}$,
  \item\label{pt:aftera-musti-implies-musti-output} %(\coqMT{after_a_must_output})
    for every $\co{\aa} \in \co{\Names}$ and every $\server' \suchthat  \server \wta{\co{\aa}}  \server' \wehavethat
    \musti{ \server' }{r}$ and $p \convi$ imply $\musti{  \server }{\aa.r \extc \tau.\Unit}$.
  \end{enumerate}
\end{lemma}
%%%%%%%%%%%%%%%%%%%%%%%%%%%%%%%%%%%%%%%%%%%%%%%%%%%%%%%%%%%%%%%%%%%%
%%%%%%%%%%%%%%%%%%%%%%%%%%%%%%%%%%%%%%%%%%%%%%%%%%%%%%%%%%%%%%%%%%%%
%%%%%%%%%%%%%%%% DO NOT EREASE
%% \begin{proof}
%%   \rPt{aftera-musti-implies-musti-input} follows from
%%   \rdefpt{sta}{sta-asynch-input} and \rlem{must-i-output-swap}.
%%   %% To show \rpt{aftera-musti-implies-musti-input} suppose  $\aa \in \Names$.
%%   %% \rdefptNOPAR{sta}{sta-asynch-input} ensures that $p \wt{\aa} p \Par \mailbox{\co{\aa}}$,
%%   %% thus $ p \Par \mailbox{\co{\aa}} \in (p \aftera \aa) $. The hypothesis implies that
%%   %% $ p \Par \mailbox{\co{\aa}} \musti r$ and thus \rlem{must-i-output-swap} implies
%%   %% $ p \musti \mailbox{\co{\aa}} \Par r$.


%%   To show \rpt{aftera-musti-implies-musti-output},
%%   %  since~$\aa.r + \tau.\Unit \Nst{\ok}$,
%%   we apply \mstep. Rule \stau\ ensures the silent transition
%%   $\aa.\client \extc \tau.\Unit \st{\tau} \Unit$ which ensures
%%   the analogous transition of the system $\csys{ p }{ \aa.r \extc
%%     \tau.\Unit }$, and thus we have to prove only
%%   that $\forevery p', r' \in \States \wehavethat \csys{ p }{ \aa.r \extc \tau.\Unit } \st{\tau} \csys{p'}{r'}$ implies $\musti{\server'}{\client'}$.

%%   We suppose $\csys{p}{\aa.\client \extc \tau.\Unit} \st{\tau} \csys{p'}{r'}$ and
%%   proceed by induction on the derivation of $p \convi$, which is true by hypothesis.
%%   In the base case $p$ is stable and the silent move is either due to
%%   a $\tau$-transition performed by the client  or a communication between the server and the client.
%%   In the first case $\aa.r \extc \tau.\Unit \st{\tau} r'$ implies $r' = \Unit$ and
%%   rule \mnow ensures the required $p \musti r'$.
%%   In the second case the only action that can be involved in a
%%   communication between the server and the client is $\aa$,
%%   so we have $p \st{\co{\aa}} p'$ and $\aa.r \extc \tau.\Unit \st{\aa} r'$.
%%   Note that $\aa.r \extc \tau.\Unit \st{\aa} r'$ implies $r' = r$.
%%   By definition $p' \in (p \aftera \co{aa})$ and the hypothesis
%%   $\musti{ (p \aftera \co{\aa})  }{ \client' }$ ensures $\musti
%%   \csys{ \server' }{ \client'}$.

%%   In the inductive case the system transition may be due to a $\tau$-transition performed by the server or the client,
%%   or a communication between the two.
%%   Both second and third cases are treated as in the base case and we only focus on the first one.
%%   We suppose $p \st{\tau} p'$ for some $p'$.
%%   The inductive hypothesis ensures $p' \musti r$ if we show $ \musti
%%   \csys{ (p' \aftera \co{\aa}) }{ r }$.
%%       This follows from the inclusion $(p' \aftera \co{\aa}) \subseteq (p \aftera \co{\aa})$,
%%       which is true because $ p \st{\tau} p'$ implies the inclusion $ (p' \aftera \co{\aa}) \subseteq (p \aftera \co{\aa})$.
%% \end{proof}
%%%%%%%%%%%%%%%% DO NOT EREASE
%%%%%%%%%%%%%%%%%%%%%%%%%%%%%%%%%%%%%%%%%%%%%%%%%%%%%%%%%%%%%%%%%%%%
%%%%%%%%%%%%%%%%%%%%%%%%%%%%%%%%%%%%%%%%%%%%%%%%%%%%%%%%%%%%%%%%%%%%






\begin{proof}[Proof of \rprop{must-iff-acnv}]
  The \emph{if} part is \rlem{acnv-must}.
%  The {\em if} part is \rlem{must-acnv}.
  To prove the only if part,% (\coqCom{must_acnv}),
  note that by construction $\lnot \good{ \testconv{ \trace } }$ for every~$\trace$, thus
  %\rlem{gen-test-unhappy-if} and
  \rlem{must-terminate} imply that $p \convi$.
  The argument is by induction on~$\trace$. In the inductive case, we apply
  \rlem{must-weak-a-mu}\TODO{This lemma does not exist}
  to be able to use the inductive hypothesis.
  %%%%% ORIGINAL ARGUMENT. IT REALIES ON AFTER. WHERE IS AFERT DEFINED ?
  %%%%% DO NOT DELETE
  %% In the base case $\trace = \varepsilon$,
  %% and we must show $p \acnvalong \varepsilon$.
  %% Since $p \convi$ we use the rule \acnvepsilon.
  %% In the inductive case  $\trace= \mu.s'$ for some $\mu \in \Act$, and
  %% $s' \in \Actfin$. We must derive the judgement $p \acnvalong \mu.s'$,
  %% and since~$s$ is not empty we have to apply rule \acnvmu.
  %% As $p \convi$, it remains to explain why $( p \aftera \mu ) \acnvalong s' $.
  %% If $( p \aftera \mu )$ is empty this is trivially true, so suppose
  %% $( p \aftera \mu ) \neq \emptyset $.
  %% The hypothesis that $\musti{p}{c(\mu.s')}$ and \rlem{must-weak-a-mu}
  %% imply that $\musti{(p \aftera \mu)}{c(s')}$.
  %% We obtain $(p \aftera \mu) \acnvalong s' $ from the the inductive hypothesis,
  %% which ensures that for every
  %% $p \in \States \wehavethat \musti{p}{c(s')} \implies p \acnvalong s'.$
  %%%%% DO NOT DELETE
  %%%%% ORIGINAL ARGUMENT
\end{proof}
\fi
%\noindent
%This is sufficient to prove half of the completeness result.




%%%%%%%%%%%%%%%%%%%%%%%%%%%%%%%%%%%%%%%%%%%%%%%%%%%%%%%%%%%%%%%%
%%%%%%%%%%%%%%%%%%%%%%%%%%%%%%%%%%%%%%%%%%%%%%%%%%%%%%%%%%%%%%%%
%%%%%%%%%%%%%%%%%%%%%%%%%%%%%%%%%%%%%%%%%%%%%%%%%%%%%%%%%%%%%%%%
%% END ARGUMENT CONVERGENCE
%%%%%%%%%%%%%%%%%%%%%%%%%%%%%%%%%%%%%%%%%%%%%%%%%%%%%%%%%%%%%%%%
%%%%%%%%%%%%%%%%%%%%%%%%%%%%%%%%%%%%%%%%%%%%%%%%%%%%%%%%%%%%%%%%
%%%%%%%%%%%%%%%%%%%%%%%%%%%%%%%%%%%%%%%%%%%%%%%%%%%%%%%%%%%%%%%%





\subsection{Testing acceptance sets}

\renewcommand{\state}{p}
\renewcommand{\stateA}{p'}

\iffalse
\begin{lemma}
\label{lem:wta-preserves-da}
For every $\mu.\trace \in \Actfin$ and $\state \in \States$,
\begin{enumerate}
\item %(\coqCom{wta_stable_set_mu})
  if $ \state \wta{ \mu } \stateA $ then $\accht{ \stateA }{\trace} \subseteq \accht{ \state }{\mu.\trace}$,
%% THE NEXT ONE IS A TRIVIALITY ONCE THE MONOTONE OF ofun IS ESTABLISHED
%\item \gb{(COQ LINK)} $\ofun{ \accht{ \stateA }{ \trace } } \subseteq \ofun{\accht{ \state }{ \mu.\trace }}$,
%\item \gb{(COQ LINK)} $\accht{ \stateA }{ \trace }  \subseteq \accht{ \state }{ \mu.\trace }$,
\item\label{pt:da-output-shift} %(\coqCom{wta_stable_set_mu_input}) if $\mu \in \Names$, then
  $\accht{\state}{\mu.\trace} \subseteq \accht{ \state \Par \mailbox{ \co{\mu} } }{\trace} $.
\end{enumerate}
\end{lemma}
\fi


In this section we present the properties of the function $\testacc{-}{-}$
that are sufficient to obtain completeness.
To begin with, $\testacc{-}{-}$ function enjoys a form of monotonicity with respect to its second argument.

%% \begin{lemma}%(\coqCom{must_gen_a_nil_subseteq})
%% \label{lem:tacc-monotonicity}
%% For every $\server \in \ACCS$,
%% $\trace \in \Actfin$, and
%% $X \subseteq Y \subseteq \Names $,
%%         if $\musti{\server}{\testacc{ \trace }{X}}$ then $\musti{\server}{\testacc{ \trace }{Y}}$.
%% \end{lemma}
%% \begin{proof}
%%         Induction on the derivation of $\musti{ \server }{\testacc{ \trace }{X}}$.
%% \end{proof}
%% %%%
%% %%%
%% %%% PAUL: you wrote again a lemma that was already in the appendix. I copied it above.
%%%
%%%
\begin{lemma}
  \label{lem:tacc-monotonicity}
  \label{lem:must-f-gen-a-subseteq}
  %%%  Let $\genlts_A = \lts{\StatesA}{L}{\st{}_A} \in \oba$. THIS IS THE WEAKEST HYPOTHESIS
  Let $\genlts_A \in \obaFB$.
  $\Forevery \serverA \in \StatesA$, trace $\trace \in \Actfin$, and sets of outputs $O_1, O_2$,
  if $\musti{\serverA}{\testacc{\trace}{O_1}}$
  and $O_1 \subseteq O_2$ then
  $\musti{\serverA}{\testacc{\trace}{O_2}}$.
\end{lemma}
\begin{proof}
  Induction on the derivation of $\musti{ \server }{\testacc{ \trace }{ O_1 }}$.
\end{proof}

Let $\oba$ denote the set of LTS of output-buffered agents. Note that any $\genlts \in \oba$ need not enjoy the \outputfeedback axiom.






\begin{lemma}{\label{lem:aft-output-must-gen-acc}}
  Let $\genlts_A \in \oba$, and
  $\genlts_B \in \oba$.
  For every $\serverA \in \StatesA$, trace $\trace \in \Actfin$,
  set of outputs $O$ and name $\aa \in \Names$,
  such that
  \begin{enumerate}[(i)]
  \item
    \label{aft-output-must-gen-acc-h-1}
    $\serverA \convi$ and,
  \item
    \label{aft-output-must-gen-acc-h-2}
    $\Forevery \server' \in \StatesA$, $\server \wt{\co{\aa}} \serverA'$ implies
    $\musti{\server'}{\testacc{\trace}{ \co{O} }}$,
  \end{enumerate}
  we have that $\musti{\server}{\testacc{\co{\aa}.\trace}{ \co{O} }}$.
\end{lemma}
\begin{proof}
  We proceed by induction on the hypothesis $\server \convi$.

  \paragraph{(Base case: $\serverA$ is stable)}
  We prove $\musti{\server}{\testacc{\co{\aa}.\trace}{ \co{O} }}$ by applying rule \mstep.
  Since \rtab{properties-functions-to-generate-clients}(\ref{gen-spec-mu-out-ex-tau}) implies that
  $\csys{ \serverA }{\testacc{\co{\aa}.\trace}{ \co{ O } }} \st{\tau}$, all we need to prove is
  the following fact,
  \begin{equation}
    \tag{$\star$}
    \label{eq:aft-output-must-gen-acc-AIM}
    \forall \server' \in \StatesA, \client \in \StatesB \wehavethat
    \text{ if }
    \csys{ \server }{\testacc{\co{\aa}.\trace}{ \co{ O } }}
    \st{\tau} \csys{\server'}{\client} \text{ then } \musti{\server'}{\client}.
  \end{equation}
  %% \begin{enumerate}[(a)]
  %% \item $\csys{ \serverA }{\testacc{\co{\aa}.\trace}{ \co{ O } }} \st{\tau}$, and
  %% \item\label{pt:aft-output-must-gen-acc-AIM-2} for each $\csys{\serverA'}{\client'}$ such that
  %%   $\csys{ \serverA }{\testacc{\co{\aa}.\trace}{ \co{ O } }}
  %%   \st{\tau} \csys{\serverA'}{\client}$, we have $\musti{\serverA'}{\client}$.
  %% \end{enumerate}
  %% The first requirement follows from
  %%  that gives us a $\client$ such that $\testacc{\co{\aa}.\trace}{O} \st{\tau} \client$
%  and thus the following transition.
%  $$
%  \csys{ p }{\testacc{\co{\aa}.\trace}{O}} \st{\tau} \csys{ p }{\client}
%  $$
  %  We are now going to show the following requirement.
  %
  %To prove \rpt{aft-output-must-gen-acc-AIM-2}
  Fix a transition $\csys{ \serverA }{\testacc{\varepsilon}{\co{O}}} \st{\tau} \csys{\serverA'}{\client}$.
  As $\serverA$ is stable, this transition can either be due to:
  \begin{enumerate}
  \item a $\tau$-transition performed by the client such that
    $\testacc{\co{\aa}.\trace}{ \co{O} } \st{\tau} \client$, or
  \item an interaction between the server $\serverA$ and the client
    $\testacc{\co{\aa}.\trace}{ \co{O} }$.
  \end{enumerate}
  In the first case \rtab{properties-functions-to-generate-clients}(\ref{gen-spec-out-good}) implies $\good{\client}$, and hence we obtain $ \musti{\server'}{\client}$ via rule \mnow.
  In the second case there exists an action $\mu$ such that
  $$
  \serverA \st{\mu} \serverA'
  \quad\text{and}\quad
  \testacc{\co{\aa}.\trace}{\co{O}} \st{\co{\mu}} \client
  $$
  \rtab{properties-functions-to-generate-clients}(\ref{gen-spec-out-mu-inp}) implies $\mu$ is $\co{\aa}$
  and $\client = \testacc{\trace}{ \co{O} }$.
  We then have $\serverA \st{\co{\aa}} \serverA'$ and thus the reduction
  $\serverA \wt{\co{\aa}} \serverA'$,
  which allows us to apply the hypothesis (\ref{aft-output-must-gen-acc-h-2}) and obtain
  $\musti{\serverA'}{\client}$ as required.


  \paragraph{(Inductive case: $\server \st{\tau} \server'$ implies $\server'$)}
  The argument is similar to one for the base case, except that
  we must also tackle the case when the transition
  $\csys{ \server }{\testacc{ \co{\aa}.\trace }{ \co{O} }}
  \st{\tau} \csys{\server'}{\client}$ is due to a $\tau$ action performed
  by $\server$, \ie $\server \st{\tau} \server'$ and $\client = \testacc{ \co{\aa}.\trace }{ \co{O} }$.
  The inductive hypothesis tells us the following fact:
  \begin{center}
    For every $\serverA_1$ and $\aa$,
    such that
    $\serverA \st{\tau} \serverA_1$, for every $\serverA_2$,
   if $\serverA_1 \wt{\co{\aa}} \serverA_2$
    then $\musti{\serverA_2}{\testacc{\trace}{O}}$.
  \end{center}
  To apply the inductive hypothesis we have to show that
  for every $\serverA_2$ such that $\serverA' \wt{\co{\aa}} \serverA_2$
  we have that $\musti{\serverA_2}{\testacc{\trace}{ \co{ O }}}$.
  This is a consequence of the hypothesis (\ref{aft-output-must-gen-acc-h-2})
  together with the reduction $\serverA \st{\tau} \serverA' \wt{\co{\aa}} \serverA_2$,
  and thus concludes the proof.
\end{proof}




%% \renewcommand{\trace}{s}
%% \renewcommand{\traceA}{s'}
%% \renewcommand{\state}{p}
%% \renewcommand{\stateA}{p'}

%% \gb{THIS LEMMA WAS ALREADY IN THE TEX, AND WE HAVE ALREADY DISCUSSED THE PROOF.
%% \begin{lemma}
%%   \label{lem:completeness-part-2.2-must-gen-a-stable}
%%   %(\coqCom{must_gen_a_stable})
%%   For every $\server \in \States$ and
%%   every finite set of outputs $O$,
%%   if $\server \stable$ then
%%   either $O( \server ) \subseteq O$ or
%%   $\musti{\server}{\testacc{\varepsilon}{\co{O(p) \setminus O}}}$.
%% \end{lemma}
%% %%%%%%%%%%%%%%%%%%%%%%%%%%%%%%%%%%%%%%%%%%%%%%%%%%%%%%%%%%%%%%%%%%%%
%% %%%%%%%%%%%%%%%%%%%%%%%%%%%%%%%%%%%%%%%%%%%%%%%%%%%%%%%%%%%%%%%%%%%%
%% %%%%%%%%%%%%%%%% FULL ARGUMENT
%% %%%%%%%%%%%%%%%% DO NOT EREASE
%% \begin{proof}
%% %% \gb{
%% If $O(p) \subseteq O$ we are done, so suppose that there exists an output $\co{\aa} \in O(p)$
%% such that $\co{\aa} \notin O$, and let $ \widehat{O} = O(p) \setminus O $.
%% We prove $\musti{ \server}{ \testacc{\varepsilon}{\co{\widehat{O}}} }$
%% via rule \mstep.
%% %, which requires us to show the following facts,
%% %% \begin{enumerate}
%% %% \item $\csys{  \server  }{ \testacc{\varepsilon}{\co{\widehat{O}}} } \st{\tau}$, and
%% %% \item for each $\csys{ \server' }{ \client }$
%% %%       such that $\csys{ \server }{ \testacc{\varepsilon}{\co{\widehat{O}}} } \st{\tau} \csys{ \server' }{ \client }$, we have
%% %%   $\musti{ \server' }{ \client }$.
%% %% \end{enumerate}
%% %% The first requirement follows from the existence of the output $\co{\aa}$
%% %% in both the sets $O(p)$ and $\widehat{O}$. This let us derive a silent move
%% To do so pick a silent move
%% $ \csys{ p }{ \testacc{\varepsilon}{\co{\widehat{O}}}}
%%   \st{\tau}
%%   \csys{ p' }{ \client } $.\footnote{One such transition can be inferred via the rule using rule \scom\ and the assumptions.}
%% %%  The argument proceeds by case analysis on the rule used to infer the move, and we omit the details for it does not pose any complication.
%% We prove the second requirement by a case analysis on
%% the rule used to infer the silent move $\csys{ p }{
%%   \testacc{\varepsilon}{\co{\widehat{O}}} } \st{\tau} \csys{p'}{r'}$.
%% As $p \stable$ and $\testacc{\varepsilon}{\co{\widehat{O}}} \stable$, this
%% $\tau$-move can have been derived
%% only via rule \scom, and so its premises are true:
%% $p \st{\mu} p'$ and $\testacc{\varepsilon}{\co{\widehat{O}}} \st{\co{\mu}}$.
%% As $\co{\widehat{O}}$ contains uniquely input actions,
%% the client $\testacc{\varepsilon}{\co{\widehat{O}}}$ perfmors only inputs, and so
%% $O(\testacc{\varepsilon}{\co{\widehat{O}}})$ is empty. It must therefore be
%% the case that $\mu \in \co{\Names}$.
%% Together with the observation\footnote{Why is this observation true? Which is the lemma on generators that we are using ?}
%% that $\testacc{\varepsilon}{\co{\widehat{O}}} \st{\co{\mu}} q$ implies $q \st{\ok}$
%% when $\mu \in \co{\Names}$, it follows that each $\csys{p'}{r'}$ reached after performing one interaction
%% goes through an happy state, as shown below,
%% $$
%% \begin{prooftree}
%%   \begin{prooftree}
%%       r \st{\ok}
%%       \justifies
%%       p \musti r
%%       \using
%%       \mnow
%%   \end{prooftree}
%%   \justifies
%%   p \musti \testacc{\varepsilon}{\co{\widehat{O}}}
%%   \using
%%   \mstep
%% \end{prooftree}
%% $$
%% %As a consequence, $p \musti \testacc{\varepsilon}{\co{\widehat{O}}}$ and we are done.%
%% \end{proof}
%% %%%%%%%%%%%%%%%% DO NOT EREASE
%% %%%%%%%%%%%%%%%%%%%%%%%%%%%%%%%%%%%%%%%%%%%%%%%%%%%%%%%%%%%%%%%%%%%%
%% %%%%%%%%%%%%%%%%%%%%%%%%%%%%%%%%%%%%%%%%%%%%%%%%%%%%%%%%%%%%%%%%%%%%
%% }





\begin{lemma}{\label{lem:must-gen-acc-stable}}
  %%%  Let $\genlts_{\StatesA}, \genlts_{\StatesA} \in \oba$.
  Let $\genlts_{\StatesA} \in \obaFB$.
  $\Forevery \serverA \in \StatesA$ and set of outputs $O$,
  if $\serverA$ is stable then either
  \begin{enumerate}[(a)]
  \item $\musti{\serverA}{\testacc{\varepsilon}{\co{ \outof{\serverA} \setminus O}} }$, or
  \item $\outof{\serverA} \subseteq O$.
  \end{enumerate}
\end{lemma}
\begin{proof}
We distinguish whether $\outof{\serverA} \setminus O$ is empty or not.
In the first case, $\outof{\serverA} \setminus O = \emptyset$ implies
$\outof{\serverA} \subseteq O$, and we are done.

In the second case, there exists $\co{\aa} \in \outof{\serverA}$ such that
$\co{\aa} \notin O$.
Note also that
\rtab{properties-functions-to-generate-clients}(\ref{gen-spec-ungood})
ensures that  $\lnot \good{\testacc{\varepsilon}{\co{ \outof{\serverA} \setminus O}}}$,
and thus we construct a derivation of
$\musti{ \serverA }{\testacc{\varepsilon}{\co{ \outof{\serverA} \setminus O}}}$
by applying the rule \mstep. This requires us to show the following facts,
\begin{enumerate}
%\item\label{must-gen-acc-stable-2-1} $\lnot \good{\testacc{\varepsilon}{\co{ \outof{\serverA} \setminus O}}}$,
\item\label{must-gen-acc-stable-2-2} $\csys{\serverA}{\testacc{\trace}{\co{ \outof{\serverA} \setminus O}}} \st{}$, and
\item\label{must-gen-acc-stable-2-3} for each $\serverA'$, $r$ such that
  $\csys{\serverA}{\testacc{\trace}{\co{ \outof{\serverA} \setminus O}} } \st{\tau} \csys{\serverA'}{r}$,
  $\musti{\serverA'}{r}$ holds.
\end{enumerate}


To prove (\ref{must-gen-acc-stable-2-2}), we show that an interaction between
the server $\serverA$ and the test $\testacc{\trace}{\co{ \outof{\serverA} \setminus O }}$ exists.
As $\co{\aa} \in \outof{\serverA}$, we have that $\serverA \st{\co{\aa}}$.
Then $\co{\aa} \in \outof{\serverA} \setminus O$
together with (\ref{gen-spec-acc-nil-mem-lts-inp}) ensure
that $\testacc{\trace}{\co{ \outof{\serverA} \setminus O }} \st{\aa}$.
An application of the rule \rname{s-com} gives us the required transition
$\csys{\serverA}{\testacc{\trace}{\co{ \outof{\serverA} \setminus O}}} \st{}$.


To show (\ref{must-gen-acc-stable-2-3}), fix a silent transition
$\csys{\server}{\testacc{\trace}{\co{ \outof{\server} \setminus O}}} \st{\tau} \csys{\server'}{r}$.
We proceed by case analysis on the rule used to derive the
transition under scrutiny.
Recall that the server $\server$ is stable by hypothesis, and that
$\testacc{\trace}{\co{ \outof{\serverA} \setminus O }}$ is stable
thanks to
\rtab{properties-functions-to-generate-clients}(\ref{gen-spec-acc-nil-stable-tau}).
This means that the silent transition must have been derived via rule \scom.
Furthermore, \rtab{properties-functions-to-generate-clients}(\ref{gen-spec-acc-nil-stable-out}) implies that the test
$\testacc{\trace}{\co{ \outof{\serverA} \setminus O }}$ does not perform any output.
As a consequence, if there is an interaction it must be because the test
performs an input and becomes $r$.
\rtab{properties-functions-to-generate-clients}(\ref{gen-spec-acc-nil-lts-inp-good})
implies that $\good{r}$, and hence we obtain the required
$\musti{\serverA'}{r}$ applying rule \mnow.
\end{proof}
%% Lemma must_gen_acc_stable                                                                        %%
%%   `{@LtsObaFW A L IL LA LOA V, @LtsObaFB B L IL LB LOB W, !Good B L good, !gen_spec_acc gen_acc} %%
%%   (p : A) (O : gset L) :                                                                         %%
%%   p ↛ -> must p (gen_acc (lts_outputs p ∖ O) []) \/ lts_outputs p ⊆ O.                           %%



\noindent%
\textbf{\rlem{completeness-part-2.2-diff-outputs}}
Let $\genlts_A \in \obaFW$.
  For every $\server \in \States$, $\trace \in \Actfin$,
  and every $L, E \subseteq \Names$, if
  $\co{L} \in \accht{ \server }{ \trace }$
  then $\Nmusti{ \server }{ \testacc{\trace}{E \setminus L}}$.
\begin{proof}%[Proof of \rlem{completeness-part-2.2-diff-outputs}]
  By hypothesis there exists a set $\co{ L } \in \accht{ \server }{ \trace }$,
  \ie for some $\server'$ we have $\server \wt{ \trace } \server' \stable$ and $O(p') = \co{L}$.
  We have to show that $\Nmusti{ \server }{\testacc{ \trace }{E \setminus L}}$, \ie
  $\musti{ \server }{\testacc{ \trace }{E \setminus L}}$ implies $\bot$.
  For convenience, let $X = E \setminus L$.

  %% there exists no derivation of                                                                %%
  %% judgement $\musti{\server}{\testacc{ \trace }{E \setminus L}}$                                  %%
  %% \footnote{Formally, we prove that $\musti{ \server }{\testacc{ \trace }{L}}$ implies $\bot$.}.  %%
  %% \rlem{gen-test-unhappy-if} implies $\lnot \good{\testacc{ \trace }{L \setminus O}}$,            %%
  %% thus no tree that ends with \mnow can have $\musti{\server}{\testacc{ \trace }{L \setminus O}}$ %%
  %% as conclusion. We have to explain why no tree that ends with \mstep                          %%
  %% has $ \musti{ \server }{\testacc{ \trace }{L \setminus O}}$ as conclusion.                      %%


  We proceed by induction on the derivation of the weak transitions $\server \wt{ \trace } \server'$.
  In the base case the derivation consists in an application of rule \wtrefl,
  which implies that $\server = \server'$ and $\trace = \varepsilon$.
  We show that there exists no derivation of judgement $\musti{\server}{\testacc{ \trace }{X}}$.
  By definition, $\lnot \good{\testacc{ \trace }{ X }}$ and
  thus no tree that ends with \mnow can have $\musti{\server}{\testacc{ \trace }{ X }}$
  as conclusion.
  The hypotheses ensure that $\server$ is stable,
  and $\testacc{ \varepsilon }{ X }$ is stable by definition.
  The set of inputs of $\testacc{\varepsilon}{ X }$ is $X$,
  which prevents an interaction between $\server$ and
  $\testacc{ \trace }{ X }$, i.e. an application of rule \scom.
  This proves that $\csys{ \server }{\testacc{ \trace }{ X }}$ is stable,
  thus a side condition of \mstep is false, and the rule cannot
  be employed to prove $\musti{ \server }{\testacc{ \trace }{ X }}$.

  \renewcommand{\traceB}{t}

  In the inductive cases $p \wt{ \trace } p'$ is derived using either:
  \begin{enumerate}[(i)]
  \item \label{completeness-part-2.2-diff-outputs-ind-case-1}
    rule \wttau such that $p \st{\tau} \widehat{ \server } \wt{ \trace } \server'$, or
  \item \label{completeness-part-2.2-diff-outputs-ind-case-2}
    rule \wtmu such that $p \st{\mu} \widehat{ \server } \wt{\traceB} \server'$, with $\trace = \mu.\traceB$.
  \end{enumerate}

  %In case \ref{completeness-part-2.2-diff-outputs-ind-case-1},
  In the first case, applying the inductive hypothesis requires us to show
  $\musti{\widehat{ \server } }{ \testacc{ \trace }{ X }}$, which is true since
  $\musti{p}{\testacc{ \trace }{ X }}$ is preserved by the $\tau$-transitions
  performed by the server.
  %% , thus since                                                                                  %%
  %% $ \csys{ \server }{ \testacc{ \trace }{L}} \st{\tau} \csys{\widehat{ \server }}{\testacc{ \trace }{L}}$ %%
  %% the hypothesis of \mstep is false, and thus the rule cannot be applied.                       %%


  %In case \ref{completeness-part-2.2-diff-outputs-ind-case-2} note that $ \traceA = \trace.\traceC = \mu.\traceB.\traceC$.
  In the second case, applying the inductive hypothesis
  requires us to show $\musti{\widehat{ \server } }{ \testacc{ \traceB }{ X }}$.
  \rtab{properties-functions-to-generate-clients}(\ref{gen-spec-mu-lts-co}) implies that $\testacc{\mu.\traceB}{ X } \st{\co{\mu}} \testacc{\traceB}{ X }$.
  Then we derive via \scom~the transition
  $\csys{ \server }{ \testacc{\mu.\traceB}{ E }} \st{ \tau } \csys{
    \widehat{\server} }{ \testacc{\traceB}{ E } }$.
  Since $\musti{ \server }{\testacc{ \trace }{ X }}$ is preserved by the interactions
  occurring between the server and the client, which implies
  $\musti{\widehat{ \server }}{\testacc{ \traceB }{ X }}$ as required.
\end{proof}


\noindent%
\textbf{\rlem{completeness-part-2.2-auxiliary}} \coqCom{must_gen_a_with_s}
 Let $\genlts_A \in \obaFW$.
  $\Forevery \server \in \States, \trace \in \Actfin$,
  and every finite set $\ohmy \subseteq \co{\Names}$,
  if $\server \cnvalong s$ then either
  \begin{enumerate}[(i)]
      \item
    %% \item\label{pt:completeness-crux-move-1} %%
    $\musti{\server}{\testacc{ \trace }{ \bigcup \co{ \accht{p}{s}
          \setminus \ohmy }}}$, or
  \item
   %% \item\label{pt:completeness-crux-move-2} %%
    there exists $\widehat{\ohmy} \in \accht{ \server }{ \trace }$ such that $\widehat{\ohmy} \subseteq \ohmy$.
  \end{enumerate}
\begin{proof}
  We proceed by induction on the trace $\trace$.


  \paragraph{(Base case, $\trace = \varepsilon$)}
  The hypothesis $p \acnvalong{\varepsilon}$ implies $p \convi$ and we continue by induction
  on the derivation of $p \convi$.\footnote{Recall that the definition of~$\convi$ is in \req{def-intentional-predicates}}
  In the base case $p \convi$ was proven using rule \mnow, and hence $\server  \stable$.
  We apply \rlem{must-gen-acc-stable}
  to obtain either:
  \begin{enumerate}[(i)]
  \item\label{completeness-part-2.2-auxiliary-1-1-1}
    $\musti{\serverA}{\testacc{\varepsilon}{ \co{ \outof{\serverA} \setminus O}} }$, or
  \item\label{completeness-part-2.2-auxiliary-1-1-2}
    $\outof{\serverA} \subseteq O$.
  \end{enumerate}
  %% sans hasard pas de chance
  In case (\ref{completeness-part-2.2-auxiliary-1-1-1}) we are done.
  In case (\ref{completeness-part-2.2-auxiliary-1-1-2}), as $\serverA$ is stable we have
  $\setof{\serverA'}{\serverA \wt{\varepsilon} \serverA' \stable} = \set{\serverA}$ and thus
  $\accht{\serverA}{\varepsilon} = \set{O(\serverA)}$ and we conclude by letting
  $\widehat{O} = O(\serverA)$.


  In the inductive case~$\server \convi$ was proven using rule \mstep.
    We know that $ \server \st{ \tau }$,
    %\pl{it is not true that the inductive case implies the existence
    %of a transition \server \st{\tau}} and hence $ \server \st{ \tau
    %} $,
    %% YES, IT IS TRUE. HAVE YOU READ [IND-RULE] AT ALL ?
    and the inductive hypothesis states that for any $\server'$ such that
    $\serverA \st{\tau} \server'$, either:
    \begin{enumerate}[(a)]
    \item\label{completeness-part-2.2-auxiliary-1-2-1-appendix}
      $\musti{\server'}{\testacc{\varepsilon}{\co{ \outof{\server'} \setminus O}}}$, or
    \item\label{completeness-part-2.2-auxiliary-1-2-2-appendix}
      there exists $\widehat{\ohmy} \in \accht{ \server' }{ \trace }$
      such that $\widehat{\ohmy} \subseteq \ohmy$.
    \end{enumerate}
%  In the second case $\serverA$ is not stable.\TODO{This is not possible in the base case of the induction on $\server \convi$.}
  %% From the inductive hypothesis of the induction on $p \convi$
  %% we know that for any $\serverA'$ such that
  %% $\serverA \st{\tau} \serverA'$, either:
  %% \begin{enumerate}[(i)]
  %% \item\label{completeness-part-2.2-auxiliary-1-2-1}
  %%   $\musti{\serverA'}{\testacc{\varepsilon}{\co{ \outof{\serverA} \setminus O}}}$, or
  %% \item\label{completeness-part-2.2-auxiliary-1-2-2}
  %%   there exists $\widehat{\ohmy} \in \accht{ \serverA' }{ \trace }$
  %%   such that $\widehat{\ohmy} \subseteq \ohmy$.
  %% \end{enumerate}
%%  By induction on \gb{the cardinality of} the set $\setof{\serverA'}{\serverA \st{\tau} \serverA'}$
%%  and the application of the inductive hypothesis we know that either:
    It follows that either
    \begin{description}
%  \begin{enumerate}[(i)]
    \item[($\forall$)]
      \label{completeness-part-2.2-auxiliary-1-2-1}
      for each $\serverA' \in \setof{\serverA'}{\serverA \st{\tau} \serverA'}$,
      $\musti{\serverA'}{\testacc{\varepsilon}{\bigcup \co{\accht{ \server' }{ \trace } \setminus O}}}$, or
  \item[($\exists$)]
    \label{completeness-part-2.2-auxiliary-1-2-2}
    there exists a $\server' \in \setof{\server'}{\serverA \st{\tau} \server'}$
    and a $\widehat{\ohmy} \in \accht{ \server' }{ \varepsilon }$
    such that $\widehat{\ohmy} \subseteq \ohmy$,
    \end{description}
%  \end{enumerate}
    \noindent
    We discuss the two cases. If ($\exists$) the argument is straightforward:
    we pick the existing $\server'$ such that $\server \st{\tau}
    \server'$. The definition of $\accht{-}{-}$ ensures that
    and show that $\accht{\server'}{\varepsilon} \subseteq
    \accht{\serverA}{\varepsilon}$, and thus we conclude by choosing $\widehat{O}$.

  Case ($\forall$) requires more work.
  We are going to show that $\musti{\serverA}{\testacc{\varepsilon}{\bigcup \co{\accht{ \server }{ \trace } \setminus O}}}$ holds.
  To do so we apply the rule \mstep and we need to show the following facts,
  \begin{enumerate}[(a)]
  \item $\csys{ p }{\testacc{\varepsilon}{\bigcup \co{\accht{ \server }{ \trace } \setminus O}}} \st{\tau}$, and
  \item for each $\csys{p'}{r'}$ such that
    $\csys{ p }{\testacc{\varepsilon}{\bigcup \co{\accht{ \server }{ \trace } \setminus O}}}
    \st{\tau} \csys{p'}{r'}$, we have $\musti{\server'}{r'}$.
  \end{enumerate}

  The first requirement follows from the fact that $\server$ is not
  stable. %
%  \TODO{this is why we need the case analysis on p being stable or not, so in fact we need both induction and case analysis}
  %% CORRECT, YET VERBOSE DETAILS
  %% and thus the existence of a $\server'$ such that $\server \st{\tau} \server'$,
  %% which implies the transition
  %% $\csys{ p }{\testacc{\varepsilon}{\bigcup \co{\accht{ \server }{ \trace } \setminus O}}}
  %% \st{\tau} \csys{ p' }{\testacc{\varepsilon}{\bigcup \co{\accht{ \server }{ \trace } \setminus O}}}$.
  To show the second requirement we proceed by case analysis on the transition
  $\csys{ p }{\testacc{\varepsilon}{\bigcup \co{\accht{ \server }{ \trace } \setminus O}}} \st{\tau} \csys{p'}{r'}$.
  As $\testacc{\varepsilon}{\co{ \outof{\serverA} \setminus O }}$ is stable by
  $(\ref{gen-spec-acc-nil-stable-tau})$, it can either be due to:
  \begin{enumerate}
  \item a $\tau$-transition performed by the server $\serverA$ such that
    $\serverA \st{\tau} \serverA'$, or
  \item an interaction between the server $\serverA$ and the client
    $\testacc{\varepsilon}{\bigcup \co{\accht{ \server }{ \trace } \setminus O}}$.
  \end{enumerate}

  In the first case we apply the first part of the inductive hypothesis %(\ref{completeness-part-2.2-auxiliary-1-2-1})\TODO{fix reference}
  to prove that 
  $\musti{\serverA'}{\testacc{\varepsilon}{\bigcup \co{\accht{ \serverA' }{ \trace } \setminus O}}}$,
  and we conclude via \rlem{must-f-gen-a-subseteq} to get the required
  $$
  \musti{\serverA'}{\testacc{\varepsilon}{\bigcup \co{\accht{ \server }{ \trace } \setminus O}}}.
  $$
%  \TODO{Mistake: the set $\co{ \outof{\serverA'} \setminus O}$ is not the correct one after the application of \rlem{must-f-gen-a-subseteq}.}
  In the second case, there exists a $\mu \in \Act$ such that
  $$
  \serverA \st{\mu} \serverA' \text{ and }
  \testacc{\varepsilon}{\bigcup \co{\accht{ \server }{ \trace } \setminus O}} \st{\co{\mu}} r
  $$

  %% From (\ref{gen-spec-acc-nil-stable-out}) we know that $\co{\mu}$ is necessary an output. %%
  Thanks to \rtab{properties-functions-to-generate-clients}(\ref{gen-spec-acc-nil-lts-inp-good}) we apply
  rule \mnow\ to prove that $\musti{\serverA'}{ \client }$ and we are done
  with the base case of the main induction on the trace~$\trace$.




  \paragraph{(Inductive case, $\trace =  \mu.s'$)}
  By induction on the set $\setof{\serverA'}{\serverA \wt{\mu} \serverA'}$
  and an application of the inductive hypothesis we know that either:
  \begin{enumerate}[(i)]
  \item\label{completeness-part-2.2-auxiliary-2-2-2}
    there exists $\serverA' \in \setof{\serverA'}{\serverA \wt{\mu} \serverA'}$
    and $\widehat{\ohmy} \in \accht{\serverA'}{\trace'}$
    such that $\widehat{\ohmy} \subseteq \ohmy$, or
  \item\label{completeness-part-2.2-auxiliary-2-2-1}
    for each $\serverA' \in \setof{\serverA'}{\serverA \wt{\mu} \serverA'}$ we have that
    $\musti{\serverA'}{\testacc{\trace'}{\bigcup \co{\accht{\serverA'}{\trace'}}}}$.
  \end{enumerate}

  In the first case, the inclusion $\accht{\serverA'}{\trace'} \subseteq \accht{\serverA}{\mu.\trace'}$
  and $\widehat{\ohmy} \in \accht{\serverA'}{\trace'}$ imply
  $\widehat{\ohmy} \in \accht{\serverA}{\trace}$ and we are done.

  In the second case, we show
  $\musti{\serverA}{\testacc{\trace}{\bigcup \co{\accht{\serverA}{\trace}}}}$
  by case analysis on the action~$\mu$, which can be either an input or an output.
  \begin{itemize}
  \item If $\mu$ is an input, $\mu = \aa$ for some $\aa \in \Names$.
    An application of the axiom of forwarders gives us a $\serverA'$ such that
    $\server \st{\aa} \server' \st{\co{\aa}} \server$.
    An application of \rtab{properties-functions-to-generate-clients}(\ref{gen-spec-mu-lts-co})
    gives us the following transition,
    $$
    \testacc{\aa.s'}{\bigcup \co{ \accht{\serverA}{\aa.\trace'} \setminus O}}
    \st{\co{\aa}}
    \testacc{s'}{\bigcup \co{ \accht{\serverA}{\aa.\trace'} \setminus O}}
    $$
    By an application of \rlem{must-output-swap-l-fw} it is enough to show
    $$
    \musti{\serverA'}{\testacc{s'}{\bigcup \co{\accht{\serverA}{\aa.\trace'} \setminus O}}}
    $$
    to obtain the required
    $\musti{\serverA}{\testacc{\aa.s'}{\bigcup
        \co{\accht{\serverA}{\aa.\trace'} \setminus O}}}$.

    %%%% THIS PART SEEMS USELESS NOW
    %% We apply (\ref{completeness-part-2.2-auxiliary-2-2-1}) and obtain
    %% $\musti{\serverA'}{\testacc{\trace'}{\bigcup \co{\accht{\serverA'}{\trace'} \setminus O}}}$.
    %% From the inclusion $\accht{\serverA'}{\trace'} \subseteq \accht{\serverA}{\aa.\trace'}$
    %% and an application of \rlem{must-f-gen-a-subseteq} we finally have that
    %% $$
    %% \musti{\serverA'}{\testacc{\trace'}{\bigcup \co{\accht{\serverA}{\aa.\trace'} \setminus O}}}
    %% $$.


  \item If $\mu$ is an output, $\mu = \co{\aa}$ for some $\aa \in \Names$ and
    we must show that
    $$
    \musti{\serverA}{\testacc{\co{\aa}.\trace'}{\bigcup \co{\accht{\serverA}{\co{\aa}.\trace'} \setminus O}}}.
    $$
    We apply \rlem{aft-output-must-gen-acc} together with (\ref{completeness-part-2.2-auxiliary-2-2-1}) to
    obtain
    $
    \musti{\serverA}{\testacc{\co{\aa}.\trace'}{\bigcup \co{\accht{\serverA'}{\trace'} \setminus O}}}
    $.
    Again, \rlem{must-f-gen-a-subseteq}
    together with the inclusion $\accht{\serverA'}{\trace'} \subseteq \accht{\serverA}{\co{\aa}.\trace'}$
    ensures the required
    $\musti{\serverA}{\testacc{\trace}{\bigcup \co{\accht{\serverA}{\trace} \setminus O}}}$.
  \end{itemize}
\end{proof}
