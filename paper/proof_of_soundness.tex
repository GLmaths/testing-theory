\section{Soundness}
\label{sec:proof-soundness}
\label{sec:bhv-soundness}\label{sec:soundness-bhv}


\renewcommand{\mustset}[2]{#1 \mathrel{\opMustset} #2}

\newcommand{\msetnow}{\rname{Mset-now}}
\newcommand{\msetstep}{\rname{Mset-step}}

\newcommand{\cnvleqset}{\mathrel{\preccurlyeq^{\mathsf{set}}_{\mathsf{cnv}}}}
\newcommand{\accleqset}{\mathrel{\preccurlyeq^{\mathsf{set}}_{\mathsf{acc}}}}
\newcommand{\Naccleqset}{\mathrel{{\not\preccurlyeq}^{\mathsf{set}}_{\mathsf{acc}}}}
\newcommand{\asleqset}{\mathrel{\preccurlyeq^{\mathsf{set}}_{\mathsf{AS}}}}
\newcommand{\Nasleqset}{\mathrel{\not\preccurlyeq^{\mathsf{set}}_{\mathsf{AS}}}}




\begin{figure}
 \hrulefill
  $$
 \begin{array}{ll}
   \msetnow
   &
   \begin{prooftree}
     \good{\client}
     \justifies
     \mustset{ X }{r}
   \end{prooftree}
   \\[30pt]
   \msetstep
   &
   \begin{prooftree}
     \begin{array}{lr}
       \lnot \good{\client} & \forall X' \wehavethat X \st{ \tau } X' %\text{ and }  X'
       \implies \mustset{X'}{\client}\\
       \forall\ \serverA \in X \wehavethat \csys{ \serverA }{ \client } \st{\tau} & \forall \client' \wehavethat \client \st{ \tau } \client' \implies \mustset{X}{\client'}
       \\
       \multicolumn{2}{r}{        \forall X', \mu \in \Actfin  %% p \stable
         \wehavethat X \wt{\co{\mu}} X' %\text{ and }  X'
         \text{ and }  \client \st{\mu} \client' \imply %% \der{p}{\co{\aa}}
         \mustset{ X' }{ \client'}}
     \end{array}
     \justifies
     \mustset{ X }{ \client }
   \end{prooftree}
 \end{array}
 $$
 \caption{Rules to define inductively the predicate $\opMustset$.}%    (\coqSou{must__set}).}
 \label{fig:rules-mustset}
 \hrulefill
\end{figure}



%% \begin{figure}
%%  \hrulefill
%%   $$
%%   \begin{array}{l}%*{c@{\hskip 2em}l}
%%     \msetnow
%%     \\[1pt]
%%     \begin{prooftree}
%%       \good{\client}
%%       \justifies
%%       \mustset{ X }{r}
%%     \end{prooftree}
%%     \\[10pt]
%%     \msetstep
%%     \\[1pt]
%%     \begin{prooftree}
%%       \begin{array}{lr}
%%         \lnot \good{\client} & \forall X' \wehavethat X \st{ \tau } X' %\text{ and }  X'
%%         \implies \mustset{X'}{\client}\\
%%       \forall\ \serverA \in X \wehavethat \csys{ \serverA }{ \client } \st{\tau} & \forall \client' \wehavethat \client \st{ \tau } \client' \implies \mustset{X}{\client'}
%%       \\
%%       \multicolumn{2}{r}{        \forall X', \mu \in \Actfin  %% p \stable
%%         \wehavethat X \wt{\co{\mu}} X' %\text{ and }  X'
%%         \text{ and }  \client \st{\mu} \client' \imply %% \der{p}{\co{\aa}}
%%         \mustset{ X' }{ \client'}}
%%       \end{array}
%%       \justifies
%%       \mustset{ X }{ \client }
%%     \end{prooftree}
%%   \end{array}
%%   $$
%%   \caption{Rules to define inductively the predicate $\opMustset$.}%    (\coqSou{must__set}).}
%%   \label{fig:rules-mustset}
%% \hrulefill
%% \end{figure}



In this section we prove the converse of \rprop{bhv-completeness},
i.e. that~$\asleq$ is included in~$\testleqS$. 
We remark immediately that a naïve reasoning does not work.
Fix two servers~$\serverA$ and~$\serverB$ such that $\serverA \asleq \serverB$.
We need to prove that for every client~$\client$, if
$\musti{\server}{\client}$ then $\musti{\serverA}{\client}$.
The reasonable first proof attempt consisting in proceeding by induction on
$\musti{\server}{\client}$ fails, as demonstrated by the following example.

\begin{example}
  \label{ex:must-set-is-helpful}
%  Recall the state $\server$ of \rexa{set-transitions}
  Consider the two servers $\server = \tau.(\co{\aa} \Par \co{b}) \extc
  \tau.(\co{\aa} \Par \co{c})$ and $\serverB = \co{\aa} \Par (\tau.\co{b} \extc
  \tau.\co{c})$ of \req{mailbox-hoisting}.
  Fix a client $\client$ such that $\musti{\serverA}{\client}$.
  Rule induction %on this fact
  yields the following inductive hypothesis:
  %on $\musti{\serverA}{\client}   tells us the
  \begin{center}
    $\forall \serverA', \serverB' \suchthat\;
    \csys{\serverA}{\client} \st{\tau} \csys{\serverA'}{\client'} \;\land\;$
    $\serverA' \asleq \serverB' \;\Rightarrow\; \musti{\serverB'}{\client'}$.
  \end{center}
  %% Showing that $\serverA \asleq \serverB$ implies $\serverA \testleqS \serverB$,
  %% \ie the soundness direction,
  %would among us to pick a client $\client$ such that $\musti{\serverA}{\client}$
  %and prove $\musti{\serverB}{\client}$.
  In the proof of
  $\musti{\serverB}{\client}$ we have to consider the case where there
  is a communication between $\serverB$ and
  $\client$ such that, for instance, $\serverB \st{\co{\aa}}
  \tau.\co{b} \extc \tau.\co{c}$ and $\client \st{\aa}
  \client'$.  In that case, we need to show that $ \musti{ \tau.\co{b}
    \extc \tau.\co{c}
  }{\client'}$. Ideally, we would like to use the inductive
  hypothesis. This requires us to exhibit a $\server'$ such that $
  \csys{\server}{\client} \st{\tau} \csys{\server'}{\client'}$ and $
  \server' \bhvleqtwo \tau.\co{b} \extc \tau.\co{c}$.
%\ilacom{What is $\accleqset$? I see now it is defined below.}
  However, note that there is no way to derive
  $\csys{\server}{\client} \st{\tau} \csys{\server'}{\client'}
  $, because $\server
  \Nst{\co{\aa}}$.  The inductive hypothesis thus cannot be applied,
  and the naïve proof does not go through.\hfill$\qed$
  %%% RATIONALE
  %% Note that, as~$\opMusti$ is defined on strong transition relations,
  %% the inductive hypothesis works \textit{one transition at a time}.
\end{example}
\noindent
This example suggests that defining an auxiliary predicate~$\opMustset$ in some sense
equivalent to~$\opMusti$, but that uses explicitly {\em weak} outputs
of servers, should be enough to prove that~$\asleq$ is sound with respect to~$\testleqS$.
Unfortunately, though, there is an additional nuisance to tackle: server
nondeterminism.
\begin{example}
  %  Following the intuition outlined thus far,
  %% Observe, though, that $ \server \wt{ \co{\aa}}$. This intuition
  %% motivates the use of the weak transition $X \wt{\co{\mu}} X'$ in \rfig{rules-mustset}.
  %% need for relying on weak transitions instead of strong transitions.
  %% To show why the use of sets is necessary,
  Assume that we defined the predicate~$\opMusti$
  using weak transitions on the server side for the case of
  communications. Recall the argument %unfolded
  put forward in the previous example.
  The inductive hypothesis now becomes the following:
  \begin{center}
    For every $\serverA', \serverB', \mu$ such that
    $\serverA \wt{\mu} \serverA'$ and $\client \st{\mu} \client'$,
    $\serverA' \asleq \serverB'$ implies $\musti{\serverB'}{\client'}$.
  \end{center}
  To use the inductive hypothesis we have to choose a $\server'$ such
  that $\server \wt{\co{\aa}} \server'$ and $\server' \asleq
  \tau.\co{b} \extc \tau.\co{c}$. This is still not enough for the
  entire proof to go through, because (modulo further $\tau$-moves)
  the particular $\server'$ we pick has to be related also to either
  $\co{b}$ or $\co{c}$. It is not possible to find such a
    $\server'$, because
    the two possible candidates
  %choose such a $\server'$, because the particular $\server'$ we can
  %choose
  are either $\co{b}$
  or $\co{c}$; neither of which can satisfy $\server' \asleq
  \tau.\co{b} \extc \tau.\co{c}$, as the right-hand side has not
  committed to a branch yet.

  %% the server $\tau.\co{b} \extc \tau.\co{c}$ has
  %% not commited yet to any branch.%  did not already decide which
  %% %  branch it would take.
  %% \leo{I think I understand by it's not super clear: is the following correct?
  %%   ... the particular $\server'$ we choose will either correspond to $\co{a} \Par \co{b}$
  %%   or to $\co{a} \Par \co{c}$; neither of which can satisfy $\server' \asleq
  %% \tau.\co{b} \extc \tau.\co{c}$, as the right-hand side has not commited to a
  %% branch yet.}

  If instead of a single state $\server$ in the novel definition of
  $\opMusti$ we used a set of %servers
  states and a suitable
  transition relation, the choice of either $\co{b}$ or $\co{c}$ will be
  suitably delayed. It suffices for instance to have the following states and transitions:
  %$X = \set{\co{b}, \co{c}}$ we have that
  $\set{\serverA} \wt{\co{\aa}} \set{\co{b}, \co{c}}.$\hfill$\qed$
  %and
%  $X \asleqset \tau.\co{b} \extc \tau.\co{c}$.
%  , for instance $X' =
%  \set{\co{b}, \co{c}}$
  %% and we defined an LTS such that $X \wt{
  %%   \co{\mu}} X'$ then in the proof the choice of either $\co{b}$ or $
  %% \co{c}$ would be suitably delayed.
  %% %In contrary, the use of sets allows to
  %% delay this choice. Indeed, by taking
  %% $X = \set{\co{b}, \co{c}}$ we have that $\set{\serverA} \wt{\co{\aa}} X$ and
  %% $X \asleqset \tau.\co{b} \extc \tau.\co{c}$.\hfill$\qed$
\end{example}

Now that we have motivated the main intuitions behind the definition of our
novel auxiliary predicate~$\opMustset$, we proceed with the formal definitions.

{\bfseries The LTS of sets.}
Let~$\pparts{ Z }$ be the set of  {\em non-empty} parts of~$Z$.
For any LTS~$\lts{\States}{ L }{~\st{}~}$, we define
for every $ X \in \pparts{ \States } $ and every $\alpha$ the sets
$$
\begin{array}{lll}
D{(\alpha, X)} & = & \setof{ \stateA }{ \exists \state \in X \suchthat \state \st{\alpha} \stateA },\\
\WD{(\alpha, X)} & = & \setof{ \stateA }{ \exists \state \in X \suchthat \state \wt{\alpha} \stateA }.
\end{array}
$$
Essentially we lift the standard notion of state derivative to sets of states.
We construct the LTS $\lts{\pparts{ \States }}{ \Acttau }{ \st{} }$
by letting $ X \st{ \alpha } D{(\alpha, X)}$ whenever $D{(\alpha, X)} \neq \emptyset$.
Similarly, we have $X \wt{ \alpha } \WD{(\alpha, X)}$ whenever $\WD{(\alpha, X)} \neq \emptyset$.
This construction is standard \cite{DBLP:conf/avmfss/CleavelandH89,DBLP:conf/aplas/BonchiCPS13,DBLP:journals/lmcs/BonchiSV22}
  and goes back to the determinisation of nondeterministic automata.
%, \ie whenever
%and there exists a $\server \in X$ such that $\server \st{ \alpha }$.

%%%%%%%%%%%%%%%%%%%%%%%%%%%%%%%%%%%%%%%%%%%%%%%%%%%%%%%%%%%%%%%%%%%%%%%%%%%%
%%%%%%%%%%%%%%%%%%%%%%%%%%%%%%%%%%%%%%%%%%%%%%%%%%%%%%%%%%%%%%%%%%%%%%%%%%%%
%%%%%%%%%%%%%%%%%%%%%%%%%%%%%%%%%%%%%%%%%%%%%%%%%%%%%%%%%%%%%%%%%%%%%%%%%%%%
\leaveout{
\begin{example}
  \label{ex:set-transitions}
    Let $\server = \tau.(\co{\aa} \Par \co{b}) \extc \tau.(\co{\aa} \Par \co{c})$.
    We have
    $$
    \begin{array}{l}
      \set{ \server } \st{ \tau} \set{ \co{\aa} \Par \co{b} }\\
      \set{ \server } \st{ \tau} \set{ \co{\aa} \Par \co{c} }\\
      \set{ \server } \st{ \tau} \set{ \co{\aa} \Par \co{b}, \co{\aa} \Par \co{c} }
      \end{array}
    $$
    \hfill$\qed$
  \end{example}
  In our arguments we will reason on the fact for a given $X \in \pparts{ \States }$
  all the states $p \in X$ must pass a given client $r$.
  The predicate $\opMusti$ is not directly usable to this aim, because
  in general $ \musti{X}{\client} $ does not imply that $\forall
  \server \musti{X}{\client}$. We show this in the next example.
  \begin{example}
    \label{ex:musti-not-directly-liftable-LTS-of-sets}
    Let $X = \set{\Nil, \co{\aa}}$ and $\client = \aa.\Unit$.
    We have that $\musti{X}{\client}$, because $X \st{\co{\aa}}$ and $X
    \st{\co{\aa}} Y$ implies that $Y = \set{ \Nil }$  and
    $$
    \begin{prooftree}
      {\good{\Unit}}
      \justifies
          {\begin{prooftree}
              \musti{\Nil}{\Unit}
              \justifies
                  { \musti{ \set{\Nil, \co{\aa}} }{\aa.\Unit} }
            \end{prooftree}%
          }
    \end{prooftree}
    $$
    On the other hand $\Nil \in X$  and $\Nmusti{\Nil}{\client}
    $.\hfill$\qed$
  \end{example}
%%   It seems natural that a server $\server$ and the singleton
%%   $\set{\server}$ be interchangeable when reasoning on~$\opMusti$,
%%   \ie $\musti{\server}{\client}$ if and only if
%%   $\musti{\set{\server}}{\client}$ for every~$\client$.

%% \TODO{If this is true, then there is the more general question:
%%   is it true that for every $X\in\pparts{A}$ .
%%   ($\forall \server \in X . \musti{\server}{\client}$) if and only if $\musti{X}{ \client}$ ?
\noindent

}
%%%%%%%%%%%%%%%%%%%%%%%%%%%%%%%%%%%%%%%%%%%%%%%%%%%%%%%%%%%%%%%%%%%%%%%%%%%%
%%%%%%%%%%%%%%%%%%%%%%%%%%%%%%%%%%%%%%%%%%%%%%%%%%%%%%%%%%%%%%%%%%%%%%%%%%%%
%%%%%%%%%%%%%%%%%%%%%%%%%%%%%%%%%%%%%%%%%%%%%%%%%%%%%%%%%%%%%%%%%%%%%%%%%%%%
Let $\opMustset$ be defined via the rules in \rfig{rules-mustset}.
This predicate let us reason on~$\opMusti$ via sets of servers,
in the following sense.
\begin{lemma}
  \label{lem:musti-if-mustset-helper}
  For every LTS $\genlts_A, \genlts_B$ and every
  $X \in \pparts{\StatesA}$, we have that
  $\mustset{X}{\client}$ if and only if for every $\serverA \in X
  \wehavethat \musti{\serverA}{\client}$.
\end{lemma}
%%%%% HIDDEN BECAUSE PERHAPS THE LEMMA IS NOT WORTH BEING PRESENTED
%%%%%%%%%%%%%%%%%%%%%% DO NOT EREASE
%%%%%%%%%%%%%%%%%%%%%% DO NOT EREASE
%% \begin{proof}
%% We proceed by rule induction on the derivation of the hypothesis $\mustset{X}{\client}$.

%% In the base case, $\mustset{X}{\client}$ has been derived using the rule \msetnow
%% such that $\good{\client}$ and $\musti{\server}{\client}$ follows from
%% an application of \mnow. %
%% %
%% %
%% In the inductive case $\mustset{X}{\client}$ has been derived using
%% rule \msetstep, and thus
%% \begin{enumerate}[(a)]
%% \item\label{musti-if-mustset-helper-h0} $\lnot \good{\client}$,
%% \item\label{musti-if-mustset-helper-h1} for every $\serverA \in X$ we have that
%%   $\csys{ \serverA }{ \client } \st{\tau}$,
%% \item\label{musti-if-mustset-helper-h2}
%%   $\forall X' \wehavethat X \st{ \tau } X' \neq \emptyset \implies \mustset{X'}{\client}$
%% \item\label{musti-if-mustset-helper-h3}
%%   $\forall \client' \wehavethat \client \st{ \tau } \client' \implies \mustset{X}{\client'}$
%% \item\label{musti-if-mustset-helper-h4}
%%   $\forall X', \mu \in \Actfin  \wehavethat X \wt{\co{\mu}} X' \neq \emptyset$
%%   and $\client \st{\mu} \client'$ imply that $\mustset{ X' }{ \client'}$
%% \end{enumerate}

%% Pick a $\server \in X$.
%% Since $\lnot \good{\client}$ and thanks to
%% (\ref{musti-if-mustset-helper-h1}), to show
%% $\musti{\serverA}{\client}$ we can apply the rule \mstep,
%% if we prove that
%% %%\begin{enumerate}[(i)]
%% %  \item $\lnot \good{\client}$,
%% %\item $\csys{ \serverA }{ \client } \st{\tau}$, and that
%% %%\item
%% \begin{equation*}
%%   \forall \serverA',
%%   \client' \wehavethat \csys{\serverA}{\client} \st{\tau}
%%   \csys{\serverA'}{\client'} \text{ imply that }\mustset{X'}{\client'}.
%%   \end{equation*}
%% %%\end{enumerate}
%% %The first requirement is a direct consequence of (\ref{musti-if-mustset-helper-h0}).
%% %% The first requirement follows from the fact that $\serverA \in X$ together with
%% %% (\ref{musti-if-mustset-helper-h1}).
%% %%To show the third requirement
%% Fix a silent transition $\csys{\serverA}{\client} \st{\tau} \csys{\serverA'}{\client'}$. We proceed by case analysis on how the transition
%% $\csys{\serverA}{\client} \st{\tau} \csys{\serverA'}{\client'}$ has been derived.
%% There are three cases, either the transition has been derived. There are the three following cases.
%% \begin{enumerate}[(1)]
%% \item[\stauserver]
%%   a $\tau$-transition performed by the server such that $\serverA \st{\tau} \serverA'$
%%   and that $\client' = \client$, or
%% \item[\stauclient]
%%   a $\tau$-transition performed by the client such that $\client \st{\tau} \client'$
%%   and that $\server' = \server$, or
%% \item[\scom]
%%   an interaction between the server $\serverA$ and the client $\client$ such that
%%   $\serverA \st{\mu} \serverA'$ and $\client \st{\co{\mu}} \client'$.
%% \end{enumerate}

%% \TODO{What does the inductive hypothesis state ?}

%% In case \stauserver we apply the inductive hypothesis. To do so we
%% have to provide a set $X'$ such that $\serverA' \in X'$ and $X \st{\tau} X'$.
%% We take $X' = \setof{\hat{\server}}{\serverA \st{\tau} \hat{\server}}$.
%% As $\serverA \st{\tau} \serverA'$, it must be the case that $\serverA' \in X'$ and we are done.
%% In case \stauclient, then conclusion is a direct consequence of the inductive hypothesis.
%% The argument for case \scom is similar to the one for case
%% \stauserver, the only difference being that we take $X' = \setof{\hat{\server}}{\serverA \st{\mu} \hat{\server}}$.
%% \end{proof}
%%%%%%%%%%%%%%%%%%%%%% DO NOT EREASE
%%%%%%%%%%%%%%%%%%%%%% DO NOT EREASE


%% \TODO{What about the converse implication ? Is the following lemma true ?
%% \begin{lemma}
%%   \label{lem:musti-if-mustset-helper-inverse}
%%   For every LTS $\genlts_A, \genlts_B$, %
%%   every $X \in \pparts{\StatesA}$, %
%%   and every $\client \in \StatesB$,
%%   if $\musti{\serverA}{\client}$ for every $\serverA \in X$,
%%   then $\mustset{X}{\client}$.
%% \end{lemma}
%% }


%%%%%%%%%%%%%%%%%%%%%% THE PREVIOUS LEMMA IS THE INTERESTING ONE
%%%%%%%%%%%%%%%%%%%%%%
%% \begin{lemma}
%%   \label{lem:musti-equals-mustset}
%%   For every LTS $\genlts_A, \genlts_B$, every
%%   $\server \in \StatesA$ and every $\client \in \StatesB$,
%%   $\musti{\server}{\client}$ if and only if $\mustset{\set{\server}}{\client}$.
%% \end{lemma}
%%%%%%%%%%%%%%%%%%%%%% DO NOT EREASE
%%%%%%%%%%%%%%%%%%%%%% DO NOT EREASE
%% \pl{
%% \begin{proof}
%% We first prove that
%% $\musti{\server}{\client}$ implies $\mustset{\set{\server}}{\client}$.
%% We proceed by rule induction on the derivation of $\mustset{\server}{\client}$.
%% In the base case it has been derived using \mnow, and thus $\good{\client}$,
%% which lets us conclude the base case by an applicaton of \mnow.
%% In the inductive case we show $\mustset{\set{\server}}{\client}$ by applying the rule \mstep.
%% We need to show the following.
%% \begin{enumerate}[(i)]
%% \item\label{musti-if-mustset-helper-g0} $\lnot \good{\client}$,
%% \item\label{musti-if-mustset-helper-g1} $\forall \serverA \in \set{\serverA}$ we have that
%%   $\csys{ \serverA }{ \client } \st{\tau}$,
%% \item\label{musti-if-mustset-helper-g2}
%%   $\forall X'$ such that $\set{\serverA} \st{ \tau } X' \neq \emptyset$, $\mustset{X'}{\client}$
%% \item\label{musti-if-mustset-helper-g3}
%%   $\forall \client'$, such that $\client \st{ \tau } \client'$, $\mustset{\set{\serverA}}{\client'}$
%% \item\label{musti-if-mustset-helper-g4}
%%   $\forall X', \mu \in \Act$,  $\set{\serverA} \wt{\co{\mu}} X' \neq \emptyset$
%%   and $\client \st{\mu} \client'$ imply $\mustset{ X' }{ \client'}$
%% \end{enumerate}

%% The first and the second requirements are a consequence of
%% the derivation of $\mustset{\server}{\client}$ by the rule \mstep.
%% To show (\ref{musti-if-mustset-helper-g2}) we apply \rlem{mx-forall} which requires us to show
%% that for each $\serverA'$, $\serverA \in X'$ implies $\mustset{\set{\serverA'}}{\client}$,
%% which is ensured by the inductive hypothesis.
%% The third requirement follows directly from an application of the inductive hypothesis.
%% The last requirement requires a bit more work.
%% We apply \rlem{mx-forall} which requires us to show
%% that for each $\serverA'$, $\serverA \in X'$ we have that
%% $\mustset{\set{\serverA'}}{\client'}$.
%% Let us pick such $\serverA'$. From the hypothesis $\set{\serverA} \wt{\co{\mu}} X'$
%% we know that $\serverA \wt{\mu} \serverA'$.
%% We continue by case analysis on the derivation of the reduction
%% $\serverA \wt{\mu} \server'$. There are two different cases to consider.
%% \begin{enumerate}
%% \item[\rname{wt-tau}]
%%   An application of \rptlem{mx-preservation}{mx-preservation-wt-mu} together
%%   with the inductive hypothesis ensures $\mustset{\set{\server'}}{\client'}$
%%   as required.
%% \item[\rname{wt-mu}]
%%   An application of \rptlem{mx-preservation}{mx-preservation-wt-nil} together
%%   with the inductive hypothesis ensures $\mustset{\set{\server'}}{\client'}$
%%   as required.
%% \end{enumerate}

%% The second direction, \ie $\mustset{\set{\server}}{\client}$ implies $\musti{\server}{\client}$,
%% is a direct consequence of \rlem{musti-if-mustset-helper}.
%% \end{proof}
%%%%%%%%%%%%%%%%%%%%%% DO NOT EREASE
%%%%%%%%%%%%%%%%%%%%%% DO NOT EREASE


  %% This definition of $\opMustset$ is notably helpful for our proof of soundness
  %% (\rlem{soundness-set}), in particular when we have to tackle
  %% the case where there is a communication between the server $\serverB$
  %% and the client $\client$ (\rptlem{soundness-set}{aim-soundness-3}).
%  We illustrate our reasoning through the following example to motivate the
%  use of the weak reduction as well as the use of sets.

  %% The next example illustrates why in rule \msetstep\ we use the weak
  %% transitions $X \wt{\co{ \mu }} X'$, and how the use of sets of
  %% servers help us deal with non-determinism.





To lift the predicates~$\bhvleqone$ and~$\bhvleqtwo$ to sets of servers,
we let $\accht{ X }{ \trace } = \setof{ O }{ \exists \server \in X . O
  \in \accht{\server}{ \trace }}$, and for every finite $X \in
\pparts{ \States }$, we write $X \conv$ to mean $\forall \state \in X \suchthat\state \conv$, we write $  X \cnvalong \trace$ to mean $\forall \state \in X \suchthat \state \cnvalong \trace$, and let

\begin{itemize}
\item $ X \cnvleqset \serverB$ to mean $\forall \trace \in \Actfin,
  \text{ if } X \cnvalong \trace
  \text{ then } \serverB \cnvalong \trace$,

\item
  $X \accleqset \serverB$ to mean $\forall \trace \in \Actfin,
  X \cnvalong{\trace} \implies \accht{ X }{ \trace } \ll \accht{ \serverB }{ \trace }$,

\item
  $X \asleqset \serverB$ to mean $X \cnvleqset \serverB \wedge X \accleqset \serverB$.
\end{itemize}
%%$$
%% \begin{array}{lcl}
%%   X \cnvleqset \serverB &\text{to mean}& \forall \trace \in \Actfin,
%%   \text{ if } X \cnvalong \trace
%%   \text{ then } \serverB \cnvalong \trace
%%   \\[5pt]
%%   %% X \accleqset \serverB & \text{to mean}&
%%   %% \forall \trace \in \Actfin,
%%   %% X \cnvalong{\trace} \implies\\
%%   %% && \forall O \in \accht{ \serverB }{ \trace } \wehavethat \\
%%   %% && \exists \serverA \in X \suchthat \\
%%   %% &&\exists \widehat{O} \in \accht{ \serverA }{ \trace } . \widehat{ O } \subseteq O
%%   %% \\[5pt]
%%   X \accleqset \serverB & \text{to mean}& \forall \trace \in \Actfin,
%%   X \cnvalong{\trace} \implies\\
%% &&  \accht{ X }{ \trace } \ll \accht{ \serverB }{ \trace }
%%   \\[5pt]
%%   X \asleqset \serverB & \text{to mean}& X \cnvleqset \serverB \wedge X \accleqset \serverB
%% \end{array}
%% $$

These definitions imply immediately the following equivalences,
  $
\set{\serverA} \accleqset \serverB \Longleftrightarrow p \bhvleqone \serverB$,
$\set{\serverA} \cnvleqset \serverB \Longleftrightarrow p \bhvleqtwo \serverB $
and thereby the following lemma.
\begin{lemma}
  \label{lem:alt-set-singleton-iff}
  For every LTS $\genlts_A, \genlts_B, \serverA \in \StatesA$,
  $\serverB \in \StatesB$,
  $\serverA \asleq \serverB$ if and only if $\set{\serverA} \asleqset \serverB$.
\end{lemma}
%% \begin{proof}
%%   The results follow directly from the following facts.
%% \end{proof}


%% \TODO{
%%   In \rlem{bhvleqone-preserved} and \rlem{bhvleqtwo-preserved}
%%   we have a stronger hypothesis than
%%   $X \wt{\mu} X'$ in the code.
%%   We say that for each
%%   $\serverA \in X$ if $\serverA \wt{\mu} \serverA'$ then $\serverA' \in X'$.
%%   I fix this tomorrow.
%% }


\newcommand{\completewrt}[2]{\ensuremath{\mathsf{cwrt} \, (#1,#2)}}

The preorder~$\asleqset$ is preserved by $\tau$-transitions on
its right-hand side, and by visible transitions
%actions performed
on both sides.
We reason separately on the two auxiliary
preorders $\cnvleqset$ and $\accleqset$.
We need one
further notion.
%% \gb{This should no longer be necessary.                                             %%
%% We say that a set $X'$ is {\em complete with respect                                %%
%%   to} a set $X$ and an action $\mu$, denoted                                        %%
%% $X' \completewrt{X}{\mu}$, if two properties hold: % are true:                      %%
%% \begin{enumerate}[i)]                                                               %%
%% \item $X \wt{ \mu } X'$ and                                                         %%
%% \item                                                                               %%
%%   for every $\server \in X$, if $\server \wt{\mu} \server'$ then $\server' \in X'$. %%
%% \end{enumerate}                                                                     %%
%% Note that if $X' \completewrt{X}{\mu}$ and $X''                                     %%
%% \completewrt{X}{\mu}$ then $X' = X''$, because the LTSs at hand are image finite.   %%
%% }                                                                                   %%
\begin{lemma}
  \label{lem:bhvleqone-preserved}
  Let $\genlts_\StatesA, \genlts_\StatesB \in \obaFW$.
  For every set $X \in \pparts{ \StatesA }$, and
  $\serverB \in \StatesB$, such that
  $X \cnvleqset \serverB$,
  \begin{enumerate}
  \item\label{pt:bhvleqone-preserved-by-tau}
    $\serverB \st{ \tau } \serverB'$ implies $X \cnvleqset \serverB'$,
  \item\label{pt:bhvleqone-preserved-by-mu}
    $X \convi$, $X \wt{\mu} X'$ and $\serverB \st{\mu} \serverB'$ imply $X' \cnvleqset \serverB'$.
  \end{enumerate}
\end{lemma}


\begin{lemma}
  \label{lem:bhvleqtwo-preserved}
  Let $\genlts_\StatesA, \genlts_\StatesB \in \obaFW$.
  For every
  $X, X' \in \pparts{ \StatesA }$ and $ \serverB \in \StatesB$,
  such that $X \accleqset \serverB$, then
  \begin{enumerate}
  \item\label{pt:bhvleqtwo-preserved-by-tau}
    $\serverB \st{ \tau } \serverB'$ implies $X \accleqset \serverB'$,
  \item\label{pt:bhvleqtwo-preserved-by-mu}
    if $X \convi$ then for every $\mu \in \Act$, every $\serverB'$ and $X'$ such that $\serverB \st{\mu} \serverB'$ and $X \wt{\mu} X'$
    we have $X' \accleqset \serverB'$.
  \end{enumerate}
\end{lemma}
%% \begin{proof}
%%   We first prove \rptlem{bhvleqtwo-preserved}{bhvleqtwo-preserved-by-tau}.
%%   We must show $X \accleqset \serverB'$.
%%   Let us fix a server $\serverB''$ such that
%%   $X \cnvalong \trace$ and $\serverB' \wt{\trace} \serverB'' \stable$.
%%   An application of the hypothesis $X \accleqset \serverB$
%%   provides us with two servers $\serverA' \in X$ and $\serverA''$ such that
%%   $\serverA' \wt{s} \serverA'' \stable$ and $O(\serverA'') \subseteq O(\serverB'')$
%%   as required.

%%   We now prove \rptlem{bhvleqtwo-preserved}{bhvleqtwo-preserved-by-mu}.
%%   Let us fix a server $\serverB''$ such that
%%   $X' \cnvalong \trace$ and $\serverB' \wt{\trace} \serverB'' \stable$.
%%   An application of the hypothesis $X \accleqset \serverB$
%%   requires us to show $X \cnvalong \mu.\trace$ which follows
%%   from $X \convi$ and $X' \cnvalong \trace$.
%%   We then obtain two servers $\serverA \in X$ and $\serverA''$ such that
%%   $\serverA \wt{\mu.\trace} \serverA'' \stable$ and $O(\serverA'') \subseteq O(\serverB'')$
%%   We decompose the reduction $\serverA \wt{\mu.\trace} \serverA''$ and obtain
%%   a server $\serverA'$ such that
%%   $\serverA \wt{\mu} \serverA' \wt{\trace} \serverA''$ and allows us
%%   to conclude this case.
%% \end{proof}



%%%%%%%%%%%%%%%%%%% HIDDEN FOR SPACE REASONS
%%%%%%%%%%%%%%%%%%%
%% We now gather sufficient conditions for the servers on the
%% right-hand side of~$\asleqset$ to perform a silent move,
%% and for servers on the left-hand side of~$\asleqset$ to perform
%% weakly visible actions.

%% \begin{lemma}%[\coqSou{unhappy_must_st_nleqx}]
%%   \label{lem:stability-Nbhvleqtwo}
%%   Let $\genlts_\StatesA, \genlts_\StatesB \in \obaFW$ and $\genlts_\StatesC \in \obaFB$.
%%   For every $X \in \pparts{ \StatesA }$ and
%%   $\serverB \in \StatesB $ such that
%%   $X \asleqset \serverB$, for every $\client \in \StatesC$
%%   if $\lnot \good{\client}$ and $\mustset{X}{\client}$
%%   then $\csys{\serverB}{\client} \st{\tau}$.
%% \end{lemma}


%% \gb{
%% \begin{lemma}
%%   \label{lem:empty-nleqx}
%%   Let $\genlts_\StatesA, \genlts_\StatesB \in \obaFW$.
%%   For every $X \in \pparts{ \StatesA }$ and
%%   $\serverB, \serverB' \in \StatesB$,
%%   such that $X \cnvleqset \serverB$, then
%%   for every $\mu \in \Act$, if
%%   $X \convi$, $X \accleqset \serverB$ and  $\serverB \st{\mu}
%%   \serverB'$
%%   then $X \Nwt{\mu}$.
%% \end{lemma}
%% \begin{proof}
%%   First note that $X \convi$ and $X \Nwt{\mu}$ imply $X \cnvalong \mu$.
%%   Then, from $X \cnvleqset \serverB$ and $X \cnvalong \mu$ we have that
%%   $\serverB \cnvalong \mu$ and thus $\serverB' \convi$.
%%   As $\serverB'$ terminates, it must be the case that
%%   there exists a $\serverB''$ such that
%%   $\serverB \wt{\mu} \serverB' \wt{\varepsilon} \serverB'' \stable$.
%%   We then apply the hypothesis $X \accleqset \serverB$ to obtain
%%   two servers $\serverA' \in X$ and $\serverA''$ such that
%%   $\serverA' \wt{\mu} \serverA'' \stable$ which contradicts the hypothesis
%%   $X \Nwt{\mu}$ and we are done.
%% \end{proof}
%% }


%% \begin{lemma}
%%   \label{lem:empty-nleqx}
%%   \label{lem:X-must-perform-visible-action}
%%   Let $\genlts_\StatesA, \genlts_\StatesB \in \obaFW$.
%%   For every $X \in \pparts{ \StatesA }$ and
%%   $\serverB, \serverB' \in \StatesB$,
%%   such that $X \cnvleqset \serverB$, then
%%   for every $\mu \in \Act$, if
%%   $X \convi$,  $X \Nwt{\mu}$ and $\serverB \st{\mu} \serverB'$ then $\lnot (X \accleqset \serverB)$.
%% \end{lemma}
%%%%%%%%%%%%%%%%%%%%%%%%%%%%%%%%%%%%%%%%%%%%%%% OLD LEMMA PROVEN VIA A CONTRADICTION
%%%%%%%%%%%%%%%%%%%%%%%%%%%%%%%%%%%%%%%%%%%%%%% OLD LEMMA PROVEN VIA A CONTRADICTION
%%%%%%%%%%%%%%%%%%%%%%%%%%%%%%%%%%%%%%%%%%%%%%% OLD LEMMA PROVEN VIA A CONTRADICTION
%% \begin{lemma}%[\coqSou{unhappy_must_st_nleqx}]
%%   \label{lem:stability-Nbhvleqtwo}
%%   Let $\genlts_\StatesA$, $\genlts_\StatesB$ be $\obaFW$ and $\genlts_\StatesC$ be $\obaFB$.
%%   For every $X \in \pparts{ \StatesA }$, $\serverB \in \pparts{ \StatesB }$ and $\client \in \StatesC$,
%%   if $\lnot \good{\client}$ and $\mustset{X}{\client}$ then
%%   $\csys{\serverB}{\client} \stable$ implies $\lnot (X \asleqset \serverB)$.
%% \end{lemma}
%% \begin{proof}
%%   We show that assuming $X \asleqset \serverB$ leads to a contradiction and thus a derivation of $\bot$.
%%   First, from the hypothesis $\csys{\serverB}{\client} \stable$ we know the following statements hold.
%%   \begin{enumerate}[(a)]
%%   \item\label{stability-Nbhvleqtwo-h1}
%%     $\serverB \stable$
%%   \item\label{stability-Nbhvleqtwo-h2}
%%     $\client \stable$
%%   \item\label{stability-Nbhvleqtwo-h3}
%%     there is no $\mu \in \Act$ such that $\serverB \st{\mu}$ and $\client \st{\co{\mu}}$
%%   \end{enumerate}

%%   The hypotheses $\lnot \good{\client}$ and $ \mustset{X}{\client}$ together with
%%   \rlem{mustx-terminate-ungood} imply $X \convi$ and thus $X \cnvalong \varepsilon$.
%%   An application of $X \accleqset q$ with $\trace = \varepsilon$, $\stateB' = \stateB$,
%%   and (\ref{stability-Nbhvleqtwo-h1}) gives us a $\serverA'$ such that
%%   $\serverA \wt{\varepsilon} \serverA' \stable$ and $O(\serverA') \subseteq O(\serverB)$.
%%   An application of $\mustset{X}{\client}$, \rlem{wt-nil-mx} and \rlem{mx-sub} ensures that
%%   $\mustset{\set{\serverA'}}{\client}$.
%%   As $\lnot \good{\client}$, it must be the case that $\mustset{\set{\serverA'}}{\client}$
%%   is derived using the rule \mstep which implies that $\csys{\serverA'}{\client} \st{\tau}$.
%%   We contine with a case analysis on how this transition has been derived.
%%   As $\serverA'$ is stable, and from (\ref{stability-Nbhvleqtwo-h2}) we know that
%%   it must be derived using \rname{s-com} such that
%%   $\serverA' \st{\mu}$ and $\client \st{\co{\mu}}$ for some $\mu \in \Act$.
%%   Finally, we distinguish whether $\mu$ is an input or an output.
%%   In the first case $\mu$ is an input.
%%   From $O(\serverA') \subseteq O(\serverB)$ it must be the case that $\serverB \st{\co{\mu}}$
%%   holds, which contradicts the hypothesis that $\csys{\serverB}{\client} \stable$.
%%   In the second case $\mu$ is an output.
%%   As $\serverB$ is $\obaFW$, we can apply the axiom for forwarders and get a server $\stateB'$
%%   such that $\serverB \st{\co{\mu}}$. This implies the existence of a communication
%%   between $\serverB$ and $\client$, which contradicts
%%   the hypothesis that $\csys{\serverB}{\client} \stable$ and lets us conclude this case.
%% \end{proof}
%%%%%%%%%%%%%%%%%%%%%%%%%%%%%%%%%%%%%%%%%%%%%%%%%%%%%%%%%%%%%%%%%%%%%%%%%%%%%%%%%%%%%%%%%%%%%%%%%%%
%%%%%%%%%%%%%%%%%%%%%%%%%%%%%%%%%%%%%%%%%%%%%%%%%%%%%%%%%%%%% END OLD LEMMA

%%%%%%%%%%%%%%%%%%% HIDDEN FOR SPACE REASONS
%%%%%%%%%%%%%%%%%%%


The main technical work for the proof of soundness is carried out by the next lemma.
\begin{lemma}
  \label{lem:soundness-set}
  Let $\genlts_\StatesA, \genlts_\StatesB \in \obaFW$ and
  $\genlts_\StatesC \in \obaFB$.
  For every set of servers $X \in \pparts{ \StatesA }$,
  server $\serverB \in \StatesB$ and client $\client \in \StatesC$,
  if $\mustset{X}{\client}$ and $X \asleqset \serverB$ then $\musti{\serverB}{\client}$.
\end{lemma}



\begin{proposition}[Soundness]
  \label{prop:bhv-soundness}
  For every $\genlts_A, \genlts_B \in \obaFB$ and
  servers $\serverA \in \StatesA, \serverB \in \StatesB $,
  if $\liftFW{ \serverA } \asleq \liftFW{ \serverB }$ then $\serverA \testleqS \serverB$.
\end{proposition}
\begin{proof}
  \rlem{musti-obafb-iff-musti-obafw}
  %\rcor{testleq-obafb-iff-testleq-obafw}
  ensures that the result follows if we prove that
$\liftFW{\serverA} \testleqS \liftFW{\serverB}.$
Fix a client $\client$ such that $\musti{\liftFW{\serverA}}{\client}$.
\rlem{soundness-set} implies the required
$\musti{\liftFW{\serverB}}{\client}$, if we show that
\begin{enumerate}[(i)]
  \item $\mustset{ \set{ \liftFW{\serverA} } }{ \client }$, and that
  \item $\set{ \liftFW{ \serverA } } \mathrel{\preccurlyeq^{\mathsf{set}}_{\mathsf{AS}}} \liftFW{\serverB}$.
\end{enumerate}
The first fact follows from the assumption that $\musti{\liftFW{\serverA}}{\client}$
and \rlem{musti-if-mustset-helper} applied to the singleton
$\set{\liftFW{\serverA}}$.
The second fact follows from the hypothesis that $\liftFW{ \serverA }
\asleq \liftFW{ \serverB }$ and \rlem{alt-set-singleton-iff}.
%%%% ORIGINAL PROOF
  %% Fix a client $\client$ such that $\musti{\liftFW{\serverA}}{\client}$.
  %% The required $\musti{\liftFW{\serverB}}{\client}$ follows from
  %% \rlem{soundness-set}, if we show that
  %% $\set{\serverA} \asleq \serverB$.
  %% We obtain $\set{\serverA} \asleq \serverB$ by
  %% an application of \rlem{alt-set-singleton-iff} in the hypothesis $\serverA \asleq \serverB$.
  %% The requirement $\musti{\set{\liftFW{\serverA}}}{\client}$ is a consequence of
  %% an application of \rlem{musti-equals-mustset} together with the hypothesis that
  %% $\musti{\liftFW{\serverA}}{\client}$.
\end{proof}

\subsection{Technical results to prove soundness}
\label{sec:appendix-soundness}

We now discuss the proofs of the main technical results
behind \rprop{bhv-soundness}. The predicate $\opMustset$ is  monotonically decreasing with respect to its first
argument, and it enjoys properties analogous to the ones
of~$\opMusti$ that have been shown in \rlem{must-terminate} and
\rlem{musti-preserved-by-left-tau}.

\begin{lemma}
  \label{lem:mx-sub}
  For every LTS  $\genlts_\StatesA, \genlts_\StatesB$ and
  every set $X_1 \subseteq X_2 \subseteq \StatesA$, client $\client \in \StatesB$,
  if $\mustset{X_2}{\client}$ %and $X_1 \subseteq X_2$
  then $\mustset{X_1}{\client}$.
\end{lemma}

\begin{lemma}
  \label{lem:mustx-terminate-ungood}
  Let $\genlts_\StatesA \in \obaFW$ and $\genlts_\StatesB \in \obaFB$.
  For every set $X \in \pparts{ \StatesA }$, client $\client \in \StatesB$,
  if $\lnot \good{\client}$ and $ \mustset{X}{\client}$ then $X \convi$.
\end{lemma}

\begin{lemma}
  \label{lem:wt-nil-mx}
  For every $\genlts_\StatesA, \genlts_\StatesB$,
  every set $X_1, X_2 \in \pparts{\StatesA}$, and client $\client \in \StatesB$,
  if $\mustset{X_1}{\client}$ and $X_1 \wt{\varepsilon} X_2$ then $\mustset{X_2}{\client}$.
\end{lemma}

\begin{lemma}
  \label{lem:mx-preservation}
  For every LTS $\genlts_A, \genlts_B$ and every
  $X \in \pparts{\StatesA}$ and $\client \in \StatesB$,
  if $\mustset{X}{\client}$ then for every $X'$ such that
  \begin{enumerate}[(a)]
  \item\label{pt:mx-preservation-wt-nil}
    If $X \wt{\varepsilon} X'$ then $\mustset{X'}{\client}$,
  \item\label{pt:mx-preservation-wt-mu}
    For any $\mu \in \Act$ and client $\client'$,
    if $X \wt{\mu} X'$, $\client \st{\co{\mu}} \client'$ and $\lnot \good{\client}$,
    then $\mustset{X'}{\client'}$.
  \end{enumerate}
\end{lemma}



\begin{lemma}
  \label{lem:mx-forall}
  Given two LTS $\genlts_A$ and $\genlts_B$ then for every
  $X \in \pparts{\StatesA}$ and $\client \in \StatesB$,
  if for each $\serverA \in X$ we have that $\musti{\serverA}{\client}$,
  then $\mustset{X}{\client}$.
\end{lemma}



\begin{lemma}%[\coqSou{unhappy_must_st_nleqx}]
  \label{lem:stability-Nbhvleqtwo}
  Let $\genlts_\StatesA, \genlts_\StatesB \in \obaFW$ and $\genlts_\StatesC \in \obaFB$.
  For every $X \in \pparts{ \StatesA }$ and
  $\serverB \in \StatesB $ such that
  $X \asleqset \serverB$, for every $\client \in \StatesC$
  if $\lnot \good{\client}$ and $\mustset{X}{\client}$
  then $\csys{\serverB}{\client} \st{\tau}$.
\end{lemma}
%% \noindent%
%% \textbf{\rlem{stability-Nbhvleqtwo}}
%%  Let $\genlts_\StatesA, \genlts_\StatesB \in \obaFW$ and $\genlts_\StatesC \in \obaFB$.
%%   For every $X \in \pparts{ \StatesA }$ and
%%   $\serverB \in \StatesB $ such that
%%   $X \asleqset \serverB$, for every $\client \in \StatesC$
%%   if $\lnot \good{\client}$ and $\mustset{X}{\client}$
%%   then $\csys{\serverB}{\client} \st{\tau}$.
\begin{proof}
  If either $\serverB \st{\tau}$ or $\client \st{\tau}$
  then we prove that $\csys{\serverB}{\client}$ performs
  a $\tau$-transition vis \stauserver or \stauclient, so suppose that both
  $\serverB$ and $\client$ are stable.
  Since $\serverB$ is stable we know that
  $$
  \accht{q}{ \varepsilon } = \set{ O(q) }
  $$
  The hypotheses $\lnot \good{\client}$ and $ \mustset{X}{\client}$ together with
  \rlem{mustx-terminate-ungood} imply $X \convi$ and thus $X \cnvalong \varepsilon$.
  The hypothesis $X \accleqset q$ with $\trace = \varepsilon$,
  gives us a $\serverA'$ such that $\serverA \wt{\varepsilon} \serverA' \stable$
  and $O(\serverA') \subseteq O(\serverB)$.
  By definition there exists the weak silent trace
  $ X \wt{ } X'$ for some set $X'$ such that $ \set{ \serverA' }
  \subseteq X' $.
  The hypothesis  $\mustset{X}{\client}$ together with
  \rlem{wt-nil-mx} and \rlem{mx-sub} ensure that
  $\mustset{\set{\serverA'}}{\client}$.

  As $\lnot \good{\client}$, $\mustset{\set{\serverA'}}{\client}$
  must have been derived using rule \mstep which implies that $\csys{\serverA'}{\client} \st{\tau}$.
  As both $\client$ is stable by assumption, and $\serverA'$ is stable
  by definition, this $\tau$-transition must have been
  derived using \rname{s-com}, and so %
%  \begin{center}
    $\serverA' \st{\mu}$ and $\client \st{\co{\mu}}$ for some $\mu \in \Act$.
%  \end{center}
  Now we distinguish whether $\mu$ is an input or an output.
  In the first case $\mu$ is an input. Since $\genlts_{\StatesB} \in
  \obaFW$ we use the \boom axiom to prove $\serverB \st{ \mu }$, and thus
  $\csys{ \serverB }{ \client} \st{ \tau }$ via rule \scom.
  In the second case $\mu$ is an output, and so the inclusion
  $O(\serverA') \subseteq O(\serverB)$ implies that $\serverB
  \st{ \mu }$, and so we conclude again applying rule \scom.
\end{proof}




  \begin{lemma}
  \label{lem:empty-nleqx}
  \label{lem:X-must-perform-visible-action}
  Let $\genlts_\StatesA, \genlts_\StatesB \in \obaFW$.
  For every $X \in \pparts{ \StatesA }$ and
  $\serverB, \serverB' \in \StatesB$,
  such that $X \asleqset \serverB$, then
  for every $\mu \in \Act$, if
  $X \cnvalong \mu$ and $\serverB \st{\mu} \serverB'$ then $X \wt{\mu}$.
  \end{lemma}
  \begin{proof}
  Then, from $X \cnvleqset \serverB$ and $X \cnvalong \mu$ we have that
  $\serverB \cnvalong \mu$ and thus $\serverB' \convi$.
  As~$\serverB'$ converges, there must exist $\serverB''$ such that
  $$
  \serverB \wt{\mu} \serverB' \wt{\varepsilon} \serverB'' \stable
  $$
  and so $\accfwp{ \serverB }{ \mu }{ \st{}_\StatesB } \neq \emptyset$.
  An application of the hypothesis $X \accleqset \serverB$ implies that there exists
  a set $\widehat{ O } \in \accfwp{ X }{ \mu }{ \st{}_\StatesA }$,
  and thus there exist two servers $\serverA' \in X$ and $\serverA''$
  such that $\serverA' \wt{\mu} \serverA'' \stable$.
  Since $ \serverA' \in X $ it follows that $X \wt{\mu}$.
  \end{proof}


%% \begin{lemma}
%%   \label{lem:empty-nleqx}
%%   \label{lem:X-must-perform-visible-action}
%%   Let $\genlts_\StatesA, \genlts_\StatesB \in \obaFW$.
%%   For every $X \in \pparts{ \StatesA }$ and
%%   $\serverB, \serverB' \in \StatesB$,
%%   such that $X \cnvleqset \serverB$, then
%%   for every $\mu \in \Act$, if
%%   $X \convi$,  $X \Nwt{\mu}$ and $\serverB \st{\mu} \serverB'$ then $\lnot (X \accleqset \serverB)$.
%% \end{lemma}
%% %% \noindent%
%% %% \textbf{\rlem{X-must-perform-visible-action}}
%% %%  Let $\genlts_\StatesA, \genlts_\StatesB \in \obaFW$.
%% %%   For every $X \in \pparts{ \StatesA }$ and
%% %%   $\serverB, \serverB' \in \StatesB$,
%% %%   such that $X \cnvleqset \serverB$, then
%% %%   for every $\mu \in \Act$, if
%% %%   $X \convi$,  $X \Nwt{\mu}$ and $\serverB \st{\mu} \serverB'$ then
%% %%   $\lnot (X \accleqset \serverB)$.
%% %%%%%%%%%%%%%%%%%%%%%%%%%%%%%%%%%%%%%%%%%%%%%%%%% THIS PROOF SHOWS THAT
%% %%%%%%%%%%%%%%%%%%%%%%%%%%%%%%%%%%%%%%%%%%%%%%%%% INFORMATION IS BADLY
%% %%%%%%%%%%%%%%%%%%%%%%%%%%%%%%%%%%%%%%%%%%%%%%%%% TREATED BY THE
%% %%%%%%%%%%%%%%%%%%%%%%%%%%%%%%%%%%%%%%%%%%%%%%%%% CURRENT TREATMENT OF SOUDNESS
%% \begin{proof}
%%   We show that assuming $X \accleqset \serverB$ let us derive $\bot$.
%%   First note that $X \convi$ and $X \Nwt{\mu}$ imply $X \cnvalong \mu$.
%%   Then, from $X \cnvleqset \serverB$ and $X \cnvalong \mu$ we have that
%%   $\serverB \cnvalong \mu$ and thus $\serverB' \convi$.
%%   As $\serverB'$ terminates, it must be the case that
%%   there exists a $\serverB''$ such that
%%   $\serverB \wt{\mu} \serverB' \wt{\varepsilon} \serverB'' \stable$.
%%   We then apply the hypothesis $X \accleqset \serverB$ to obtain
%%   two servers $\serverA' \in X$ and $\serverA''$ such that
%%   $\serverA' \wt{\mu} \serverA'' \stable$. Together with the
%%   hypothesis $X \Nwt{\mu}$ this let us prove $\bot$.%
%% %  which contradicts the hypothesis $X \Nwt{\mu}$ and we are done.
%% \end{proof}



  %% \newcommand{\completewrt}[2]{\ensuremath{\mathsf{cwrt} \, (#1,#2)}}

  %% We say that a set $X'$ is {\em complete with respect to}
  %% a set $X$ and an action $\mu$, denoted $X' \completewrt{X}{\mu}$,
  %% %% gio: at present (23/01) this predicate is defined via a logical __and__
  %% %% an alternative could be to define it via an implication.
  %% if two properties are true:
  %% \begin{enumerate}[i)]
  %% \item $X \wt{ \mu } X'$ and
  %% \item
  %%   for every $\server \in X$, if $\server \wt{\mu} \server'$ then $\server' \in X'$.
  %% \end{enumerate}
%%  \TODO{IDK if this definition is fine or if we must define a ternary relation on $X$, $X'$ and $\mu$.}

\noindent%
\textbf{\rlem{bhvleqone-preserved}}
  Let $\genlts_\StatesA, \genlts_\StatesB \in \obaFW$.
  For every set $X \in \pparts{ \StatesA }$, and
  $\serverB \in \StatesB$, such that
  $X \cnvleqset \serverB$ then
  \begin{enumerate}
  \item
    $\serverB \st{ \tau } \serverB'$ implies $X \cnvleqset \serverB'$,
  \item
    if $X \convi$ and $\serverB \st{\mu} \serverB'$
    then for every set $X \wt{\mu} X'$ % $X \wt{\mu} X'$
    we have that $X' \cnvleqset \serverB'$.
%    for every complete set $X'$ such that $X \wt{\mu} X'$
%    we have that $X' \cnvleqset \serverB'$.
  \end{enumerate}
\begin{proof}
  We first prove \rpt{bhvleqone-preserved-by-tau}.
  Let us fix a trace $\trace$ such that $X \cnvalong \trace$.
  We must show $\serverB' \cnvalong \trace$.
  An application of the hypothesis $X \cnvleqset \serverB$ ensures $\serverB \cnvalong \trace$.
  From the transition $\serverB \st{ \tau } \serverB'$ and the fact that
  convergence is preserved by the $\tau$-transitions we have that
  $\serverB' \cnvalong \trace$ as required.

  We now prove \rpt{bhvleqone-preserved-by-mu}.
  Fix a trace $\trace$ such that $X' \cnvalong \trace$.
  Since  $\serverB \st{\mu} \serverB'$, the required  $\serverB' \cnvalong \trace$ follows
  from $ \serverB \cnvalong \mu.\trace $.
  Thanks to the hypothesis $X \cnvleqset \serverB$ it suffices to show that
  $X \cnvalong \mu.\trace'$, \ie that
  $$
  \forall \server \in X . \server \cnvalong \mu.\trace'
  $$
  Fix a server $\server \in X$.
  We must show that
  \begin{enumerate}
  \item $\server \convi$ and that
  \item for any $\server'$ such that $\server \wt{\mu} \server'$ we have $\server' \cnvalong \trace$.
  \end{enumerate}
  The first requirement follows from the hypothesis $X \convi$.
  The second requirement follows from the transition $\server \wt{\mu} \server'$,
  from the assumption $X' \cnvalong \trace$,
  and the hypothesis that  $X \wt{\mu} X'$, which ensures that
  $\server' \in X'$ and thus by definition of $X' \cnvalong \trace$ that $\server' \cnvalong \trace$.
  %%%%%%%%%%%%%%%%%%%%%%%% OLD PROOF
  %%%%%%%%%%%%%%%%%%%%%%%%
  %% We now prove \rpt{bhvleqone-preserved-by-mu}.
  %% Let us fix a trace $\trace$ such that $X' \cnvalong \trace$.
  %% As $\serverB \st{\mu} \serverB'$, it is enough to show $\serverB \cnvalong \mu.\trace$
  %% to obtain $\serverB' \cnvalong \trace$.
  %% To do so we apply the hypothesis $X \cnvleqset \serverB$ which requires us to show
  %% $X \cnvalong \mu.\trace'$.
  %% Let us fix a server $\serverA \in X$.
  %% We must show $\serverA \convi$ and that for any $\serverA'$ such that
  %% $\serverA \wt{\mu} \serverA'$ we have $\serverA' \cnvalong \trace$.
  %% The first requirement is a direct consequence of $X \convi$ and $\serverA \in X$.
  %% The second requirement follows from the reduction $\serverA \wt{\mu} \serverA'$
  %% which implies that $\serverA' \in X'$, together with $X' \cnvalong \trace$.
\end{proof}



\noindent%
\textbf{\rlem{bhvleqtwo-preserved}}
  Let $\genlts_\StatesA, \genlts_\StatesB \in \obaFW$.
  For every
  $X, X' \in \pparts{ \StatesA }$ and $ \serverB \in \StatesB$,
  such that $X \accleqset \serverB$, then
  \begin{enumerate}
  \item
    $\serverB \st{ \tau } \serverB'$ implies $X \accleqset \serverB'$,
  \item
    for every $\mu \in \Act$,
    if $X \convi$, then for every  $\serverB \st{\mu} \serverB'$ and set $X \wt{\mu} X'$
    we have $X' \accleqset \serverB'$.
  \end{enumerate}
  \newcommand{\attaboy}{\server_{\mathsf{attaboy}}}
  \begin{proof}
      To prove \rpt{bhvleqtwo-preserved-by-tau}
      fix a trace $\trace \in \Actfin$ such that
      $X \acnvalong \trace$.
      We have to explain why
      $\accht{ X }{ \trace } \ll \accht{ \serverB' }{ \trace }$.
      By unfolding the definitions, this amounts to showing that
      \begin{equation}
        \label{eq:attaboy-1}
        \tag{$\star$}
        \forall O \in \accht{ \serverB' }{\trace} \wehavethat
        \exists \attaboy \in X \suchthat
        \exists \widehat{O} \in \accht{ \attaboy }{ \trace } \suchthat
        \widehat{O} \subseteq O
      \end{equation}

      Fix a set $O \in \accht{\serverB'}{\trace}$.
      By definition there exists some $\serverB''$ such that
      $\serverB' \wt{\trace} \serverB'' \stable$, and that $O =
      O(\serverB'')$.
      The definition of $\accht{-}{-}$ and the silent move $\serverB \st{ \tau } \serverB'$
      ensures that $O \in \accht{\serverB}{ \trace }$.
      The hypothesis $ X \accleqset \serverB $ and that
      $X \cnvalong{ \trace } $ now imply that
      $\accht{ X }{ \trace } \ll \accht{ \serverB }{ \trace }$,
      which together with $ O \in \accht{\serverB}{ \trace }$
      implies exactly \req{attaboy-1}.
      %% by definition of $\accht{
      %%   X }{ \trace }$ there exists a $\attaboy \in X$
      %% such that there exists a set $\widehat{ O } \in
      %% \accht{ \attaboy }{ \trace }$ such that $ \widehat{ O
      %% } \subseteq O$.


    We now prove \rpt{bhvleqtwo-preserved-by-mu}.
    To show $X' \accleqset \serverB'$ fix a trace $\trace \in \Actfin$
    such that $X' \cnvalong \trace$.


    We have to explain why $\accht{X}{\trace} \ll
    \accht{\serverB'}{\trace}$.
    By unfolding the definitions we obtain our aim,
      \begin{equation}
      \label{eq:aim}
      \tag{$\star\star$}
      \forall O \in \accht{ \serverB' }{ \trace } \wehavethat
      \exists \attaboy \in X' \suchthat
      \exists \widehat{ O } \in \accht{ \attaboy }{ \trace } \suchthat
      \widehat{ O } \subseteq O
    \end{equation}

      To begin with, we prove that $X \cnvalong \mu.\trace$.
      Since $X \wt{\mu} X'$ we know that $ X \wt{ \mu } X'$.
      This, together with $X \convi$ and $X' \cnvalong{ \trace} $
      implies the convergence property we are after.

      Now fix a set $ O \in \accht{ \serverB' }{ \trace }$.
      Thanks to the transition $ \serverB \st{ \mu } \serverB'$,
      we know that $ O \in \accht{ \serverB }{ \mu.\trace }$.
      The hypothesis $ X \accleqset \serverB $ together with
      $ X \cnvalong \mu.\trace$ implies that there exists
      a server $\attaboy \in X$ such that there exists an
      $ \widehat{O} \in \accht{ \attaboy }{ \mu.\trace }$.
      This means that $ \attaboy \wt{ \mu } \attaboy'$ and that
      $ \widehat{O} \in \accht{ \attaboy' }{ \trace } $.
      Since $X \wt{\mu} X'$ we know that $\attaboy' \in X' $
      and this concludes the argument.
    %%%%%% OLD PROOF
    %%%%%%
  %% We first prove \rpt{bhvleqtwo-preserved-by-tau}.
  %% We must show $X \accleqset \serverB'$.
  %% Let us fix a server $\serverB''$ such that
  %% $X \cnvalong \trace$ and $\serverB' \wt{\trace} \serverB'' \stable$.
  %% An application of the hypothesis $X \accleqset \serverB$
  %% provides us with two servers $\serverA' \in X$ and $\serverA''$ such that
  %% $\serverA' \wt{s} \serverA'' \stable$ and $O(\serverA'') \subseteq O(\serverB'')$
  %% as required.


  %% We now prove \rptlem{bhvleqtwo-preserved}{bhvleqtwo-preserved-by-mu}.
  %% Let us fix a server $\serverB''$ such that
  %% $X' \cnvalong \trace$ and $\serverB' \wt{\trace} \serverB'' \stable$.
  %% An application of the hypothesis $X \accleqset \serverB$
  %% requires us to show $X \cnvalong \mu.\trace$ which follows
  %% from $X \convi$ and $X' \cnvalong \trace$.
  %% We then obtain two servers $\serverA \in X$ and $\serverA''$ such that
  %% $\serverA \wt{\mu.\trace} \serverA'' \stable$ and $O(\serverA'') \subseteq O(\serverB'')$
  %% We decompose the reduction $\serverA \wt{\mu.\trace} \serverA''$ and obtain
  %% a server $\serverA'$ such that
  %% $\serverA \wt{\mu} \serverA' \wt{\trace} \serverA''$ and allows us
  %% to conclude this case.
  \end{proof}


\begin{lemma}
  \label{lem:ungood-cnv-mu}
  For every $\genlts_\StatesA \in \obaFW$, $\genlts_\StatesB \in \obaFB$,
  every set of processes $X \in \pparts{ \StatesA }$, every $\client \in \StatesB$, and every $\mu \in \Act$,
  if $\mustset{X}{\client}$, $\lnot \good{\client}$ and $\client \st{\mu}$ then $X \cnvalong \co{\mu}$.
\end{lemma}

\noindent%
\textbf{\rlem{soundness-set}}
Let $\genlts_\StatesA, \genlts_\StatesB \in \obaFW$ and
$\genlts_\StatesC \in \obaFB$.
For every set of processes $X \in \pparts{ \StatesA }$,
server $\serverB \in \StatesB$ and client $\client \in \StatesC$,
if $\mustset{X}{\client}$ and $X \asleqset \serverB$ then
$\musti{\serverB}{\client}$.
\begin{proof}
  We proceed by induction on the derivation of $\mustset{X}{\client}$.
  In the base case, $\good{\client}$ so we trivially derive
  $\musti{\serverB}{\client}$. %
  %
  %
  In the inductive case the proof of the hypothesis  $\mustset{X}{\client}$
  terminates with an application of \rname{Mset-step}.
  Since $\lnot \good{\client}$, we show the result applying \mstep.
  This requires us to prove that
  \begin{enumerate}[(1)]
  \item $ \csys{\serverB}{\client} \st{ \tau }$, and that
  \item for all $\serverB', \client'$ such that $\csys{\serverB}{\client} \st{ \tau } \csys{\serverB'}{\client'} $
    we have $\musti{\serverB'}{\client'} $.
  \end{enumerate}
  The first fact is a consequence of \rlem{stability-Nbhvleqtwo},
  which we can apply because $\lnot \good{\client}$ and thanks to the
  hypothesis $X \accleqset q$ and $\mustset{X}{\client}$.
  To prove the second fact, fix a transition $
  \csys{\serverB}{\client} \st{ \tau } \csys{\serverB'}{\client'}$. We
  have to explain why the following properties are true,
  \begin{enumerate}[(a)]
  \item\label{pt:aim-soundness-1}
    for every
    $\serverB' \suchthat \serverB \st{\tau} \serverB' \implies
    \musti{\serverB'}{\client}$,
  \item\label{pt:aim-soundness-2}
    for every
    $\client \suchthat \client \st{\tau} \client' \implies
    \musti{\serverB}{\client'}$,
  \item\label{pt:aim-soundness-3}
    for every $\serverB', \client'$ and $\mu \in \Act$,
    $\serverB \st{\mu} \serverB'$ and
    $\client \st{\co{\mu}} \client'$ imply
    $\musti{\serverB'}{\client'}$.
  \end{enumerate}

  First, note that $\mustset{ X  }{ \client }$, $\lnot
  \good{\client}$, and \rlem{mustx-terminate-ungood} imply~$X \conv$. Second, the inductive hypotheses
  state that for every $\client'$, non-empty set $X'$, and $\serverB$ the following facts hold,
  \begin{enumerate}[(i)]
  \item\label{soundness-IH1}
    $X \st{\tau} X'$ and $X' \asleqset \serverB$ implies
    $\musti{\serverB}{\client}$,
  \item\label{soundness-IH2}
    $\client \st{\tau} \client'$ and $X \asleqset \serverB$ implies
    $\musti{\serverB}{\client'}$,
  \item\label{soundness-IH3}
    for every and $\mu \in \Act$,
    $X \wt{\co{\mu}} X'$ and $\client \st{\mu} \client'$, and $X' \asleqset \serverB$ implies
    $\musti{\serverB}{\client'}$.
  \end{enumerate}


  To prove (\ref{pt:aim-soundness-1}) we use $X \conv$ and the hypothesis
  $X \cnvleqset q$ to obtain~$q \convi$. A rule induction on $q
  \convi$ now suffices: in the base case (\ref{pt:aim-soundness-1}) is
  trivially true and in the inductive case (\ref{pt:aim-soundness-1}) follows from
  \rptlem{bhvleqone-preserved}{bhvleqone-preserved-by-tau} and
  \rptlem{bhvleqtwo-preserved}{bhvleqtwo-preserved-by-tau},
  and the inductive hypothesis.

  The requirement (\ref{pt:aim-soundness-2}) follows directly from
  the hypothesis $X \asleqset \serverB$ and
  %an application of the inductive hypothesis
  part (\ref{soundness-IH2}) of the inductive hypothesis.


  To see why (\ref{pt:aim-soundness-3}) holds, fix an action $\mu$ such that $\serverB \st{\mu} \serverB'$
  and $\client \st{\co{\mu}} \client'$.
  %    \pl{From an application of \rlem{ungood-cnv-mu} with obtain $ X \cnvalong \mu $.}
  Since $\lnot \good{ \client }$ \rlem{ungood-cnv-mu} implies that $ X \cnvalong \mu $,
  %    Either $ X \cnvalong \mu $ or not.
  and so \rlem{X-must-perform-visible-action} proves that $X \wt{ \mu }$.
  In turn this implies that there exists a $X'$ such that $X \wt{\mu} X'$, and thus
  \rptlem{bhvleqone-preserved}{bhvleqone-preserved-by-mu} and
  \rptlem{bhvleqtwo-preserved}{bhvleqtwo-preserved-by-mu}
  prove that $X' \asleqset \serverB'$ holds, and
  (\ref{soundness-IH3}) ensures the result, \ie that $\musti{\serverB'}{\client'}$.
  %% If $\lnot{  X \cnvalong \mu }$ then, since $X \convi $, it must be the case that
  %% $X \wt{ \mu } X' $ for some $X'$ that diverges. The assumption that
  %% $\client \st{\co{\mu}} \client'$ now imply that $ \csys{ X }{ r } \wt{ } \csys{ X' }{ r' }$.
    %% Since $X'$ diverges the hypothesis $ \musti{X}{\client}$ imply that $\good{r'}$,
    %% and thus know via the axiom that $\musti{\serverB'}{\client'}$.
    %
    %
  %%%%% ORIGINAL PROOF DO NOT EREASE
  %% To see why (\ref{pt:aim-soundness-3}) holds, fix an action $\mu$ such that $\serverB \st{\mu} \serverB'$
  %% and $\client \st{\co{\mu}} \client'$.
  %% Consider the set $X' = \setof{\serverA'}{\exists \serverA \in X. \serverA \wt{\mu} \serverA'}$
  %% such that for every $\serverA \in X$, if $\serverA \wt{\mu} \serverA'$ then $\serverA' \in X'$.
  %% We know by construction $X \wt{\mu} X'$.
  %% We distinguish whether $X'$ is empty or not.
  %% If $X'$ is empty then $X \Nwt{\mu}$ and $\lnot (X \accleqset \serverB)$ follows from
  %% an application of \rlem{X-must-perform-visible-action}.
  %% The hypothesis $X \accleqset \serverB$ together with $\lnot (X \accleqset \serverB)$ allows us
  %% to derive a proof of $\bot$ which concludes this case.
  %% If $X'$ is not empty then we have $X \wt{\mu} X'$.
  %% As the $X \wt{\mu} X'$ %set $X'$ is complete with respect to $X$ and the action $\mu$,
  %% \rptlem{bhvleqone-preserved}{bhvleqone-preserved-by-mu} and
  %% \rptlem{bhvleqtwo-preserved}{bhvleqtwo-preserved-by-mu}
  %% prove that $X' \asleqset \serverB'$ holds, and so
  %% (\ref{soundness-IH3}) ensures the result, \ie that $\musti{\serverB'}{\client'}$.
  %%%%
\end{proof}
  %% gio: I HAVEN'T UNDERSTOOD ANYTHING OF THE FOLLOWING ENGLISH
  %% Since the LTS is finite-image and $X \cnvalong{\mu}$
  %% we know that $X'$ is finite.
  %% \rptlem{bhvleqtwo-preserved}{empty-nleqx} together with the hypothesis $X \accleqset \serverB$
  %% imply that $X' \neq \emptyset$.
