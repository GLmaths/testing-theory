\section{Selinger axiom for asynchrony}

%% \begin{figure*}[t]
%% \hrulefill\\
%% \begin{tabular}{c@{\hskip 20pt}c@{\hskip 20pt}c@{\hskip 20pt}c}
%% %\begin{framed}[16em]
%% %  \begin{minipage}{60pt}
%%     \begin{tabular}{l@{\hskip 4pt}c@{\hskip 0pt}l}
%%     \begin{tikzcd}
%%       p \arrow[r, "\co{\aa}"]
%%       &
%%       p' \arrow[d, "\alpha"]
%%       \\
%%       &
%%       q
%%     \end{tikzcd}
%%     &$\Rightarrow$&
%%     \begin{tikzcd}
%%       p
%%       \arrow[r, "\co{\aa}"]
%%       \arrow[d, "\alpha"]
%%       &
%%       p' \arrow[d, "\alpha"]
%%       \\
%%       p'' \arrow[r, "\co{\aa}"]
%%       &
%%       q
%%     \end{tikzcd}
%%     \end{tabular}
%% %    \end{minipage}
%% %\end{framed}
%% &
%% %\begin{framed}[16em]%\outputconfluencediagram
%% \begin{tikzcd}
%% p \arrow[r, "\co{\aa}"]
%% &
%% p' \arrow[d, "\aa"]
%% \\
%% &
%% q
%% \end{tikzcd}
%% $\Rightarrow$
%% \begin{tikzcd}
%% p
%% \arrow[r, "\co{\aa}"]
%% \arrow[rd, "\tau"]
%% &
%% p' \arrow[d, "\aa"]
%% \\
%% &
%% q
%% \end{tikzcd}
%% \\
%% \outputcommutativity  &  \outputfeedback
%% \\[10pt]
%% \begin{tikzcd}
%% p \arrow[r, "\co{\aa}"]
%% \arrow[d, "\co{\aa}"]
%% &
%% p'
%% \\
%% p''
%% \end{tikzcd}
%% $\Rightarrow p' = p''$
%% &
%% \begin{tikzcd}
%% p \arrow[r, "\co{\aa}"]
%% \arrow[d, "\tau"]
%% &
%% p'
%% \\
%% p''
%% &
%% \end{tikzcd}
%% $\Rightarrow$
%% \begin{tikzcd}
%% p
%% \arrow[r, "\co{\aa}"]
%% \arrow[d, "\tau"]
%% &
%% p'
%% \arrow[d, "\tau"]
%% \\
%% p''
%% \arrow[r, "\co{\aa}"]
%% &
%% q
%% \end{tikzcd}
%% \,\, or
%% \begin{tikzcd}
%% p
%% \arrow[r, "\co{\aa}"]
%% \arrow[d, "\tau"]
%% &
%% p'
%% \arrow[ld, "\aa"]
%% \\
%% p''
%% \end{tikzcd}
%% &

%% \\[5pt]
%% \outputdeterminacy & \outputtau
%% \\[10pt]
%% \begin{tikzcd}
%% p \arrow[r, "\co{\aa}"]
%% \arrow[d, "\alpha"]
%% &
%% p'
%% \\
%% p''
%% &
%% \end{tikzcd}
%% $\Rightarrow$
%% \begin{tikzcd}
%% p
%% \arrow[r, "\co{\aa}"]
%% \arrow[d, "\alpha"]
%% &
%% p' \arrow[d, "\alpha"]
%% \\
%% p'' \arrow[r, "\co{\aa}"]
%% &
%% q
%% \end{tikzcd}
%% &
%% \begin{tikzcd}
%% &
%% p'
%% \arrow[d, "\co{\aa}"]
%% \\
%% p''\arrow[r, "\co{\aa}"]
%% &
%% p
%% \end{tikzcd}
%% $\Rightarrow p' = p''$
%% \\
%% where $\alpha \neq \co{\aa}$ and $\alpha \neq \tau$
%% \\
%% \outputconfluence & \outputdeterminacyinv
%% \end{tabular}
%% \caption{
%% First-order axioms for output-buffered agents with feedback as given by   \cite%[Table 5]
%%   {DBLP:conf/concur/Selinger97}, plus the \outputdeterminacyinv axiom.}
%% %\caption{Two of the axioms by \cite{DBLP:conf/concur/Selinger97} for
%% %  asynchronous out-buffered agents.}
%% \label{fig:axioms}
%% \hrulefill
%% \end{figure*}


%%%%%%%%%%%%%%%%%%%%%%%%%%%%%%%%%%%%%%%%%%%%%%%
\paragraph{A closer look at the Selinger axioms}
We now discuss the axioms in \rfig{axioms}.
%Let us now discuss the axioms in Figure~\ref{fig:axioms}.
%As mentioned already,
The \outputcommutativity axiom expresses the non-blocking behaviour of
outputs: %it says that
an output cannot be a cause of any subsequent transition, since it can
also be executed after it, leading to the same resulting state. Hence,
outputs are concurrent with any subsequent transition.  The
\outputfeedback axiom says that an output followed by a complementary
input can also synchronise with it to produce a $\tau$-transition
(since the output and the input are in fact concurrent). These first
two axioms specify properties of outputs that are followed by another
transition. Instead, the following three axioms, \outputconfluence,
\outputdeterminacy and \outputtau, specify properties of outputs that
are co-initial with another transition\footnote{Two transitions are
  co-initial if they stem from the same state.}. The
\outputdeterminacy and \outputtau axioms apply to the case where the
co-initial transition is an identical output or a $\tau$-transition
respectively, while the \outputconfluence axiom applies to the other
cases.  When taken in conjunction, these three axioms state that outputs
cannot be in conflict %(i.e., mutually exclusive)
with any co-initial transition, except when this is a
$\tau$-transition: in this case, the \outputtau axiom allows for a
confluent nondeterminism between the $\tau$-transition on one side and
the output followed by the complementary input on the other side,
since the $\tau$-transition may represent a synchronisation between the
output and the input.

Since the axiom \outputdeterminacyinv is not part of Selinger
axioms, we discuss it in the next paragraph.



We now explain the novel
\outputdeterminacyinv axiom.
It is the dual of \outputdeterminacy, as it states that
also backward transitions with identical outputs lead to the same
state. The intuition is that if two programs, by removing the
same message from the mailbox, arrive exactly at the same
state, then
they must coincide. This seems natural if communication takes place
via a shared mailbox.
This axiom need not be assumed in \cite{DBLP:conf/concur/Selinger97}
because it can be derived from Selinger axioms when modelling a calculus like~\ACCS
equipped with a parallel composition operator~$\parallel$ (see
Lemma~\ref{lem:output-shape} in Appendix~\ref{sec:accs}).
We use the \outputdeterminacyinv axiom only to prove a technical property of clients (\rlem{inversion-gen-mu}) that we use to prove our completeness result.
%%%
%%% gio (27/12/2023) the next sentence sounds like the axiom is
%%% entirely ad-hoc: we added it just to make the proofs go thourgh
%%%
%On the other hand, it is needed in our calculus-independent setting, where
%such a parallel operator on LTSs is not given.


%% \ila{We may now explain why we added the new axiom
%%   \outputdeterminacyinv axiom
%%   \ilacom{a better name would be ``backward output determinacy''}
%%   to Selinger ones in \rfig{axioms}.
%%   Note that this axiom is the dual of \outputdeterminacy, as it states that
%%   also backward transitions with identical outputs lead to the same
%%   state. This axiom was not assumed by Selinger because it can be
%%   derived from his axioms when modelling a calculus like \ACCS
%%   equipped with a parallel composition operator $\parallel$ (see
%%   Lemma~\ref{lem:output-shape} in Appendix~\ref{sec:accs}). On the
%%   other hand, it is needed in our calculus-independent setting, where
%%   such a parallel operator on LTSs is not given.
%%   %%%}
